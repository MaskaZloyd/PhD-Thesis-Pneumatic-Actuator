%%% Основные сведения %%%
\newcommand{\thesisAuthorLastName}{Шейкин}
\newcommand{\thesisAuthorOtherNames}{Максим Олегович}
\newcommand{\thesisAuthorInitials}{М.\,О.}
\newcommand{\thesisAuthor}             % Диссертация, ФИО автора
{%
    \texorpdfstring{% \texorpdfstring takes two arguments and uses the first for ()TeX and the second for pdf
        \thesisAuthorLastName~\thesisAuthorOtherNames% так будет отображаться на титульном листе или в тексте, где будет использоваться переменная
    }{%
        \thesisAuthorLastName, \thesisAuthorOtherNames% эта запись для свойств pdf-файла. В таком виде, если pdf будет обработан программами для сбора библиографических сведений, будет правильно представлена фамилия.
    }
}
\newcommand{\thesisAuthorShort}        % Диссертация, ФИО автора инициалами
{\thesisAuthorInitials~\thesisAuthorLastName}
%\newcommand{\thesisUdk}                % Диссертация, УДК
%{\fixme{xxx.xxx}}
\newcommand{\thesisTitle}              % Диссертация, название
{Повышение статико-динамических качеств и оптимизация рабочего процесса позиционного пневмопривода с дискретными распределителями}
\newcommand{\thesisSpecialtyNumber}    % Диссертация, специальность, номер
{2.5.10}
\newcommand{\thesisSpecialtyTitle}     % Диссертация, специальность, название (название взято с сайта ВАК для примера)
{Гидравлические машины, вакуумная, компрессорная техника, гидро-~и~пневмосистемы}
%% \newcommand{\thesisSpecialtyTwoNumber} % Диссертация, вторая специальность, номер
%% {\fixme{XX.XX.XX}}
%% \newcommand{\thesisSpecialtyTwoTitle}  % Диссертация, вторая специальность, название
%% {\fixme{Теория и~методика физического воспитания, спортивной тренировки,
%% оздоровительной и~адаптивной физической культуры}}
\newcommand{\thesisDegree}             % Диссертация, ученая степень
{кандидат технических наук}
\newcommand{\thesisDegreeShort}        % Диссертация, ученая степень, краткая запись
{\fixme{канд. техн. наук}}
\newcommand{\thesisCity}               % Диссертация, город написания диссертации
{Москва}
\newcommand{\thesisYear}               % Диссертация, год написания диссертации
{\the\year}
\newcommand{\thesisOrganization}       % Диссертация, организация
{Федеральное государственное бюджетное образовательное учреждение высшего образования <<Национальный исследовательский университет <<МЭИ>>}
\newcommand{\thesisOrganizationShort}  % Диссертация, краткое название организации для доклада
{ФГБОУ ВО <<НИУ <<МЭИ>>}

\newcommand{\thesisInOrganization}     % Диссертация, организация в предложном падеже: Работа выполнена в ...
{Федеральном государственном бюджетном образовательном учреждении высшего образования «Национальный исследовательский университет «МЭИ»}

%% \newcommand{\supervisorDead}{}           % Рисовать рамку вокруг фамилии
\newcommand{\supervisorFio}              % Научный руководитель, ФИО
{Черкасских Сергей Николаевич}
\newcommand{\supervisorRegalia}          % Научный руководитель, регалии
{кандидат технических наук, доцент}
\newcommand{\supervisorFioShort}         % Научный руководитель, ФИО
{С.\,Н.~Черкасских}
\newcommand{\supervisorRegaliaShort}     % Научный руководитель, регалии
{канд.~техн.~наук, доцент}

%% \newcommand{\supervisorTwoDead}{}        % Рисовать рамку вокруг фамилии
%% \newcommand{\supervisorTwoFio}           % Второй научный руководитель, ФИО
%% {\fixme{Фамилия Имя Отчество}}
%% \newcommand{\supervisorTwoRegalia}       % Второй научный руководитель, регалии
%% {\fixme{уч. степень, уч. звание}}
%% \newcommand{\supervisorTwoFioShort}      % Второй научный руководитель, ФИО
%% {\fixme{И.\,О.~Фамилия}}
%% \newcommand{\supervisorTwoRegaliaShort}  % Второй научный руководитель, регалии
%% {\fixme{уч.~ст.,~уч.~зв.}}

\newcommand{\opponentOneFio}           % Оппонент 1, ФИО
{Труханов Кирилл Алексеевич}
\newcommand{\opponentOneRegalia}       % Оппонент 1, регалии
{доктор технических наук, профессор}
\newcommand{\opponentOneJobPlace}      % Оппонент 1, место работы
{\fixme{Основное место работы}}
\newcommand{\opponentOneJobPost}       % Оппонент 1, должность
{\fixme{Должность}}

\newcommand{\opponentTwoFio}           % Оппонент 2, ФИО
{\fixme{Фамилия Имя Отчество}}
\newcommand{\opponentTwoRegalia}       % Оппонент 2, регалии
{\fixme{Регалии}}
\newcommand{\opponentTwoJobPlace}      % Оппонент 2, место работы
{\fixme{Основное место работы}}
\newcommand{\opponentTwoJobPost}       % Оппонент 2, должность
{\fixme{Должность}}

%% \newcommand{\opponentThreeFio}         % Оппонент 3, ФИО
%% {\fixme{Фамилия Имя Отчество}}
%% \newcommand{\opponentThreeRegalia}     % Оппонент 3, регалии
%% {\fixme{кандидат физико-математических наук}}
%% \newcommand{\opponentThreeJobPlace}    % Оппонент 3, место работы
%% {\fixme{Основное место работы c длинным длинным длинным длинным названием}}
%% \newcommand{\opponentThreeJobPost}     % Оппонент 3, должность
%% {\fixme{старший научный сотрудник}}

\newcommand{\leadingOrganizationTitle} % Ведущая организация, дополнительные строки. Удалить, чтобы не отображать в автореферате
{Акционерное общество <<Центральный научно-исследовательский институт автоматики и гидравлики>>}

\newcommand{\defenseDate}              % Защита, дата
{\fixme{06~сентября~2025~г.~в~11~часов~30~минут}}
\newcommand{\defenseCouncilNumber}     % Защита, номер диссертационного совета
{МЭИ.131}
\newcommand{\defenseCouncilTitle}      % Защита, учреждение диссертационного совета
{ФГБОУ~ВО~<<НИУ~<<МЭИ>>}
\newcommand{\defenseCouncilAddress}    % Защита, адрес учреждение диссертационного совета
{111250,~г.~Москва,~Красноказарменная~ул.,~д.~17,~корп.~Г,~ауд.~Г-102а}
\newcommand{\defenseCouncilPhone}      % Телефон для справок
{\fixme{+7~(0000)~00-00-00}}

\newcommand{\defenseSecretaryFio}      % Секретарь диссертационного совета, ФИО
{Лямасов Александр Константинович}
\newcommand{\defenseSecretaryRegalia}  % Секретарь диссертационного совета, регалии
{канд. техн. наук}            % Для сокращений есть ГОСТы, например: ГОСТ Р 7.0.12-2011 + http://base.garant.ru/179724/#block_30000

\newcommand{\synopsisLibrary}          % Автореферат, название библиотеки
{<<НИУ <<МЭИ>>}
\newcommand{\synopsisDate}             % Автореферат, дата рассылки
{\fixme{20~сентября~}\the\year~года}

% To avoid conflict with beamer class use \providecommand
\providecommand{\keywords}%            % Ключевые слова для метаданных PDF диссертации и автореферата
{}
