%% Согласно ГОСТ Р 7.0.11-2011:
%% 5.3.3 В заключении диссертации излагают итоги выполненного исследования, рекомендации, перспективы дальнейшей разработки темы.
%% 9.2.3 В заключении автореферата диссертации излагают итоги данного исследования, рекомендации и перспективы дальнейшей разработки темы.
В результате проведенного диссертационного исследования достигнута поставленная
цель комплексного повышения статико-динамических и ресурсных показателей позиционного пневмопривода с
дискретными распределителями в условиях их конфликтности. Получены следующие основные результаты:

\begin{enumerate}
    \item Разработана комплексная математическая модель позиционного пневмопривода с дискретными распределителями,
    которая включает в себя как силовую, так и управляющую структуры. Модель учитывает нелинейные термодинамические
    процессы в полостях пневмоцилиндра, переменную структуру системы при переключении распределителей, а также
    динамические эффекты сухого и вязкого трения. Высокая степень согласованности результатов моделирования с
    экспериментальными данными (расхождение не превышает 5~--~7\%) подтверждает адекватность и достоверность
    разработанной модели для решения задач анализа и синтеза пневмоприводов различной структуры.

    \item Проведен комплексный анализ различных структур пневмопривода, что позволило выявить оптимальные
    режимы переключения распределителей при позиционировании рабочего органа. Разработаны структуры управления
    на основе различных принципов: модифицированная структура ПИД-регулятора с блоком прогнозирования тормозного пути,
    управление в скользящих режимах с интегральной и терминальной поверхностями, нечеткое управление и прогнозное управление.
    Экспериментальный анализ подтвердил, что прогнозное управление обеспечивает повышение точности позиционирования на 25~--~30\%
    при одновременном сокращении числа переключений распределителей на 35~--~45\% по сравнению с традиционными алгоритмами.

    \item Проведен натурный эксперимент на специально разработанном лабораторном стенде, который подтвердил работоспособность
    предложенных структур и высокую достоверность результатов, полученных с использованием разработанной математической модели.
    Сравнительный анализ экспериментальных данных показал, что наибольшую точность моделирования демонстрируют алгоритмы прогнозного
    управления и семирежимного управления в скользящих режимах, где средняя относительная ошибка не превышает 4,65\%.
    Расхождение между расчетными и экспериментальными данными по основным показателям качества составляет
    не более 7\% для всех исследованных алгоритмов управления.

    \item Разработана методика многокритериального структурно-параметрического синтеза, основанная
    на использовании фронтов Парето с применением замещающей суррогатной нейросетевой модели. Для
    формирования обучающей выборки применен метод латинского гиперкуба, обеспечивающий оптимальное
    заполнение пространства параметров, что позволило сократить вычислительные затраты при построении
    фронтов Парето на 48\%. Использование суррогатной модели обеспечило среднюю точность аппроксимации 91\%
    при максимальном отклонении 12\%.

    \item Определена взаимосвязь между конфликтными статико-динамическими и ресурсными показателями
    для позиционных пневмоприводов различной структуры на основе анализа фронтов Парето. Установлено,
    что для ПИД-регулятора с ШИМ диапазон изменения критерия $ITAE$ составляет от 0,0298 до 0,0427 \si{\meter\per\second\square}
    при изменении точности от 0,41 до 2,57 мм. Семирежимное управление с интегральной поверхностью
    обеспечивает точность 0,13 мм при $ITAE = 0,019$ \si{\meter\per\second\square} и $SI = 15$, а прогнозное управление демонстрирует
    точность 0,12 мм при $ITAE = 0,032$ \si{\meter\per\second\square} и $SI = 14$~--~$16$. Для нечеткого регулятора характерны
    наилучшие показатели по критерию динамической эффективности ($ITAE = 0,006$ \si{\meter\per\second\square}) при точности 0,40 мм.

    \item Сформированы практические рекомендации по выбору оптимальной структуры позиционного
    пневмопривода с дискретными распределителями. Выделены четыре предпочтительные области
    применения различных алгоритмов управления:
    \begin{itemize}
        \item область высоконагруженных систем (УСР-И-5, MPC) с точностью 0,12~--~0,36 мм;
        \item область точных манипуляторов (УСР-И-7) с точностью 0,14 мм;
        \item область промышленной автоматики (УСР-Т-5, УСР-Т-7) с точностью 0,13~--~0,28 мм;
        \item область простых систем (ПИД+ШИМ, УСР-И-3) с точностью 0,5~--~2,5 мм.
    \end{itemize}
    Определены рекомендуемые значения параметров алгоритмов управления для типовых задач
    позиционирования, что позволяет сократить сроки проектирования и повышает эффективность разрабатываемых систем.
\end{enumerate}

Полученные результаты позволяют на научной основе выбирать оптимальную структуру и
параметры позиционного пневмопривода с дискретными распределителями в зависимости
от конкретных требований, существенно сокращая сроки проектирования и повышая
эффективность разрабатываемых систем.