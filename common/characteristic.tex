
{\actuality} %Обзор, введение в тему, обозначение места данной работы в мировых исследованиях и~т.\:п.

\ifsynopsis
%Этот абзац появляется только в~автореферате.

\else
%Этот абзац появляется только в~диссертации.

\fi

% {\progress}
% Этот раздел должен быть отдельным структурным элементом по
% ГОСТ, но он, как правило, включается в описание актуальности
% темы. Нужен он отдельным структурынм элемементом или нет ---
% смотрите другие диссертации вашего совета, скорее всего не нужен.

{\aim}  диссертационной работы является повышение эффективности электропневматических приводов с
дискретными распределителями путем анализа и оптимизации различных алгоритмов управления.
Работа направлена на разработку методики выбора оптимального алгоритма управления для пневмоприводов
с учетом специфических условий эксплуатации и множества показателей качества, таких как точность позиционирования,
частота переключений, и других релевантных показателей.

Предлагаемая методика основывается на многокритериальной оптимизации с использованием фронтов Парето
и суррогатных моделей, что позволяет обеспечить баланс между различными требованиями к качеству работы пневмопривода.

Для~достижения поставленной цели необходимо было решить следующие {\tasks}:
\begin{enumerate}[beginpenalty=10000] % https://tex.stackexchange.com/a/476052/104425
\item Разработать математические модели электропневматического привода с дискретными распределителями,
которые учитывают нелинейную динамику системы и дискретный характер управляющих воздействий для различных алгоритмов управления.

\item Исследовать и проанализировать эффективность различных алгоритмов управления пневмоприводом,
включая скользящее управление, управление с широтно-импульсной модуляцией (ШИМ),
интеллектуальные алгоритмы (нечеткая логика, нейронные сети и другие) и предложенный
алгоритм управления, с учетом их влияния на ключевые показатели качества работы привода.

\item Определить критерии оптимальности и ограничения для многокритериальной оптимизации
параметров различных алгоритмов управления, учитывая широкий спектр показателей качества,
таких как точность позиционирования, частота переключений, энергопотребление, надежность и быстродействие системы.

\item Разработать методику построения фронтов Парето для сравнительного анализа
эффективности различных алгоритмов управления, используя суррогатные модели для
снижения вычислительной сложности и объективной оценки компромиссов между показателями качества.

\item Провести численное моделирование работы пневмопривода под управлением различных алгоритмов,
включая предложенный алгоритм, и построить фронты Парето для оценки их эффективности в условиях реальных эксплуатационных требований.

\item Выполнить сравнительный анализ полученных фронтов Парето для определения областей эффективности каждого
алгоритма управления и выявить условия, при которых тот или иной алгоритм наиболее оптимален.

\item Разработать методику выбора оптимального алгоритма управления электропневматическим приводом,
ориентированную на конкретные эксплуатационные условия и требования к качеству работы системы, включая предложенный алгоритм.

\item Провести экспериментальные исследования работы электропневматического привода с различными
алгоритмами управления, в том числе с предложенным алгоритмом, для верификации полученных
численных результатов и подтверждения адекватности разработанных моделей и методик в реальных условиях.

Оценить робастность исследуемых алгоритмов управления и предложенной методики к изменению
параметров системы и ошибкам моделирования для обеспечения стабильной и
надежной работы пневмопривода в реальных эксплуатационных условиях.
\end{enumerate}


{\novelty}
\begin{enumerate}[beginpenalty=10000] % https://tex.stackexchange.com/a/476052/104425
    \item Разработана методика многокритериального сравнения различных алгоритмов управления электропневматическими
    приводами с дискретными распределителями на основе анализа фронтов Парето и суррогатных
    моделей, что позволяет объективно оценить компромиссы между
    различными показателями качества управления.
    \item Впервые проведен комплексный сравнительный анализ эффективности
    скользящего управления, управления с ШИМ, интеллектуальных алгоритмов (нечеткая логика, нейронные сети и другие)
    и предложенного алгоритма, что позволяет выявить условия эффективности каждого
    алгоритма в зависимости от эксплуатационных требований.
    \item Разработан новый алгоритм управления для электропневматических приводов, который демонстрирует
    преимущества по сравнению с традиционными методами, обеспечивая более эффективное
    сочетание точности позиционирования и частоты переключений.
    \item Предложен подход к выбору оптимального алгоритма управления на основе
    экспериментальных данных и численного моделирования, что позволяет адаптировать
    систему управления под конкретные условия эксплуатации, обеспечивая
    высокую робастность и надежность работы привода.
    \item Проведены экспериментальные исследования для верификации численных моделей и методик,
    что подтверждает их адекватность и применимость для реальных промышленных задач.
\end{enumerate}

{\influence} \ldots

{\methods} В работе использовался комплексный подход, включающий как теоретические, так и
экспериментальные методы исследования. Основу методологии составила многокритериальная оптимизация
алгоритмов управления электропневматическими приводами с дискретными распределителями на основе анализа
фронтов Парето и суррогатного моделирования. Теоретическая часть исследования включала разработку математических
моделей пневмопривода, описывающих его динамику и взаимодействие с различными алгоритмами управления, включая скользящее
управление, ШИМ, интеллектуальные алгоритмы (нечеткая логика, нейронные сети и другие), а также предложенный алгоритм.

Для снижения вычислительной сложности и ускорения процесса оптимизации использовались суррогатные модели, реализованные
в виде нейронных сетей, что позволило эффективно оценивать компромиссы между различными критериями качества управления,
такими как точность позиционирования, частота переключений и энергопотребление. На основе численного моделирования проводился
сравнительный анализ эффективности различных алгоритмов управления, строились фронты Парето, с помощью которых выявлялись
оптимальные области применения каждого алгоритма.

Экспериментальная часть исследования включала проведение испытаний на реальном электропневматическом приводе для верификации
разработанных моделей и методик. В ходе экспериментов проверялись основные положения теоретических расчетов, проводился
анализ переходных процессов и проверялась робастность алгоритмов управления к изменению параметров системы. Результаты
экспериментов использовались для коррекции моделей и уточнения рекомендаций по
выбору алгоритма управления в зависимости от условий эксплуатации. Такой подход обеспечил комплексную проверку
эффективности разработанных методов и их применимость для реальных промышленных задач.

{\defpositions}
\begin{enumerate}[beginpenalty=10000] % https://tex.stackexchange.com/a/476052/104425
    \item Разработанная методика многокритериального сравнения алгоритмов управления
    электропневматическими приводами с дискретными распределителями на основе фронтов Парето и
    суррогатного моделирования, позволяющая учитывать компромисс между различными показателями качества управления.
    \item Комплексный сравнительный анализ скользящего управления, управления с ШИМ, интеллектуальных
    алгоритмов (нечеткая логика, нейронные сети) и предложенного алгоритма, который выявил условия
    эффективности каждого алгоритма в зависимости от эксплуатационных требований.
    \item Новый алгоритм управления для электропневматических приводов, демонстрирующий улучшение сочетания
    точности позиционирования и частоты переключений по сравнению с традиционными методами.
    \item Разработанная методика выбора оптимального алгоритма управления на основе численного моделирования
    и экспериментальных данных, обеспечивающая адаптацию системы управления под конкретные условия эксплуатации.
    \item Результаты экспериментальных исследований, подтверждающие адекватность и применимость разработанных
    моделей и методик для решения практических задач управления электропневматическими приводами с дискретными распределителями.
\end{enumerate}

{\reliability} полученных результатов обеспечивается \ldots \ Результаты находятся в соответствии с результатами, полученными другими авторами.


{\probation}
Основные результаты работы докладывались~на:
перечисление основных конференций, симпозиумов и~т.\:п.

{\contribution} Автор принимал активное участие \ldots

\ifnumequal{\value{bibliosel}}{0}
{%%% Встроенная реализация с загрузкой файла через движок bibtex8. (При желании, внутри можно использовать обычные ссылки, наподобие `\cite{vakbib1,vakbib2}`).
    {\publications} Основные результаты по теме диссертации изложены
    в~XX~печатных изданиях,
    X из которых изданы в журналах, рекомендованных ВАК,
    X "--- в тезисах докладов.
}%
{%%% Реализация пакетом biblatex через движок biber
    \begin{refsection}[bl-author, bl-registered]
        % Это refsection=1.
        % Процитированные здесь работы:
        %  * подсчитываются, для автоматического составления фразы "Основные результаты ..."
        %  * попадают в авторскую библиографию, при usefootcite==0 и стиле `\insertbiblioauthor` или `\insertbiblioauthorgrouped`
        %  * нумеруются там в зависимости от порядка команд `\printbibliography` в этом разделе.
        %  * при использовании `\insertbiblioauthorgrouped`, порядок команд `\printbibliography` в нём должен быть тем же (см. biblio/biblatex.tex)
        %
        % Невидимый библиографический список для подсчёта количества публикаций:
        \printbibliography[heading=nobibheading, section=1, env=countauthorvak,          keyword=biblioauthorvak]%
        \printbibliography[heading=nobibheading, section=1, env=countauthorwos,          keyword=biblioauthorwos]%
        \printbibliography[heading=nobibheading, section=1, env=countauthorscopus,       keyword=biblioauthorscopus]%
        \printbibliography[heading=nobibheading, section=1, env=countauthorconf,         keyword=biblioauthorconf]%
        \printbibliography[heading=nobibheading, section=1, env=countauthorother,        keyword=biblioauthorother]%
        \printbibliography[heading=nobibheading, section=1, env=countregistered,         keyword=biblioregistered]%
        \printbibliography[heading=nobibheading, section=1, env=countauthorpatent,       keyword=biblioauthorpatent]%
        \printbibliography[heading=nobibheading, section=1, env=countauthorprogram,      keyword=biblioauthorprogram]%
        \printbibliography[heading=nobibheading, section=1, env=countauthor,             keyword=biblioauthor]%
        \printbibliography[heading=nobibheading, section=1, env=countauthorvakscopuswos, filter=vakscopuswos]%
        \printbibliography[heading=nobibheading, section=1, env=countauthorscopuswos,    filter=scopuswos]%
        %
        \nocite{*}%
        %
        {\publications} Основные результаты по теме диссертации изложены в~\arabic{citeauthor}~печатных изданиях,
        \arabic{citeauthorvak} из которых изданы в журналах, рекомендованных ВАК\sloppy%
        \ifnum \value{citeauthorscopuswos}>0%
            , \arabic{citeauthorscopuswos} "--- в~периодических научных журналах, индексируемых Web of~Science и Scopus\sloppy%
        \fi%
        \ifnum \value{citeauthorconf}>0%
            , \arabic{citeauthorconf} "--- в~тезисах докладов.
        \else%
            .
        \fi%
        \ifnum \value{citeregistered}=1%
            \ifnum \value{citeauthorpatent}=1%
                Зарегистрирован \arabic{citeauthorpatent} патент.
            \fi%
            \ifnum \value{citeauthorprogram}=1%
                Зарегистрирована \arabic{citeauthorprogram} программа для ЭВМ.
            \fi%
        \fi%
        \ifnum \value{citeregistered}>1%
            Зарегистрированы\ %
            \ifnum \value{citeauthorpatent}>0%
            \formbytotal{citeauthorpatent}{патент}{}{а}{}\sloppy%
            \ifnum \value{citeauthorprogram}=0 . \else \ и~\fi%
            \fi%
            \ifnum \value{citeauthorprogram}>0%
            \formbytotal{citeauthorprogram}{программ}{а}{ы}{} для ЭВМ.
            \fi%
        \fi%
        % К публикациям, в которых излагаются основные научные результаты диссертации на соискание учёной
        % степени, в рецензируемых изданиях приравниваются патенты на изобретения, патенты (свидетельства) на
        % полезную модель, патенты на промышленный образец, патенты на селекционные достижения, свидетельства
        % на программу для электронных вычислительных машин, базу данных, топологию интегральных микросхем,
        % зарегистрированные в установленном порядке.(в ред. Постановления Правительства РФ от 21.04.2016 N 335)
    \end{refsection}%
    \begin{refsection}[bl-author, bl-registered]
        % Это refsection=2.
        % Процитированные здесь работы:
        %  * попадают в авторскую библиографию, при usefootcite==0 и стиле `\insertbiblioauthorimportant`.
        %  * ни на что не влияют в противном случае
        \nocite{vakbib2}%vak
        \nocite{patbib1}%patent
        \nocite{progbib1}%program
        \nocite{bib1}%other
        \nocite{confbib1}%conf
    \end{refsection}%
        %
        % Всё, что вне этих двух refsection, это refsection=0,
        %  * для диссертации - это нормальные ссылки, попадающие в обычную библиографию
        %  * для автореферата:
        %     * при usefootcite==0, ссылка корректно сработает только для источника из `external.bib`. Для своих работ --- напечатает "[0]" (и даже Warning не вылезет).
        %     * при usefootcite==1, ссылка сработает нормально. В авторской библиографии будут только процитированные в refsection=0 работы.
}
