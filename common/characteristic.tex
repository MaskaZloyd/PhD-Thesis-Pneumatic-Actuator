
{\actuality}

\ifsynopsis
%Этот абзац появляется только в~автореферате.

\else
%Этот абзац появляется только в~диссертации.

\fi

% {\progress}
% Этот раздел должен быть отдельным структурным элементом по
% ГОСТ, но он, как правило, включается в описание актуальности
% темы. Нужен он отдельным структурынм элемементом или нет ---
% смотрите другие диссертации вашего совета, скорее всего не нужен.

{\aim} 
диссертационной работы
разработка методики многокритериального сравнительного анализа алгоритмов дискретного
управления электропневматическим приводом на основе построения и
исследования фронтов Парето, обеспечивающей научно обоснованный выбор
оптимального алгоритма управления в соответствии с заданными эксплуатационными требованиями.

Предлагаемая методика основывается на многокритериальной оптимизации с использованием фронтов Парето
и суррогатных моделей, что позволяет обеспечить баланс между различными требованиями к качеству работы пневмопривода.

Для~достижения поставленной цели необходимо было решить следующие {\tasks}:
\begin{enumerate}[beginpenalty=10000] % https://tex.stackexchange.com/a/476052/104425
\item Разработать математические модели электропневматического привода с дискретными распределителями,
которые учитывают нелинейную динамику системы и дискретный характер управляющих воздействий для различных алгоритмов управления.

\item Проведение анализа существующих алгоритмов дискретного управления электропневматическим приводом и
методов их оптимизации для выявления основных характеристик и закономерностей функционирования системы

\item Разработка математических моделей и критериев оценки эффективности электропневматического привода с дискретными
распределителями, учитывающих как качество позиционирования, так и интенсивность переключений распределителей 
и остальных интересующих показателей качества.

\item Синтез и исследование различных алгоритмов дискретного управления электропневматическим приводом,
включая управление в скользящих режимах, ПИД-регулирование с ШИМ, нечеткое управление и прогнозное управление.

\item Разработка методов построения нейросетевых суррогатных моделей для аппроксимации зависимостей между
параметрами алгоритмов управления и показателями качества функционирования электропневматического привода.

\item Создание алгоритмов построения и анализа фронтов Парето для различных методов
управления с использованием суррогатных моделей и эволюционных алгоритмов оптимизации.

\item Разработка программно-аппаратного комплекса для
экспериментальных исследований электропневматического привода и верификации теоретических результатов.

\item Проведение сравнительного анализа эффективности различных алгоритмов управления на основе построенных фронтов Парето и
формирование рекомендаций по их практическому применению в зависимости от требований к системе.

Решение данных задач позволило создать комплексную методику многокритериального анализа и
обоснованного выбора алгоритмов управления электропневматическим приводом.
\end{enumerate}


{\novelty}
\begin{enumerate}[beginpenalty=10000] % https://tex.stackexchange.com/a/476052/104425
    \item Разработан новый подход к сравнительному анализу алгоритмов дискретного управления
    электропневматическим приводом, основанный на построении и исследовании фронтов Парето с
    использованием нейросетевых суррогатных моделей, что позволяет осуществлять научно
    обоснованный выбор алгоритма управления в соответствии с заданными требованиями к системе.

    \item Предложена модифицированная архитектура нейросетевой суррогатной модели на основе
    остаточных блоков для аппроксимации зависимостей между
    параметрами алгоритмов управления и показателями качества функционирования
    электропневматического привода, обеспечивающая повышенную точность
    прогнозирования и устойчивость к шуму в данных.

    \item Разработаны новые критерии оценки эффективности алгоритмов управления
    электропневматическим приводом, учитывающие как точность позиционирования, так
    и интенсивность переключений распределителей, что позволяет
    проводить комплексную оценку качества функционирования системы.

    \item Создан новый метод построения и анализа фронтов Парето для алгоритмов
    управления электропневматическим приводом, основанный на комбинации эволюционных
    алгоритмов оптимизации и суррогатного моделирования, что позволяет существенно сократить
    вычислительные затраты при сохранении точности результатов.

    \item Проведены экспериментальные исследования для верификации численных моделей и методик,
    что подтверждает их адекватность и применимость для реальных промышленных задач.
\end{enumerate}

{\influence} \ldots

{\methods} В рамках диссертационного исследования применялись теоретические и экспериментальные методы.
Теоретическая часть работы базируется на фундаментальных положениях теории автоматического 
управления, математического моделирования и оптимизации. При разработке математических моделей
электропневматического привода использовались методы термодинамики и механики сплошных
сред. Анализ динамики системы осуществлялся с применением методов качественной теории 
дифференциальных уравнений и теории устойчивости. Синтез алгоритмов управления проводился
с использованием теории скользящих режимов, нечеткой логики и прогнозного управления.

Для построения суррогатных моделей применялись методы глубокого обучения и нейросетевого
моделирования. Многокритериальная оптимизация параметров алгоритмов управления осуществлялась
с использованием эволюционных алгоритмов и теории Парето-оптимальности. Статистическая обработка
результатов выполнялась методами математической статистики и регрессионного анализа.

Экспериментальные исследования проводились на специально разработанном
лабораторном стенде с использованием современных средств измерения и
обработки данных. Верификация теоретических результатов осуществлялась
путем сравнительного анализа расчетных и экспериментальных данных.

Программная реализация алгоритмов управления и методов оптимизации
выполнена с использованием современных языков программирования Python и
Julia, а также специализированных библиотек для научных вычислений и машинного
обучения. Визуализация результатов осуществлялась с применением
современных графических средств и методов многомерного анализа данных.

Достоверность полученных результатов обеспечивается корректным применением
апробированных научных методов, согласованностью теоретических и экспериментальных
данных.

{\defpositions}
\begin{enumerate}[beginpenalty=10000] % https://tex.stackexchange.com/a/476052/104425

\item Методика многокритериального сравнительного анализа алгоритмов дискретного управления
электропневматическим приводом, основанная на построении и исследовании фронтов Парето
с использованием нейросетевых суррогатных моделей, обеспечивающая научно обоснованный
выбор оптимального алгоритма управления для заданных эксплуатационных требований.

\item Модифицированная архитектура нейронной сети с остаточными блоками для построения суррогатных моделей
электропневматического привода, позволяющая с высокой точностью
аппроксимировать зависимости между параметрами алгоритмов управления
и показателями качества функционирования системы при существенном сокращении вычислительных затрат.

\item Комплекс критериев оценки эффективности алгоритмов управления электропневматическим
приводом, учитывающий как точность позиционирования, так и интенсивность
переключений распределителей, позволяющий проводить всесторонний анализ качества
функционирования системы в различных режимах работы.

\item Метод построения и анализа фронтов Парето для алгоритмов управления электропневматическим
приводом на основе комбинации эволюционных алгоритмов оптимизации и суррогатного моделирования,
обеспечивающий существенное сокращение вычислительных затрат при сохранении точности результатов.

\item Результаты экспериментальных исследований, подтверждающие эффективность разработанной
методики многокритериального анализа и практические рекомендации по выбору алгоритмов
управления электропневматическим приводом в зависимости от требований к системе.

\end{enumerate}

{\reliability} 
полученных результатов обеспечивается применением строгого математического аппарата теории автоматического управления, 
методов оптимизации и численного моделирования; использованием апробированных методов термодинамики и механики для
построения математических моделей электропневматического привода; корректностью принятых допущений
при разработке математических моделей; применением современных методов машинного обучения и
статистического анализа при построении суррогатных моделей и обработке экспериментальных данных.

Теоретические положения и результаты моделирования подтверждаются данными экспериментальных
исследований, проведенных на специально разработанном лабораторном стенде с использованием
современных средств измерения и регистрации данных. Полученные результаты согласуются с известными
теоретическими положениями и экспериментальными данными других исследователей в области управления пневматическими системами.

Обоснованность научных положений и выводов подтверждается публикациями в рецензируемых
научных изданиях, положительной оценкой результатов работы на международных и всероссийских научных конференциях.

{\probation}
Основные результаты работы докладывались~на:
перечисление основных конференций, симпозиумов и~т.\:п.

{\contribution} 
Личный вклад автора заключается в разработке методики многокритериального
сравнительного анализа алгоритмов дискретного управления электропневматическим
приводом на основе построения и исследования фронтов Парето. Автором предложены
оригинальные модификации существующих алгоритмов управления, включая многорежимное
скользящее управление и нечеткое управление с расширенной базой правил, обеспечивающие улучшенные показатели качества позиционирования.

В рамках исследования автором разработана модифицированная архитектура
нейронной сети с остаточными блоками для построения суррогатных моделей электропневматического
привода. Предложен новый метод построения и анализа фронтов Парето на основе комбинации эволюционных
алгоритмов оптимизации и суррогатного моделирования. Сформулирован
комплекс критериев оценки эффективности алгоритмов управления,
учитывающий как точность позиционирования, так и интенсивность переключений распределителей.

Автором самостоятельно спроектирован и изготовлен экспериментальный
стенд, разработано программное обеспечение для проведения исследований,
выполнены все экспериментальные исследования и обработка их результатов
Проведена верификация теоретических результатов путем сравнения с экспериментальными
данными. На основе полученных результатов сформулированы практические рекомендации
по выбору алгоритмов управления электропневматическим приводом в зависимости от требований к системе.

Все основные результаты диссертационной работы получены автором самостоятельно.
Личный вклад в работах, опубликованных в соавторстве, заключается в
постановке задач исследования, разработке методов их решения, проведении
теоретических и экспериментальных исследований, анализе и обобщении полученных результатов.

\ifnumequal{\value{bibliosel}}{0}
{%%% Встроенная реализация с загрузкой файла через движок bibtex8. (При желании, внутри можно использовать обычные ссылки, наподобие `\cite{vakbib1,vakbib2}`).
    {\publications} Основные результаты по теме диссертации изложены
    в~XX~печатных изданиях,
    X из которых изданы в журналах, рекомендованных ВАК,
    X "--- в тезисах докладов.
}%
{%%% Реализация пакетом biblatex через движок biber
    \begin{refsection}[bl-author, bl-registered]
        % Это refsection=1.
        % Процитированные здесь работы:
        %  * подсчитываются, для автоматического составления фразы "Основные результаты ..."
        %  * попадают в авторскую библиографию, при usefootcite==0 и стиле `\insertbiblioauthor` или `\insertbiblioauthorgrouped`
        %  * нумеруются там в зависимости от порядка команд `\printbibliography` в этом разделе.
        %  * при использовании `\insertbiblioauthorgrouped`, порядок команд `\printbibliography` в нём должен быть тем же (см. biblio/biblatex.tex)
        %
        % Невидимый библиографический список для подсчёта количества публикаций:
        \phantom{\printbibliography[heading=nobibheading, section=1, env=countauthorvak,          keyword=biblioauthorvak]%
        \printbibliography[heading=nobibheading, section=1, env=countauthorwos,          keyword=biblioauthorwos]%
        \printbibliography[heading=nobibheading, section=1, env=countauthorscopus,       keyword=biblioauthorscopus]%
        \printbibliography[heading=nobibheading, section=1, env=countauthorconf,         keyword=biblioauthorconf]%
        \printbibliography[heading=nobibheading, section=1, env=countauthorother,        keyword=biblioauthorother]%
        \printbibliography[heading=nobibheading, section=1, env=countregistered,         keyword=biblioregistered]%
        \printbibliography[heading=nobibheading, section=1, env=countauthorpatent,       keyword=biblioauthorpatent]%
        \printbibliography[heading=nobibheading, section=1, env=countauthorprogram,      keyword=biblioauthorprogram]%
        \printbibliography[heading=nobibheading, section=1, env=countauthor,             keyword=biblioauthor]%
        \printbibliography[heading=nobibheading, section=1, env=countauthorvakscopuswos, filter=vakscopuswos]%
        \printbibliography[heading=nobibheading, section=1, env=countauthorscopuswos,    filter=scopuswos]}%
        %
        \nocite{*}%
        %
        {\publications} Основные результаты по теме диссертации изложены в~\arabic{citeauthor}~печатных изданиях,
        \arabic{citeauthorvak} из которых изданы в журналах, рекомендованных ВАК%
        \ifnum \value{citeauthorscopuswos}>0%
            , \arabic{citeauthorscopuswos} "--- в~периодических научных журналах, индексируемых Web of~Science и Scopus%
        \fi%
        \ifnum \value{citeauthorconf}>0%
            , \arabic{citeauthorconf} "--- в~тезисах докладов.
        \else%
            .
        \fi%
        \ifnum \value{citeregistered}=1%
            \ifnum \value{citeauthorpatent}=1%
                Зарегистрирован \arabic{citeauthorpatent} патент.
            \fi%
            \ifnum \value{citeauthorprogram}=1%
                Зарегистрирована \arabic{citeauthorprogram} программа для ЭВМ.
            \fi%
        \fi%
        \ifnum \value{citeregistered}>1%
            Зарегистрированы\ %
            \ifnum \value{citeauthorpatent}>0%
            \formbytotal{citeauthorpatent}{патент}{}{а}{}%
            \ifnum \value{citeauthorprogram}=0 . \else \ и~\fi%
            \fi%
            \ifnum \value{citeauthorprogram}>0%
            \formbytotal{citeauthorprogram}{программ}{а}{ы}{} для ЭВМ.
            \fi%
        \fi%
        % К публикациям, в которых излагаются основные научные результаты диссертации на соискание учёной
        % степени, в рецензируемых изданиях приравниваются патенты на изобретения, патенты (свидетельства) на
        % полезную модель, патенты на промышленный образец, патенты на селекционные достижения, свидетельства
        % на программу для электронных вычислительных машин, базу данных, топологию интегральных микросхем,
        % зарегистрированные в установленном порядке.(в ред. Постановления Правительства РФ от 21.04.2016 N 335)
    \end{refsection}%
    \begin{refsection}[bl-author, bl-registered]
        % Это refsection=2.
        % Процитированные здесь работы:
        %  * попадают в авторскую библиографию, при usefootcite==0 и стиле `\insertbiblioauthorimportant`.
        %  * ни на что не влияют в противном случае
        \nocite{vakbib2}%vak
        \nocite{patbib1}%patent
        \nocite{progbib1}%program
        \nocite{bib1}%other
        \nocite{confbib1}%conf
    \end{refsection}%
        %
        % Всё, что вне этих двух refsection, это refsection=0,
        %  * для диссертации - это нормальные ссылки, попадающие в обычную библиографию
        %  * для автореферата:
        %     * при usefootcite==0, ссылка корректно сработает только для источника из `external.bib`. Для своих работ --- напечатает "[0]" (и даже Warning не вылезет).
        %     * при usefootcite==1, ссылка сработает нормально. В авторской библиографии будут только процитированные в refsection=0 работы.
}
