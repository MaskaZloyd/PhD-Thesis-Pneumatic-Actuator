\pdfbookmark{Общая характеристика работы}{characteristic}             % Закладка pdf
\section*{Общая характеристика работы}

\newcommand{\actuality}{\pdfbookmark[1]{Актуальность}{actuality}\underline{\textbf{\actualityTXT}}}
\newcommand{\progress}{\pdfbookmark[1]{Разработанность темы}{progress}\underline{\textbf{\progressTXT}}}
\newcommand{\aim}{\pdfbookmark[1]{Цели}{aim}\underline{{\textbf\aimTXT}}}
\newcommand{\tasks}{\pdfbookmark[1]{Задачи}{tasks}\underline{\textbf{\tasksTXT}}}
\newcommand{\aimtasks}{\pdfbookmark[1]{Цели и задачи}{aimtasks}\aimtasksTXT}
\newcommand{\novelty}{\pdfbookmark[1]{Научная новизна}{novelty}\underline{\textbf{\noveltyTXT}}}
\newcommand{\influence}{\pdfbookmark[1]{Практическая значимость}{influence}\underline{\textbf{\influenceTXT}}}
\newcommand{\methods}{\pdfbookmark[1]{Методология и методы исследования}{methods}\underline{\textbf{\methodsTXT}}}
\newcommand{\defpositions}{\pdfbookmark[1]{Положения, выносимые на защиту}{defpositions}\underline{\textbf{\defpositionsTXT}}}
\newcommand{\reliability}{\pdfbookmark[1]{Достоверность}{reliability}\underline{\textbf{\reliabilityTXT}}}
\newcommand{\probation}{\pdfbookmark[1]{Апробация}{probation}\underline{\textbf{\probationTXT}}}
\newcommand{\contribution}{\pdfbookmark[1]{Личный вклад}{contribution}\underline{\textbf{\contributionTXT}}}
\newcommand{\publications}{\pdfbookmark[1]{Публикации}{publications}\underline{\textbf{\publicationsTXT}}}


{\actuality}

\ifsynopsis
%Этот абзац появляется только в~автореферате.

\else
%Этот абзац появляется только в~диссертации.

\fi

% {\progress}
% Этот раздел должен быть отдельным структурным элементом по
% ГОСТ, но он, как правило, включается в описание актуальности
% темы. Нужен он отдельным структурынм элемементом или нет ---
% смотрите другие диссертации вашего совета, скорее всего не нужен.

{\aim} 
диссертационной работы
разработка методики многокритериального сравнительного анализа алгоритмов дискретного
управления электропневматическим приводом на основе построения и
исследования фронтов Парето, обеспечивающей научно обоснованный выбор
оптимального алгоритма управления в соответствии с заданными эксплуатационными требованиями.

Предлагаемая методика основывается на многокритериальной оптимизации с использованием фронтов Парето
и суррогатных моделей, что позволяет обеспечить баланс между различными требованиями к качеству работы пневмопривода.

Для~достижения поставленной цели необходимо было решить следующие {\tasks}:
\begin{enumerate}[beginpenalty=10000] % https://tex.stackexchange.com/a/476052/104425
\item Разработать математические модели электропневматического привода с дискретными распределителями,
которые учитывают нелинейную динамику системы и дискретный характер управляющих воздействий для различных алгоритмов управления.

\item Проведение анализа существующих алгоритмов дискретного управления электропневматическим приводом и
методов их оптимизации для выявления основных характеристик и закономерностей функционирования системы

\item Разработка математических моделей и критериев оценки эффективности электропневматического привода с дискретными
распределителями, учитывающих как качество позиционирования, так и интенсивность переключений распределителей 
и остальных интересующих показателей качества.

\item Синтез и исследование различных алгоритмов дискретного управления электропневматическим приводом,
включая управление в скользящих режимах, ПИД-регулирование с ШИМ, нечеткое управление и прогнозное управление.

\item Разработка методов построения нейросетевых суррогатных моделей для аппроксимации зависимостей между
параметрами алгоритмов управления и показателями качества функционирования электропневматического привода.

\item Создание алгоритмов построения и анализа фронтов Парето для различных методов
управления с использованием суррогатных моделей и эволюционных алгоритмов оптимизации.

\item Разработка программно-аппаратного комплекса для
экспериментальных исследований электропневматического привода и верификации теоретических результатов.

\item Проведение сравнительного анализа эффективности различных алгоритмов управления на основе построенных фронтов Парето и
формирование рекомендаций по их практическому применению в зависимости от требований к системе.

Решение данных задач позволило создать комплексную методику многокритериального анализа и
обоснованного выбора алгоритмов управления электропневматическим приводом.
\end{enumerate}


{\novelty}
\begin{enumerate}[beginpenalty=10000] % https://tex.stackexchange.com/a/476052/104425
    \item Разработан новый подход к сравнительному анализу алгоритмов дискретного управления
    электропневматическим приводом, основанный на построении и исследовании фронтов Парето с
    использованием нейросетевых суррогатных моделей, что позволяет осуществлять научно
    обоснованный выбор алгоритма управления в соответствии с заданными требованиями к системе.

    \item Предложена модифицированная архитектура нейросетевой суррогатной модели на основе
    остаточных блоков для аппроксимации зависимостей между
    параметрами алгоритмов управления и показателями качества функционирования
    электропневматического привода, обеспечивающая повышенную точность
    прогнозирования и устойчивость к шуму в данных.

    \item Разработаны новые критерии оценки эффективности алгоритмов управления
    электропневматическим приводом, учитывающие как точность позиционирования, так
    и интенсивность переключений распределителей, что позволяет
    проводить комплексную оценку качества функционирования системы.

    \item Создан новый метод построения и анализа фронтов Парето для алгоритмов
    управления электропневматическим приводом, основанный на комбинации эволюционных
    алгоритмов оптимизации и суррогатного моделирования, что позволяет существенно сократить
    вычислительные затраты при сохранении точности результатов.

    \item Проведены экспериментальные исследования для верификации численных моделей и методик,
    что подтверждает их адекватность и применимость для реальных промышленных задач.
\end{enumerate}

{\influence} \ldots

{\methods} В рамках диссертационного исследования применялись теоретические и экспериментальные методы.
Теоретическая часть работы базируется на фундаментальных положениях теории автоматического 
управления, математического моделирования и оптимизации. При разработке математических моделей
электропневматического привода использовались методы термодинамики и механики сплошных
сред. Анализ динамики системы осуществлялся с применением методов качественной теории 
дифференциальных уравнений и теории устойчивости. Синтез алгоритмов управления проводился
с использованием теории скользящих режимов, нечеткой логики и прогнозного управления.

Для построения суррогатных моделей применялись методы глубокого обучения и нейросетевого
моделирования. Многокритериальная оптимизация параметров алгоритмов управления осуществлялась
с использованием эволюционных алгоритмов и теории Парето-оптимальности. Статистическая обработка
результатов выполнялась методами математической статистики и регрессионного анализа.

Экспериментальные исследования проводились на специально разработанном
лабораторном стенде с использованием современных средств измерения и
обработки данных. Верификация теоретических результатов осуществлялась
путем сравнительного анализа расчетных и экспериментальных данных.

Программная реализация алгоритмов управления и методов оптимизации
выполнена с использованием современных языков программирования Python и
Julia, а также специализированных библиотек для научных вычислений и машинного
обучения. Визуализация результатов осуществлялась с применением
современных графических средств и методов многомерного анализа данных.

Достоверность полученных результатов обеспечивается корректным применением
апробированных научных методов, согласованностью теоретических и экспериментальных
данных.

{\defpositions}
\begin{enumerate}[beginpenalty=10000] % https://tex.stackexchange.com/a/476052/104425

\item Методика многокритериального сравнительного анализа алгоритмов дискретного управления
электропневматическим приводом, основанная на построении и исследовании фронтов Парето
с использованием нейросетевых суррогатных моделей, обеспечивающая научно обоснованный
выбор оптимального алгоритма управления для заданных эксплуатационных требований.

\item Модифицированная архитектура нейронной сети с остаточными блоками для построения суррогатных моделей
электропневматического привода, позволяющая с высокой точностью
аппроксимировать зависимости между параметрами алгоритмов управления
и показателями качества функционирования системы при существенном сокращении вычислительных затрат.

\item Комплекс критериев оценки эффективности алгоритмов управления электропневматическим
приводом, учитывающий как точность позиционирования, так и интенсивность
переключений распределителей, позволяющий проводить всесторонний анализ качества
функционирования системы в различных режимах работы.

\item Метод построения и анализа фронтов Парето для алгоритмов управления электропневматическим
приводом на основе комбинации эволюционных алгоритмов оптимизации и суррогатного моделирования,
обеспечивающий существенное сокращение вычислительных затрат при сохранении точности результатов.

\item Результаты экспериментальных исследований, подтверждающие эффективность разработанной
методики многокритериального анализа и практические рекомендации по выбору алгоритмов
управления электропневматическим приводом в зависимости от требований к системе.

\end{enumerate}

{\reliability} 
полученных результатов обеспечивается применением строгого математического аппарата теории автоматического управления, 
методов оптимизации и численного моделирования; использованием апробированных методов термодинамики и механики для
построения математических моделей электропневматического привода; корректностью принятых допущений
при разработке математических моделей; применением современных методов машинного обучения и
статистического анализа при построении суррогатных моделей и обработке экспериментальных данных.

Теоретические положения и результаты моделирования подтверждаются данными экспериментальных
исследований, проведенных на специально разработанном лабораторном стенде с использованием
современных средств измерения и регистрации данных. Полученные результаты согласуются с известными
теоретическими положениями и экспериментальными данными других исследователей в области управления пневматическими системами.

Обоснованность научных положений и выводов подтверждается публикациями в рецензируемых
научных изданиях, положительной оценкой результатов работы на международных и всероссийских научных конференциях.

{\probation}
Основные результаты работы докладывались~на:
перечисление основных конференций, симпозиумов и~т.\:п.

{\contribution} 
Личный вклад автора заключается в разработке методики многокритериального
сравнительного анализа алгоритмов дискретного управления электропневматическим
приводом на основе построения и исследования фронтов Парето. Автором предложены
оригинальные модификации существующих алгоритмов управления, включая многорежимное
скользящее управление и нечеткое управление с расширенной базой правил, обеспечивающие улучшенные показатели качества позиционирования.

В рамках исследования автором разработана модифицированная архитектура
нейронной сети с остаточными блоками для построения суррогатных моделей электропневматического
привода. Предложен новый метод построения и анализа фронтов Парето на основе комбинации эволюционных
алгоритмов оптимизации и суррогатного моделирования. Сформулирован
комплекс критериев оценки эффективности алгоритмов управления,
учитывающий как точность позиционирования, так и интенсивность переключений распределителей.

Автором самостоятельно спроектирован и изготовлен экспериментальный
стенд, разработано программное обеспечение для проведения исследований,
выполнены все экспериментальные исследования и обработка их результатов
Проведена верификация теоретических результатов путем сравнения с экспериментальными
данными. На основе полученных результатов сформулированы практические рекомендации
по выбору алгоритмов управления электропневматическим приводом в зависимости от требований к системе.

Все основные результаты диссертационной работы получены автором самостоятельно.
Личный вклад в работах, опубликованных в соавторстве, заключается в
постановке задач исследования, разработке методов их решения, проведении
теоретических и экспериментальных исследований, анализе и обобщении полученных результатов.

\ifnumequal{\value{bibliosel}}{0}
{%%% Встроенная реализация с загрузкой файла через движок bibtex8. (При желании, внутри можно использовать обычные ссылки, наподобие `\cite{vakbib1,vakbib2}`).
    {\publications} Основные результаты по теме диссертации изложены
    в~XX~печатных изданиях,
    X из которых изданы в журналах, рекомендованных ВАК,
    X "--- в тезисах докладов.
}%
{%%% Реализация пакетом biblatex через движок biber
    \begin{refsection}[bl-author, bl-registered]
        % Это refsection=1.
        % Процитированные здесь работы:
        %  * подсчитываются, для автоматического составления фразы "Основные результаты ..."
        %  * попадают в авторскую библиографию, при usefootcite==0 и стиле `\insertbiblioauthor` или `\insertbiblioauthorgrouped`
        %  * нумеруются там в зависимости от порядка команд `\printbibliography` в этом разделе.
        %  * при использовании `\insertbiblioauthorgrouped`, порядок команд `\printbibliography` в нём должен быть тем же (см. biblio/biblatex.tex)
        %
        % Невидимый библиографический список для подсчёта количества публикаций:
        \phantom{\printbibliography[heading=nobibheading, section=1, env=countauthorvak,          keyword=biblioauthorvak]%
        \printbibliography[heading=nobibheading, section=1, env=countauthorwos,          keyword=biblioauthorwos]%
        \printbibliography[heading=nobibheading, section=1, env=countauthorscopus,       keyword=biblioauthorscopus]%
        \printbibliography[heading=nobibheading, section=1, env=countauthorconf,         keyword=biblioauthorconf]%
        \printbibliography[heading=nobibheading, section=1, env=countauthorother,        keyword=biblioauthorother]%
        \printbibliography[heading=nobibheading, section=1, env=countregistered,         keyword=biblioregistered]%
        \printbibliography[heading=nobibheading, section=1, env=countauthorpatent,       keyword=biblioauthorpatent]%
        \printbibliography[heading=nobibheading, section=1, env=countauthorprogram,      keyword=biblioauthorprogram]%
        \printbibliography[heading=nobibheading, section=1, env=countauthor,             keyword=biblioauthor]%
        \printbibliography[heading=nobibheading, section=1, env=countauthorvakscopuswos, filter=vakscopuswos]%
        \printbibliography[heading=nobibheading, section=1, env=countauthorscopuswos,    filter=scopuswos]}%
        %
        \nocite{*}%
        %
        {\publications} Основные результаты по теме диссертации изложены в~\arabic{citeauthor}~печатных изданиях,
        \arabic{citeauthorvak} из которых изданы в журналах, рекомендованных ВАК%
        \ifnum \value{citeauthorscopuswos}>0%
            , \arabic{citeauthorscopuswos} "--- в~периодических научных журналах, индексируемых Web of~Science и Scopus%
        \fi%
        \ifnum \value{citeauthorconf}>0%
            , \arabic{citeauthorconf} "--- в~тезисах докладов.
        \else%
            .
        \fi%
        \ifnum \value{citeregistered}=1%
            \ifnum \value{citeauthorpatent}=1%
                Зарегистрирован \arabic{citeauthorpatent} патент.
            \fi%
            \ifnum \value{citeauthorprogram}=1%
                Зарегистрирована \arabic{citeauthorprogram} программа для ЭВМ.
            \fi%
        \fi%
        \ifnum \value{citeregistered}>1%
            Зарегистрированы\ %
            \ifnum \value{citeauthorpatent}>0%
            \formbytotal{citeauthorpatent}{патент}{}{а}{}%
            \ifnum \value{citeauthorprogram}=0 . \else \ и~\fi%
            \fi%
            \ifnum \value{citeauthorprogram}>0%
            \formbytotal{citeauthorprogram}{программ}{а}{ы}{} для ЭВМ.
            \fi%
        \fi%
        % К публикациям, в которых излагаются основные научные результаты диссертации на соискание учёной
        % степени, в рецензируемых изданиях приравниваются патенты на изобретения, патенты (свидетельства) на
        % полезную модель, патенты на промышленный образец, патенты на селекционные достижения, свидетельства
        % на программу для электронных вычислительных машин, базу данных, топологию интегральных микросхем,
        % зарегистрированные в установленном порядке.(в ред. Постановления Правительства РФ от 21.04.2016 N 335)
    \end{refsection}%
    \begin{refsection}[bl-author, bl-registered]
        % Это refsection=2.
        % Процитированные здесь работы:
        %  * попадают в авторскую библиографию, при usefootcite==0 и стиле `\insertbiblioauthorimportant`.
        %  * ни на что не влияют в противном случае
        \nocite{vakbib2}%vak
        \nocite{patbib1}%patent
        \nocite{progbib1}%program
        \nocite{bib1}%other
        \nocite{confbib1}%conf
    \end{refsection}%
        %
        % Всё, что вне этих двух refsection, это refsection=0,
        %  * для диссертации - это нормальные ссылки, попадающие в обычную библиографию
        %  * для автореферата:
        %     * при usefootcite==0, ссылка корректно сработает только для источника из `external.bib`. Для своих работ --- напечатает "[0]" (и даже Warning не вылезет).
        %     * при usefootcite==1, ссылка сработает нормально. В авторской библиографии будут только процитированные в refsection=0 работы.
}
 % Характеристика работы по структуре во введении и в автореферате не отличается (ГОСТ Р 7.0.11, пункты 5.3.1 и 9.2.1), потому её загружаем из одного и того же внешнего файла, предварительно задав форму выделения некоторым параметрам

%Диссертационная работа была выполнена при поддержке грантов \dots

%\underline{\textbf{Объем и структура работы.}} Диссертация состоит из~введения,
%четырех глав, заключения и~приложения. Полный объем диссертации
%\textbf{ХХХ}~страниц текста с~\textbf{ХХ}~рисунками и~5~таблицами. Список
%литературы содержит \textbf{ХХX}~наименование.

\pdfbookmark{Содержание работы}{description}                          % Закладка pdf
\section*{Содержание работы}
Во \underline{\textbf{введении}} обосновывается актуальность
исследований, проводимых в~рамках данной диссертационной работы,
приводится обзор научной литературы по~изучаемой проблеме,
формулируется цель, ставятся задачи работы, излагается научная новизна
и практическая значимость представляемой работы. В~последующих главах
сначала описывается общий принцип, позволяющий \dots, а~потом идёт
апробация на частных примерах: \dots  и~\dots.


\underline{\textbf{Первая глава}}
посвящена анализу проблемы управления электропневматическим приводом с
дискретными распределителями. В ней представлено аналитическое исследование
особенностей пневмоприводных устройств, включая их преимущества и недостатки
по сравнению с другими типами приводов. Особое внимание уделено специфике работы позиционного пневмопривода, использующего дискретные распределители вместо более дорогостоящих пропорциональных аналогов.

Выполнен комплексный анализ задач разгона и торможения пневмопривода. Рассмотрены
существующие методы повышения быстродействия, включая использование встроенных
полостей-резервуаров и оптимизацию конструкции распределителей. Исследованы различные
подходы к торможению, в том числе механические, пневматические и комбинированные тормозные
устройства. Представлен анализ эффективности двенадцати схем торможения, систематизированных по принципу действия.

В контексте систем управления проанализированы современные методы,
включая широтно-импульсную модуляцию, скользящие режимы и нечеткую логику.
Показано, что применение классических алгоритмов управления к системам с
дискретными распределителями сопряжено с определенными трудностями,
требующими разработки специализированных подходов.

На основе проведенного анализа сформулирована научная проблема
разработки эффективных методов управления позиционным пневмоприводом с
дискретными распределителями, обеспечивающих компромисс между точностью
позиционирования и интенсивностью переключений распределительной аппаратуры.

\underline{\textbf{Вторая глава}} 
посвящена разработке математического описания электропневматического
привода с дискретными распределителями как объекта управления. Представлена
нелинейная математическая модель, учитывающая термодинамические процессы в
полостях пневмоцилиндра, динамику механической части, характеристики распределителей и влияние сил трения.

Термодинамические процессы в полостях пневмоцилиндра описаны системой
дифференциальных уравнений, включающей уравнения сохранения массы и
энергии для сжимаемого газа. Массовые расходы через распределители определены с использованием
уравнений Сен-Венана-Ванцеля для докритических и сверхкритических режимов течения.

Предложена модифицированная модель трения LuGre, адаптированная для пневматических систем.
Модель учитывает эффект Штрибека, а также упругие и демпфирующие свойства микроконтактов
между поверхностями трения. Разработан алгоритм численной реализации модели,
обеспечивающий устойчивость вычислений при малых скоростях движения.

Динамика электромагнитных распределителей представлена апериодическим
звеном первого порядка с ограничением на минимальное время между переключениями.
Предложен метод идентификации параметров модели на основе экспериментальных
данных с использованием методов машинного обучения.

Разработанная математическая модель верифицирована путем сравнения
с экспериментальными данными. Показано, что модель обеспечивает
точность прогнозирования положения штока не хуже 0,1 мм и адекватно
отражает динамику переходных процессов в системе. Проведен
анализ чувствительности модели к вариациям параметров, что позволило выделить
наиболее значимые факторы, влияющие на поведение системы.

На основе полной модели синтезирована упрощенная версия для использования в
алгоритмах управления реального времени. Применен метод сингулярных
возмущений для разделения быстрых и медленных процессов, что позволило
существенно снизить вычислительную сложность модели при сохранении приемлемой точности.

\underline{\textbf{Третья глава}} посвящена исследованию

Можно сослаться на свои работы в автореферате. Для этого в файле
\verb!Synopsis/setup.tex! необходимо присвоить положительное значение
счётчику \verb!\setcounter{usefootcite}{1}!. В таком случае ссылки на
работы других авторов будут подстрочными.
Изложенные в третьей главе результаты опубликованы в~\cite{vakbib1, vakbib2}.
Использование подстрочных ссылок внутри таблиц может вызывать проблемы.

В \underline{\textbf{четвертой главе}} приведено описание

\FloatBarrier
\pdfbookmark{Заключение}{conclusion}                                  % Закладка pdf
В \underline{\textbf{заключении}} приведены основные результаты работы, которые заключаются в следующем:
%% Согласно ГОСТ Р 7.0.11-2011:
%% 5.3.3 В заключении диссертации излагают итоги выполненного исследования, рекомендации, перспективы дальнейшей разработки темы.
%% 9.2.3 В заключении автореферата диссертации излагают итоги данного исследования, рекомендации и перспективы дальнейшей разработки темы.
В результате проведенного диссертационного исследования достигнута поставленная
цель комплексного повышения статико-динамических и ресурсных показателей позиционного пневмопривода с
дискретными распределителями в условиях их конфликтности. Получены следующие основные результаты:

\begin{enumerate}
    \item Разработана комплексная математическая модель позиционного пневмопривода с дискретными распределителями,
    которая включает в себя как силовую, так и управляющую структуры. Модель учитывает нелинейные термодинамические
    процессы в полостях пневмоцилиндра, переменную структуру системы при переключении распределителей, а также
    динамические эффекты сухого и вязкого трения. Высокая степень согласованности результатов моделирования с
    экспериментальными данными (расхождение не превышает 5~--~7\%) подтверждает адекватность и достоверность
    разработанной модели для решения задач анализа и синтеза пневмоприводов различной структуры.

    \item Проведен комплексный анализ различных структур пневмопривода, что позволило выявить оптимальные
    режимы переключения распределителей при позиционировании рабочего органа. Разработаны структуры управления
    на основе различных принципов: модифицированная структура ПИД-регулятора с блоком прогнозирования тормозного пути,
    управление в скользящих режимах с интегральной и терминальной поверхностями, нечеткое управление и прогнозное управление.
    Экспериментальный анализ подтвердил, что прогнозное управление обеспечивает повышение точности позиционирования на 25~--~30\%
    при одновременном сокращении числа переключений распределителей на 35~--~45\% по сравнению с традиционными алгоритмами.

    \item Проведен натурный эксперимент на специально разработанном лабораторном стенде, который подтвердил работоспособность
    предложенных структур и высокую достоверность результатов, полученных с использованием разработанной математической модели.
    Сравнительный анализ экспериментальных данных показал, что наибольшую точность моделирования демонстрируют алгоритмы прогнозного
    управления и семирежимного управления в скользящих режимах, где средняя относительная ошибка не превышает 4,65\%.
    Расхождение между расчетными и экспериментальными данными по основным показателям качества составляет
    не более 7\% для всех исследованных алгоритмов управления.

    \item Разработана методика многокритериального структурно-параметрического синтеза, основанная
    на использовании фронтов Парето с применением замещающей суррогатной нейросетевой модели. Для
    формирования обучающей выборки применен метод латинского гиперкуба, обеспечивающий оптимальное
    заполнение пространства параметров, что позволило сократить вычислительные затраты при построении
    фронтов Парето на 48\%. Использование суррогатной модели обеспечило среднюю точность аппроксимации 91\%
    при максимальном отклонении 12\%.

    \item Определена взаимосвязь между конфликтными статико-динамическими и ресурсными показателями
    для позиционных пневмоприводов различной структуры на основе анализа фронтов Парето. Установлено,
    что для ПИД-регулятора с ШИМ диапазон изменения критерия $ITAE$ составляет от 0,0298 до 0,0427 \si{\meter\per\second\square}
    при изменении точности от 0,41 до 2,57 мм. Семирежимное управление с интегральной поверхностью
    обеспечивает точность 0,13 мм при $ITAE = 0,019$ \si{\meter\per\second\square} и $SI = 15$, а прогнозное управление демонстрирует
    точность 0,12 мм при $ITAE = 0,032$ \si{\meter\per\second\square} и $SI = 14$~--~$16$. Для нечеткого регулятора характерны
    наилучшие показатели по критерию динамической эффективности ($ITAE = 0,006$ \si{\meter\per\second\square}) при точности 0,40 мм.

    \item Сформированы практические рекомендации по выбору оптимальной структуры позиционного
    пневмопривода с дискретными распределителями. Выделены четыре предпочтительные области
    применения различных алгоритмов управления:
    \begin{itemize}
        \item область высоконагруженных систем (УСР-И-5, MPC) с точностью 0,12~--~0,36 мм;
        \item область точных манипуляторов (УСР-И-7) с точностью 0,14 мм;
        \item область промышленной автоматики (УСР-Т-5, УСР-Т-7) с точностью 0,13~--~0,28 мм;
        \item область простых систем (ПИД+ШИМ, УСР-И-3) с точностью 0,5~--~2,5 мм.
    \end{itemize}
    Определены рекомендуемые значения параметров алгоритмов управления для типовых задач
    позиционирования, что позволяет сократить сроки проектирования и повышает эффективность разрабатываемых систем.
\end{enumerate}

Полученные результаты позволяют на научной основе выбирать оптимальную структуру и
параметры позиционного пневмопривода с дискретными распределителями в зависимости
от конкретных требований, существенно сокращая сроки проектирования и повышая
эффективность разрабатываемых систем.

\pdfbookmark{Литература}{bibliography}                                % Закладка pdf
При использовании пакета \verb!biblatex! список публикаций автора по теме
диссертации формируется в разделе <<\publications>>\ файла
\verb!common/characteristic.tex!  при помощи команды \verb!\nocite!

\ifdefmacro{\microtypesetup}{\microtypesetup{protrusion=false}}{} % не рекомендуется применять пакет микротипографики к автоматически генерируемому списку литературы
\urlstyle{rm}                               % ссылки URL обычным шрифтом
\ifnumequal{\value{bibliosel}}{0}{% Встроенная реализация с загрузкой файла через движок bibtex8
    \renewcommand{\bibname}{\large \bibtitleauthor}
    \nocite{*}
    \insertbiblioauthor           % Подключаем Bib-базы
    %\insertbiblioexternal   % !!! bibtex не умеет работать с несколькими библиографиями !!!
}{% Реализация пакетом biblatex через движок biber
    % Цитирования.
    %  * Порядок перечисления определяет порядок в библиографии (только внутри подраздела, если `\insertbiblioauthorgrouped`).
    %  * Если не соблюдать порядок "как для \printbibliography", нумерация в `\insertbiblioauthor` будет кривой.
    %  * Если цитировать каждый источник отдельной командой --- найти некоторые ошибки будет проще.
    %
    %% authorvak
    \nocite{vakbib1}%
    \nocite{vakbib2}%
    \nocite{vakbib3}%
    \nocite{vakbib4}%
    \nocite{vakbib5}%
    \nocite{vakbib6}%
    \nocite{vakbib7}%
    \nocite{vakbib8}%
    \nocite{vakbib9}%
    \nocite{vakbib10}%
    \nocite{vakbib11}%
    \nocite{vakbib12}%
    %
    %% authorwos
    \nocite{wosbib1}%
    %
    %% authorscopus
    \nocite{scbib1}%
    %
    %% authorpatent
    \nocite{patbib1}%
    %
    %% authorprogram
    \nocite{progbib1}%
    %
    %% authorconf
    \nocite{confbib1}%
    \nocite{confbib2}%
    %
    %% authorother
    \nocite{bib1}%
    \nocite{bib2}%

    \ifnumgreater{\value{usefootcite}}{0}{
        \begin{refcontext}[labelprefix={}]
            \ifnum \value{bibgrouped}>0
                \insertbiblioauthorgrouped    % Вывод всех работ автора, сгруппированных по источникам
            \else
                \insertbiblioauthor      % Вывод всех работ автора
            \fi
        \end{refcontext}
    }{
        \ifnum \totvalue{citeexternal}>0
            \begin{refcontext}[labelprefix=A]
                \ifnum \value{bibgrouped}>0
                    \insertbiblioauthorgrouped    % Вывод всех работ автора, сгруппированных по источникам
                \else
                    \insertbiblioauthor      % Вывод всех работ автора
                \fi
            \end{refcontext}
        \else
            \ifnum \value{bibgrouped}>0
                \insertbiblioauthorgrouped    % Вывод всех работ автора, сгруппированных по источникам
            \else
                \insertbiblioauthor      % Вывод всех работ автора
            \fi
        \fi
        %  \insertbiblioauthorimportant  % Вывод наиболее значимых работ автора (определяется в файле characteristic во второй section)
        \begin{refcontext}[labelprefix={}]
            \insertbiblioexternal            % Вывод списка литературы, на которую ссылались в тексте автореферата
        \end{refcontext}
        % Невидимый библиографический список для подсчёта количества внешних публикаций
        % Используется, чтобы убрать приставку "А" у работ автора, если в автореферате нет
        % цитирований внешних источников.
        \printbibliography[heading=nobibheading, section=0, env=countexternal, keyword=biblioexternal, resetnumbers=true]%
    }
}
\ifdefmacro{\microtypesetup}{\microtypesetup{protrusion=true}}{}
\urlstyle{tt}                               % возвращаем установки шрифта ссылок URL
