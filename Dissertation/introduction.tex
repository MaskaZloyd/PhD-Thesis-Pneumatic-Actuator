\chapter*{Введение}                         % Заголовок
\addcontentsline{toc}{chapter}{Введение}    % Добавляем его в оглавление

\newcommand{\actuality}{}
\newcommand{\progress}{}
\newcommand{\aim}{{\textbf\aimTXT}}
\newcommand{\tasks}{\textbf{\tasksTXT}}
\newcommand{\novelty}{\textbf{\noveltyTXT}}
\newcommand{\influenceTheoretical}{\textbf{\influenceTheoreticalTXT}}
\newcommand{\influencePractical}{\textbf{\influencePracticalTXT}}
\newcommand{\methods}{\textbf{\methodsTXT}}
\newcommand{\defpositions}{\textbf{\defpositionsTXT}}
\newcommand{\reliability}{\textbf{\reliabilityTXT}}
\newcommand{\probation}{\textbf{\probationTXT}}
\newcommand{\contribution}{\textbf{\contributionTXT}}
\newcommand{\publications}{\textbf{\publicationsTXT}}


{\actuality}

Позиционные пневмоприводы нашли широкое применение при автоматизации различных производственных процессов.
Они позволяют обеспечить возможность остановки и фиксации выходного звена в любой промежуточной точке между
его крайними положениями. При этом контроль траектории движения, как правило, не требуется. Наиболее часто
позиционные пневмоприводы используются в станочном и кузнечнопрессовом оборудовании, в сварочных и
литейных машинах, в роботехнических комплексах, а также в автоматизированных производственных линиях
и системах упаковки. Известно их применение в качестве приводов трубопроводной арматуры \cite{Dolgov2017}. С их
помощью реализуются такие массовые операции как сдвиг, зажим или подъем деталей и инструментов.

В настоящее время наблюдается устойчивая тенденция замещения дорогостоящих пропорциональных распределителей,
используемых в позиционных пневмоприводах, более дешевыми дискретными пневмоаппаратами клапанного
типа \cite{Markov2007,Pilgunov2016, An2015}. В этом случае управление реализуется за счет изменения периодов нахождения
распределителей в открытом и закрытом состоянии.

Однако в таких пневмоприводах с дискретными распределителями увеличение точности
позиционирования или быстродействия ухудшает качество рабочего процесса, что выражается
в повышенной частоте срабатывания распределителей. Согласно данным Varseveld R.B., Bone G.M. и
Shiee M. \cite{pwm:Varseveld, Shiee}, обеспечение точности позиционирования в диапазоне 0,1~--~1,0 мм требует интенсификации
переключений распределителей до 100~--~150 раз за цикл, что значительно сокращает ресурс электромагнитных
клапанов и повышает вероятность отказов. Данная проблема особенно актуальна в условиях высокой цикличности
производственных процессов, типичной для современных автоматизированных линий.

Кроме того, увеличение быстродействия таких пневмоприводов при наличии инерционной нагрузки приводит к
увеличению колебательности процессов, что снижает точность позиционирования. Проведенные экспериментальные
исследования \cite{Greshnyakov2016,Hodgson:article1} показывают, что уже при
инерционных нагрузках 1~--~10 кг точность позиционирования уменьшается на 20~--~40\%.

Таким образом позиционные пневмоприводы с дискретными распределителями
характеризуются высокой конфликтностью показателей качества, оценивающих
быстродействие, точность позиционирования и интенсивность переключения распределителей.
Традиционные подходы к разрешению данного противоречия основаны преимущественно на применении
дополнительных специальных тормозных и разгонных устройств \cite{крутиков:способы_торможения_12,DaoTheAnh2016,Grishchenko2010}. Однако это существенно
усложняет конструкцию пневмопривода и значительно увеличивают стоимость системы, нивелируя ключевые
преимущества использования дискретных распределителей.

В тоже время ряд работ \cite{Barth:PWM_pneumatit, Nazari:Position_tracking}
показывает, что улучшения качества рабочего процесса и повышения
статико-динамических характеристик во многих случаях можно достичь за счет применения
современных методов управления без использования специальных дополнительных устройств
торможения и разгона. Существующие алгоритмические решения, рассмотренные в работах
Elsayed D., Hodgson S. и Zhonglin L. \cite{Elsayed,Zhonglin,Hodgson:article1}, демонстрируют значительный потенциал
в повышении точности позиционирования, быстродействия и снижения интенсивности переключения
дискретных распределителей позиционных пневмоприводов. Однако существующие разработки
направлены только на улучшение отдельных показателей и не рассматривают проблему повышения
статико-динамических и ресурсных показателей в комплексе. В настоящее время не исследованы
вопросы влияния улучшения статико-динамических характеристик позиционного пневмопривода с
дискретными распределителями на качество его рабочего процесса, определяемое частотой срабатывания
распределителей и приводящее к ухудшению ресурсных показателей.

В этой связи особую актуальность приобретает решение проблемы комплексного улучшения показателей
быстродействия, точности и ресурса позиционных пневмоприводов с дискретными распределителями с
учетом их конфликтности. Решение данной проблемы позволит научно обосновать выбор наилучшей
структуры пневмопривода, включая управляющую часть, а также оптимального сочетания параметров.
Также это даст возможность определить предельные возможности различных структур, что даст
возможность сократить сроки проектирования за счет обеспечения их быстрого выбора в зависимости
от предъявляемых требований.

\ifsynopsis

\else
	%Этот абзац появляется только в~диссертации.
\fi

% {\progress}
% Этот раздел должен быть отдельным структурным элементом по
% ГОСТ, но он, как правило, включается в описание актуальности
% темы. Нужен он отдельным структурынм элемементом или нет ---
% смотрите другие диссертации вашего совета, скорее всего не нужен.

{\aim}~диссертационной работы является комплексное повышение статико-динамических
и ресурсных показателей позиционного пневмопривода с дискретными распределителями в условиях их конфликтности.

Для~достижения поставленной цели необходимо было решить следующие {\tasks}:
\begin{enumerate}[beginpenalty=10000] % https://tex.stackexchange.com/a/476052/104425
	\item Разработать комплексную математическую модель пневмопривода с дискретными распределителями,
	      включающую силовую и управляющие структуры, учитывающую нелинейные эффекты
	      при функционировании и реализующую различные алгоритмы управления.

	\item Провести комплексный анализ различных структур пневмопривода для выявления оптимальных
	      режимов переключения при позиционировании рабочего органа.

	\item Провести натурный эксперимент для подтверждения работоспособности предлагаемых структур
	      и достоверности результатов, полученных по разработанной математической модели.

	\item Разработать методику многокритериального структурно-\allowbreak па\-ра\-ме\-три\-че\-ско\-го
	      синтеза, основанную на использовании фронтов Парето.

	\item Определить взаимосвязь между конфликтными статико-динамическими и ресурсными
	      показателями для позиционных пневмоприводов различной структуры на
	      основе анализа фронтов Парето, выявить области их предпочтительные применения.

	\item Сформировать практические рекомендации по выбору оптимальной структуры
	      позиционного пневмопривода с дискретными распределителями.

\end{enumerate}


{\novelty}
\begin{enumerate}[beginpenalty=10000] % https://tex.stackexchange.com/a/476052/104425

	\item Выявлены особенности функционирования пневмопривода при позиционировании выходного звена,
	      на основе которых предложена модифицированная структура ПИД-регулятора, включающая блок прогнозирования
	      тормозного пути и формирование упреждающего управляющего воздействия, что обеспечивает адаптивное торможение
	      и существенно снижает перерегулирование и колебания системы по сравнению с классической схемой ПИД-регулятора с ШИМ.

	\item Определены оптимальные режимы переключения дискретных распределителей, на основе
	      чего был разработан оригинальный способ прогнозного управления пневмоприводом с дискретными
	      распределителями, основанный на оптимизации последовательности их переключений.

	\item Предложена методика многокритериального структурно-\allowbreak па\-ра\-ме\-три\-че\-ско\-го
	      позиционных пневмоприводов с дискретными распределителями, основанная на использовании
	      фронтов Парето и отличающаяся тем, что для построения фронтов Парето используется
	      замещающая суррогатная нейросетевая модель.

	\item Определена взаимосвязь между конфликтными статико-динамическими и ресурсными
	      показателями для позиционных пневмоприводов различной структуры, что позволило выявить
	      их предельные возможности и выделить предпочтительные области их применения.
\end{enumerate}

{\influenceTheoretical}
Исследование расширяет теорию пневмоприводных систем в области использования
таких современных методов управления как модифицированный ШИМ, управление
в скользящих режимах, нечеткое и прогнозное управление, снижения вычислительной
сложности моделирования за счет применения замещающих суррогатных нейросетевых моделей.

{\influencePractical}
Реализация предложенных автором научно-технических решений в области построения
перспективных позиционных пневмоприводов с дискретными распределителями позволит:

\begin{itemize}
	\item улучшить точность позиционирования и быстродействие позиционных пневмоприводов
	      за счет применения предложенных способов управления переключением распределителей;

	\item повысить ресурс позиционных пневмоприводов благодаря
	      снижению интенсивности срабатывания распределителей;

	\item сократить сроки проектирования позиционных пневмоприводов за счет использования
	      выявленных областей предпочтительного применения различных структур и
	      предложенной методики многокритериального структурно-параметрического синтеза.
\end{itemize}

{\methods}
В рамках диссертационного исследования применялся комплексный подход,
сочетавший методы теоретических и эмпирических  исследований.

Теоретическая составляющая работы опиралась на фундаментальные положения теории
автоматического управления, дополненные современными методами математического моделирования
и оптимизации. Разработка математических моделей электропневматического привода базировалась
на классических законах термодинамики и механики сплошных сред, а анализ динамики системы
проводился с использованием методов качественной теории дифференциальных уравнений и теории
устойчивости. При синтезе алгоритмов управления применялись методы теории скользящих режимов,
нечеткой логики и прогнозного управления.

В области машинного обучения исследование опиралось на методы глубокого обучения и нейросетевого
моделирования, использованные для построения суррогатных моделей. Многокритериальная оптимизация
параметров управления реализовывалась на основе теории Парето-оптимальности.

Экспериментальная часть исследования проводилась на специально разработанном лабораторном стенде,
оснащенном современными средствами измерения и системами сбора данных. Обработка и интерпретация
результатов осуществлялась с использованием методов математической статистики и регрессионного анализа.

{\defpositions}
\begin{enumerate}[beginpenalty=10000] % https://tex.stackexchange.com/a/476052/104425
	\item Комплексная математическая модель позиционного пневматического привода с дискретными
	      распределителями, включающая силовую и управляющие структуры, учитывающая нелинейные
	      термодинамические процессы в полостях пневмоцилиндра, переменную структуру системы при
	      переключении распределителей, а также динамические эффекты сухого и вязкого трения, что
	      обеспечивает высокую степень согласованности с экспериментальными данными.

	\item Модифицированная структура ПИД-регулятора с ШИМ, включающая блок прогнозирования тормозного
	      пути и формирование упреждающего управляющего воздействия, что обеспечивает существенное снижение
	      перерегулирования и уменьшение колебательности системы по сравнению с классической
	      схемой ПИД-регулятора для пневмопривода с дискретными распределителями.


	\item Способ прогнозного управления пневмоприводом с дискретными распределителями, основанный
	      на оптимизации последовательности переключений распределителей с учетом прогнозируемой
	      динамики системы на заданном горизонте, обеспечивающий значительное повышение точности
	      позиционирования при одновременном сокращении числа переключений
	      распределителей по сравнению с традиционными алгоритмами управления.

	\item Методика многокритериального структурно-параметрического синтеза позиционного пневмопривода,
	      основанная на использовании фронтов Парето, обеспечивающая нахождение оптимальной структуры и
	      компромисса между конфликтующими показателями качества: точностью позиционирования,
	      быстродействием и ресурсными показателями системы.

	\item Результаты комплексного сравнительного анализа различных структур позиционных
	      пневмоприводов с дискретными распределителями, позволяющие на основе полученных фронтов
	      Парето определить предпочтительные области применения каждой структуры управления
	      в зависимости от приоритетных требований к системе.
\end{enumerate}

{\reliability}
результатов исследования определяется использованием общепринятых положений теории автоматического управления и
пневматических приводов, а также современных информационно-управляющих и программно-аппаратных средств.

Обоснованность выводов подтверждается согласованностью данных, полученных в результате математического
моделирования и экспериментальных исследований, проведенных на специально разработанном стенде.
Воспроизводимость результатов обеспечивалась многократным повторением экспериментов в идентичных условиях,
а полученные научные положения согласуются с результатами исследований других авторов в
смежных областях и не противоречат фундаментальным принципам теории управления и термодинамики.

В ходе исследования корректно применялись апробированные в научной практике методы математического
моделирования и оптимизации, включая методы теории скользящих режимов, нечеткой логики,
прогнозного управления и многокритериальной оптимизации на основе фронтов Парето. Достоверность
разработанной математической модели пневмопривода подтверждается высокой степенью
согласованности расчетных и экспериментальных данных.

{\probation}
Основные результаты работы докладывались~на:
\begin{itemize}
	\item заседаниях кафедры гидромеханики и гидравлических машин им.
	      В.С.~Квятковского ФГБОУ ВО <<НИУ~<<МЭИ>>, 2021~--~2024~гг.;

	\item XXV, XXVI международных научно-технических конференциях <<Гидромашины,
	      гидроприводы и гидропневмоавтоматика>>, 2021,~2022~гг., г. Москва, ФГБОУ ВО <<НИУ~<<МЭИ>>;

	\item XXVII, XXVIII международных научно-технических конференциях <<Гидравлические и теплотехнические
	      системы и агрегаты>>, 2023,~2024~гг., г. Москва, ФГБОУ ВО <<НИУ~<<МЭИ>>;

	\item XXVII, XXVIII, XXIX международных научно-технических конференциях студентов и
	      аспирантов «Радиоэлектроника, электротехника и энергетика», 2021~--~2023 гг., г. Москва, ФГБОУ ВО «НИУ «МЭИ»;

	\item XII всероссийской научно-технической конференции <<Гидромашины, гидроприводы и гидропневмоавтоматика>>,
	      2022~г. г. Санкт-Петербург, ФГАОУ ВО <<СПбПУ>>;

	\item научно-технических конференциях «Гидравлика», 2021,~2023~гг., г.~Москва, МГТУ
	      им. Н.Э. Баумана;

\end{itemize}

% {\contribution}

\ifnumequal{\value{bibliosel}}{0}
{%%% Встроенная реализация с загрузкой файла через движок bibtex8. (При желании, внутри можно использовать обычные ссылки, наподобие `\cite{vakbib1,vakbib2}`).
	{\publications} Основные результаты по теме диссертации изложены
	в~XX~печатных изданиях,
	X из которых изданы в журналах, рекомендованных ВАК,
	X "--- в тезисах докладов.
}%
{%%% Реализация пакетом biblatex через движок biber
	\begin{refsection}[bl-author, bl-registered]
		% Это refsection=1.
		% Процитированные здесь работы:
		%  * подсчитываются, для автоматического составления фразы "Основные результаты ..."
		%  * попадают в авторскую библиографию, при usefootcite==0 и стиле `\insertbiblioauthor` или `\insertbiblioauthorgrouped`
		%  * нумеруются там в зависимости от порядка команд `\printbibliography` в этом разделе.
		%  * при использовании `\insertbiblioauthorgrouped`, порядок команд `\printbibliography` в нём должен быть тем же (см. biblio/biblatex.tex)
		%
		% Невидимый библиографический список для подсчёта количества публикаций:
		\phantom{\printbibliography[heading=nobibheading, section=1, env=countauthorvak,          keyword=biblioauthorvak]%
			\printbibliography[heading=nobibheading, section=1, env=countauthorwos,          keyword=biblioauthorwos]%
			\printbibliography[heading=nobibheading, section=1, env=countauthorscopus,       keyword=biblioauthorscopus]%
			\printbibliography[heading=nobibheading, section=1, env=countauthorconf,         keyword=biblioauthorconf]%
			\printbibliography[heading=nobibheading, section=1, env=countauthorother,        keyword=biblioauthorother]%
			\printbibliography[heading=nobibheading, section=1, env=countregistered,         keyword=biblioregistered]%
			\printbibliography[heading=nobibheading, section=1, env=countauthorpatent,       keyword=biblioauthorpatent]%
			\printbibliography[heading=nobibheading, section=1, env=countauthorprogram,      keyword=biblioauthorprogram]%
			\printbibliography[heading=nobibheading, section=1, env=countauthor,             keyword=biblioauthor]%
			\printbibliography[heading=nobibheading, section=1, env=countauthorvakscopuswos, filter=vakscopuswos]%
			\printbibliography[heading=nobibheading, section=1, env=countauthorscopuswos,    filter=scopuswos]}%
		%
		\nocite{*}%
		%
		{\publications} Основные результаты по теме диссертации изложены в~\arabic{citeauthor}~печатных изданиях,
		\arabic{citeauthorvak} из которых изданы в журналах, рекомендованных ВАК%
		\ifnum \value{citeauthorscopuswos}>0%
			, \arabic{citeauthorscopuswos} "--- в~периодических научных журналах, индексируемых Web of~Science и Scopus%
		\fi%
		\ifnum \value{citeauthorconf}>0%
			, \arabic{citeauthorconf} "--- в~тезисах докладов.
		\else%
			.
		\fi%
		\ifnum \value{citeregistered}=1%
			\ifnum \value{citeauthorpatent}=1%
				Зарегистрирован \arabic{citeauthorpatent} патент.
			\fi%
			\ifnum \value{citeauthorprogram}=1%
				Зарегистрирована \arabic{citeauthorprogram} программа для ЭВМ.
			\fi%
		\fi%
		\ifnum \value{citeregistered}>1%
			Зарегистрированы\ %
			\ifnum \value{citeauthorpatent}>0%
				\formbytotal{citeauthorpatent}{патент}{}{а}{}%
				\ifnum \value{citeauthorprogram}=0 . \else \ и~\fi%
			\fi%
			\ifnum \value{citeauthorprogram}>0%
				\formbytotal{citeauthorprogram}{программ}{а}{ы}{} для ЭВМ.
			\fi%
		\fi%
		% К публикациям, в которых излагаются основные научные результаты диссертации на соискание учёной
		% степени, в рецензируемых изданиях приравниваются патенты на изобретения, патенты (свидетельства) на
		% полезную модель, патенты на промышленный образец, патенты на селекционные достижения, свидетельства
		% на программу для электронных вычислительных машин, базу данных, топологию интегральных микросхем,
		% зарегистрированные в установленном порядке.(в ред. Постановления Правительства РФ от 21.04.2016 N 335)
	\end{refsection}%
	\begin{refsection}[bl-author, bl-registered]
		% Это refsection=2.
		% Процитированные здесь работы:
		%  * попадают в авторскую библиографию, при usefootcite==0 и стиле `\insertbiblioauthorimportant`.
		%  * ни на что не влияют в противном случае
		\nocite{patbib1}%patent
		\nocite{pub3}
		\nocite{pub4}
		\nocite{pub6}
		\nocite{pub8}
		\nocite{pub13}
		\nocite{pub14}
		\nocite{pub16}
		\nocite{pub19}
		\nocite{pub21}
		\nocite{pub22}
	\end{refsection}%
	%
	% Всё, что вне этих двух refsection, это refsection=0,
	%  * для диссертации - это нормальные ссылки, попадающие в обычную библиографию
	%  * для автореферата:
	%     * при usefootcite==0, ссылка корректно сработает только для источника из `external.bib`. Для своих работ --- напечатает "[0]" (и даже Warning не вылезет).
	%     * при usefootcite==1, ссылка сработает нормально. В авторской библиографии будут только процитированные в refsection=0 работы.
}
 % Характеристика работы по структуре во введении и в автореферате не отличается (ГОСТ Р 7.0.11, пункты 5.3.1 и 9.2.1), потому её загружаем из одного и того же внешнего файла, предварительно задав форму выделения некоторым параметрам

\textbf{Объем и структура работы.} Диссертация состоит из~введения,
\formbytotal{totalchapter}{глав}{ы}{}{},
заключения и
\formbytotal{totalappendix}{приложен}{ия}{ий}{}.
%% на случай ошибок оставляю исходный кусок на месте, закомментированным
%Полный объём диссертации составляет  \ref*{TotPages}~страницу
%с~\totalfigures{}~рисунками и~\totaltables{}~таблицами. Список литературы
%содержит \total{citenum}~наименований.
%
Полный объём диссертации составляет
\formbytotal{TotPages}{страниц}{у}{ы}{}, включая
\formbytotal{totalcount@figure}{рисун}{ок}{ка}{ков} и
\formbytotal{totalcount@table}{таблиц}{у}{ы}{}.
Список литературы содержит
\formbytotal{citenum}{наименован}{ие}{ия}{ий}.
