\chapter{ЭКСПЕРИМЕНТАЛЬНО ИССЛЕДОВАНИЕ ПНЕВМОПРИВОДА С ДИСКРЕТНЫМ УПРАВЛЕНИЕМ}\label{ch:ch5}
Основная цель данной части работы заключается в проверке разработанных в
теоретических разделах математических моделей и алгоритмов управления,
а также в оценке их эффективности на реальной установке. Проведение экспериментов позволяет
выявить особенности работы пневмопривода, определить степень влияния
нелинейных факторов на динамику системы и предложить рекомендации по
выбору оптимальных параметров управления для различных условий эксплуатации.

Исследование основывается на применении экспериментальной установки,
специально спроектированной для анализа динамических характеристик и
точности позиционирования привода. В рамках главы описывается конструкция
установки, включая аппаратную часть и программное обеспечение, детализируется
методика проведения экспериментов и приводятся результаты сравнительного
анализа различных алгоритмов управления. Полученные данные сопоставляются с
теоретическими расчетами, что позволяет оценить точность предложенных моделей
и обосновать выбор наиболее эффективных методов управления.

\section{Описание экспериментальной установки}\label{sec:ch5/sec1}
Экспериментальная установка, разработанная для исследования пневматического привода с
дискретным управлением, представляет собой комплекс технических средств и
программных решений, обеспечивающих проведение экспериментов с достаточной точностью и достоверностью.
Спроектированная установка позволяет реализовать различные режимы работы
пневмопривода, включая разгон, торможение и удержание, а также включая остальные режимы работы.
Проводить сравнительный анализ эффективности алгоритмов управления.

Экспериментальная установка включает в себя два основных компонента:
аппаратную часть, отвечающую за физическое выполнение процессов управления и
измерение параметров системы, и программное обеспечение, которое реализует
управление приводом, сбор данных и их анализ.

\subsection{Аппаратная чпасть установки}\label{sec:ch5/sec1/subsec1}
Аппаратная часть установки была разработана с целью обеспечения высокодетерминированного контура управления,
поддерживающего реализацию сложных алгоритмов регулирования, а также удобства интеграции широкого
спектра датчиков и исполнительных механизмов. Особое внимание уделено структурированию системы,
включающей микроконтроллерный узел для ввода-вывода, одноплатный компьютер с программируемыми в
реальном времени модулями (PRU) и вспомогательные элементы, обеспечивающие устойчивость и точность системы

В качестве центральной вычислительной платформы используется одноплатный компьютер BeagleBone Black.
Данный компонент избран благодаря наличию PRU -- специализированных модулей, работающих независимо от основного
ARM-ядра и предназначенных для выполнения задач в реальном времени с предельно
малой задержкой. Это решение позволило достичь требуемого уровня детерминизма в контуре управления.

К BeagleBone по высокоскоростной последовательной шине (SPI) подключён
микроконтроллер (STM32 или аналогичный), на котором сосредоточены
задачи низкоуровневой обработки сигналов. Подобное распределение обязанностей
даёт возможность физически и функционально отделить высокоскоростное взаимодействие
с внешними устройствами от вычислительных процессов высшего уровня.

На микроконтроллер STM32 возложены функции непосредственного взаимодействия с объектом управления.
С этой целью предусматриваются следующие аппаратные решения:


Аппаратную часть установки можно разделить на несколько основных блоков:

\subsection{Програмное обеспечение установки}\label{sec:ch5/sec1/subsec2}

\section{Методика проведения экспериментов}\label{sec:ch5/sec2}

\section{Экспериментальное исследование ШИМ управления с использованием ПИД регулятора}\label{sec:ch5/sec3}
\section{Экспериментальное исследование управления в скользящих режимах}\label{sec:ch5/sec4}
\section{Экспериментальное исследование нечеткого управления}\label{sec:ch5/sec5}
\section{Экспериментальное исследование прогнозного управления}\label{sec:ch5/sec6}
\section{Сравниттельный анализ методов управления с расчетными данными}\label{sec:ch5/sec7}
