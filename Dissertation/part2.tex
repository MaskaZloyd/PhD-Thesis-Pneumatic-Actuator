\chapter{СОСТАВЛЕНИЕ МАТЕМАТИЧЕСКОЙ МОДЕЛИ ПНЕВМОПРИВОДА}\label{ch:ch2}
Данная глава посвящена математическому моделированию ПП
с дискретными распределителями. Рассматриваются следующие ключевые аспекты:

\begin{enumerate}
    \item Структура и принцип работы исследуемого ПП;
    \item Моделирование пневмоцилиндра;
    \item Моделирование дискретных распределителей;
    \item Моделирование силы трения и упругих деформаций;
    \item Адаптация математической модели к эффективному численному расчету на ЭВМ.
\end{enumerate}

Математическая модель включает:

\begin{enumerate}
    \item Уравнения движения поршня;
    \item Уравнения изменения давления в полостях цилиндра;
    \item Уравнения изменения температуры рабочего тела в полостях цилиндра;
    \item Модель массового расхода воздуха;
    \item Модель динамики распределителей;
    \item Модель сил трения;
    \item Модель силы реакции опоры.
\end{enumerate}

Используются уравнения термодинамики и газовой динамики. Учитываются нелинейные эффекты:
сжимаемость воздуха, особенности течения через дросселирующие элементы, силы трения и реакции опоры.
Приводится методика идентификации параметров модели и ее верификация.

\section{Структура и паринцип работы исследуемого пневмопривода}\label{sec:ch2/sec1}

Исследуемый электропневматический привод включает пневматический цилиндр двустороннего действия
c односторонним штоком и четыре двухпозиционных распределителя.
Цилиндр содержит две рабочие полости, разделенные поршнем.

Ключевой особенностью привода является конфигурация распределителей.
К каждой полости цилиндра подключены два независимых распределителя:
один для подачи сжатого воздуха из магистрали, другой для выхлопа в атмосферу.
Такая конфигурация обеспечивает гибкое управление потоками воздуха в обеих полостях цилиндра.

[Рисунок: Принципиальная пневматическая схема привода с указанием подключения распределителей]

Каждый распределитель имеет два дискретных состояния: открыто и закрыто.
Общее количество возможных комбинаций состояний распределителей
определяется формулой:
\begin{equation*}
    N = 2^k,
\end{equation*}
где $N$ - число комбинаций, $k$ - количество распределителей.

Для рассматриваемой системы с четырьмя распределителями:
\begin{equation*}
    N = 2^4 = 16.
\end{equation*}

Таким образом, система имеет 16 дискретных состояний, что обеспечивает широкие возможности
управления при сохранении относительной простоты конструкции.

Система управления генерирует дискретные сигналы для активации
электромагнитных клапанов распределителей. Обратная связь по положению реализуется посредством датчика линейного перемещения на штоке цилиндра. Дополнительно могут применяться датчики давления в полостях цилиндра для повышения точности управления.

[Рисунок: Функциональная схема системы управления приводом]

Использование дискретных распределителей вместо пропорциональных
снижает стоимость системы, однако требует разработки более сложных
алгоритмов управления для компенсации нелинейного характера коммутации пневматических линий.

\section{Моделирование пневмоцилиндра}\label{sec:ch2/sec2}

При разработке математической модели пневмоцилиндра были приняты следующие основные допущения:
\begin{enumerate}
    \item Рабочим телом является идеальный газ (воздух), подчиняющийся уравнению состояния Клапейрона-Менделеева;
    \item Процессы в рабочих полостях пневмоцилиндра рассматриваются как адиабатические, теплообмен с окружающей средой не учитывается;
    \item Температура газа в магистрали и температура окружающей среды принимаются постоянными;
    \item Утечки газа через уплотнения поршня и штока не учитываются;
    \item Давление в выхлопной магистрали принимается равным атмосферному;
    \item Влияние сил тяжести на движение поршня не учитывается ввиду горизонтального расположения пневмоцилиндра;
    \item Пневматические линии между распределителями и рабочими полостями цилиндра считаются короткими,
          их объем пренебрежимо мал по сравнению с объемом рабочих полостей;
    \item Эффекты сжимаемости воздуха в трубопроводах не учитываются.
\end{enumerate}

С учетом принятых допущений, математическая модель пневмоцилиндра может быть представлена системой
дифференциальных уравнений, описывающих изменение давлений в рабочих полостях, температур газа и
движение поршня. Данная система уравнений формирует основу для дальнейшего анализа динамики пневмопривода
и синтеза алгоритмов управления.

\subsection{Уравнение движения пневмоцилиндра}\label{sec:ch2/sec2/subsec1}

Рассмотрим пневмоцилиндр как систему с одной степень свободы. Обобщенной координатой выберем положение поршня $x$.
Для применения принципа Гамильтона необходимо составить функцию Лагранжа:
\begin{equation}\label{eq:ch2/eq0}
    L = T - U,
\end{equation}
где $T$ -- кинетическая энергия системы; $U$ -- потенциальная энергия системы; $L$ -- функция Лагранжа.

Кинетическую энергию системы можно представить в виде:
\begin{equation}\label{eq:ch2/eq1}
    T = \frac{M}{2} \dot{x}^2,
\end{equation}
где $M$ -- масса подвижных частей системы; $\dot{x}$ -- скорость движения поршня.

Потенциальная энергия системы включает в себя работу силы давления газа в рабочих полостях цилиндра:

\begin{equation}\label{eq:ch2/eq2}
    U = p_1 V_1 + p_2 V_2 + p_\text{атм} (V_1 - V_2),
\end{equation}
где $p_1$ и $p_2$ -- давления в рабочих полостях цилиндра;
$V_1$ и $V_2$ -- объемы рабочих полостей;
$p_\text{атм}$ -- атмосферное давление.

Объемы рабочих полостей связаны с положением поршня следующим образом:
\begin{equation}\label{eq:ch2/eq3}
    \begin{aligned}
        V_1 & = F_1 (x - x_0),     \\
        V_2 & = F_2 (L - x + x_0),
    \end{aligned}
\end{equation}
где $F_1=\pi D_{поршня}^2/4 $ и $F_2 = \pi (D_{поршня}^2/4 - D_{штока}^2/4)$ -- активные площади поперечного сечения поршня в поршневой и штоковой полостях соответственно;
$x_0$ -- начальное положение поршня;
$L$ -- ход поршня.

Подставляя \eqref{eq:ch2/eq1}, \eqref{eq:ch2/eq2} и \eqref{eq:ch2/eq3} в \eqref{eq:ch2/eq0}, получим функцию Лагранжа:

\begin{equation}\label{eq:ch2/eq4}
    \begin{aligned}
        L & = \frac{M}{2} \dot{x}^2 - p_1 F_1 (x - x_0) - p_2 F_2 (L - x + x_0) \\
          & - p_\text{атм} (F_1 (x - x_0) - F_2 (L - x + x_0)).
    \end{aligned}
\end{equation}

Обобщенную силу $Q_x$, действующую на систему учитывющую силу трения и реакции упоров, можно представить в виде:
\begin{equation}\label{eq:ch2/eq5}
    Q_x = - R_{\text{тр}} - R_{\text{упор}},
\end{equation}
где $R_{\text{тр}}$ -- сила трения; $R_{\text{упор}}$ -- сила реакции упора.

Запишем уравнение лагранжа в общем виде:

\begin{equation}\label{eq:ch2/eq6}
    \frac{d}{dt} \left( \frac{\partial L}{\partial \dot{x}} \right) - \frac{\partial L}{\partial x} = Q_x.
\end{equation}

Вычислим частные производные функции Лагранжа \eqref{eq:ch2/eq4} по обобщенной координате $x$ и ее производной $\dot{x}$:

\begin{equation}\label{eq:ch2/eq7}
    \begin{aligned}
        \frac{\partial L}{\partial \dot{x}} & = M \dot{x},                                    \\
        \frac{\partial L}{\partial x}       & = p_1 F_1 - p_2 F_2 - p_\text{атм} (F_1 - F_2).
    \end{aligned}
\end{equation}

Подставляя \eqref{eq:ch2/eq7} и \eqref{eq:ch2/eq5} в \eqref{eq:ch2/eq6}, получим уравнение движения пневмоцилиндра:

\begin{equation}\label{eq:ch2/eq8}
    M \ddot{x} = p_1 F_1 - p_2 F_2 - p_\text{атм} (F_1 - F_2) - R_{\text{тр}} - R_{\text{упор}}.
\end{equation}
где $\ddot{x}$ -- ускорение поршня.

\subsection{Уравнения изменения давлений в полостях пневмоцилиндра}\label{sec:ch2/sec2/subsec2}

Первоначально получим дифференциальные уравнения изменения давлений в
полостях пневмоцилиндра, основываясь на началах термодинамики
и уравнения Максвелла с учетом ранее принятых допущений.
Ключевыми допущениями для данного вывода являются: рассмотрение
воздуха как идеального газа, адиабатический характер процессов в полостях
цилиндра, отсутствие утечек газа через уплотнения, и пренебрежение объемом
пневматических линий между распределителями и рабочими полостями.

Согласно первому началу термодинамики для открытой системы, изменение внутренней энергии рабочего тела
описывается уравнением:

\begin{equation}
    dU = \partial Q - \partial W + hdm,
\end{equation}
где $dU$ - изменение внутренней энергии;
$δQ$ -- подведенное тепло;
$δW$ -- совершенная работа;
$h$ -- удельная энтальпия;
$dm$ -- изменение массы системы.

Учитвая, что процесс рассматривается как адиабатический, то $δQ = 0$ и
работа совершается только за счет изменения объема газа в полостях цилиндра:

\begin{equation*}
    \partial W = -pdV.
\end{equation*}

Тогда уравнение изменения внутренней энергии примет вид:

\begin{equation*}
    dU = -pdV + hdm.
\end{equation*}

Используя определение энтальпии $H - U = pV$, получим:

\begin{equation*}
    d(H-pV) = -pdV + hdm.
\end{equation*}

Раскрывая дифференциалы, получим:

\begin{equation*}
    dH - pdV - Vdp = -pdV + hdm.
\end{equation*}

Сокращая слагаемые, получим уравнение изменения энтальпии:

\begin{equation}\label{eq:ch2/eq9}
    dH = Vdp + hdm.
\end{equation}

Выражение \eqref{eq:ch2/eq9} позволяет описать изменение энтальпии рабочего тела в процессе сжатия и
расширения, а также в при изменении массы рабочего тела.

Обратися к уравнению Максвелла для энтальпии, которое связывает изменение энтальпии с изменением давления и температуры:

\begin{equation}\label{eq:ch2/eq10}
    dH = \left(
        \frac{\partial H}{\partial p}
    \right)_T dp + \left(
        \frac{\partial H}{\partial T}
    \right)_p dT.
\end{equation}

Приравнивая правые части уравнений \eqref{eq:ch2/eq9} и \eqref{eq:ch2/eq10}, получаем:

\begin{equation}\label{eq:ch2/eq11}
    \left(
        \frac{\partial H}{\partial p}
    \right)_T dp + \left(
        \frac{\partial H}{\partial T}
    \right)_p dT = Vdp + hdm.
\end{equation}

На данном этапе воспольуемся одним из соотношений Максвелла:

\begin{equation*}
    \left(
        \frac{\partial H}{\partial p}
    \right)_T = V - T \left(
        \frac{\partial V}{\partial T}
    \right)_p.
\end{equation*}

Подставляя это выражение в \eqref{eq:ch2/eq11}, получим:

\begin{equation*}
    \left( V - T \left(
            \frac{\partial V}{\partial T}
        \right)_p \right) dp + \left(
        \frac{\partial H}{\partial T}
    \right)_p dT = Vdp + hdm.
\end{equation*}

После упрощения и группировки членов, приходим к следующему выражению:
\begin{equation}\label{eq:ch2/eq12}
    -T \left(
        \frac{\partial V}{\partial T}
    \right)_p dp + \left(
        \frac{\partial H}{\partial T}
    \right)_p dT = hdm.
\end{equation}

Здесь можно воспользоваться еще одним важным термодинамически соотножениемЖ
\begin{equation}\label{eq:ch2/eq13}
    \left(
        \frac{\partial H}{\partial T}
    \right)_p = c_p,
\end{equation}
где $c_p$ -- удельная теплоемкость при постоянном давлении.

Подставляя \eqref{eq:ch2/eq13} в уравнение \eqref{eq:ch2/eq12}, получим:
\begin{equation}\label{eq:ch2/eq14}
    -T \left(
        \frac{\partial V}{\partial T}
    \right)_p dp + c_p dT = hdm.
\end{equation}

Для идеального газа справедливо соотношение:

\begin{equation*}
    \left(\frac{\partial V}{\partial T}\right)_p = \frac{R}{p},
\end{equation*}
где $R$ -- газовая постоянная.

Используя данное выражение, преобразуем уравнение:

\begin{equation*}
    -\frac{RT}{p} dp + c_p dT = h dm.
\end{equation*}

Разделив обе части на $dt$ и учитывая, что $dm/dt = G$ (массовый расход), получаем дифференциальное уравнение:

\begin{equation*}
    -\frac{RT}{p} \frac{dp}{dt} + c_p \frac{dT}{dt} = hG.
\end{equation*}

Теперь обратимся к уравнению состояния идеального газа $pV = mRT$. Дифференцируя его по времени, получаем:

\begin{equation*}
    V\frac{dp}{dt} + p\frac{dV}{dt} = RT\frac{dm}{dt} + mR\frac{dT}{dt}.
\end{equation*}

Подставляя выражение для $dT/dt$ из предыдущего уравнения и учитывая, что для идеального газа

\begin{equation*}
    c_p - c_v = R,
\end{equation*}
где $c_v$ -- удельная теплоемкость при постоянном объеме, после ряда алгебраических преобразований получаем:

\begin{equation}\label{eq:ch2/eq15}
    \frac{dp}{dt} = \frac{\gamma}{V}\left(RT\frac{dm}{dt} - p\frac{dV}{dt}\right),
\end{equation}
где $\gamma = c_p/c_v$ -- показатель адиабаты.

Теперь учтем, что к каждой полости пневмоцилиндра подключены два независимых распределителя:
один для подачи сжатого воздуха из магистрали, другой для выхлопа в атмосферу.

Для полного описания динамики пневмоцилиндра записываем уравнение \eqref{eq:ch2/eq15} для обеих полостей:

\begin{equation}\label{eq:ch2/eq16}
    \begin{cases}
        \begin{aligned}
            \frac{dp_1}{dt} & = \frac{\gamma}{V_1}\left(RT_1\frac{dm_1}{dt} - p_1\frac{dV_1}{dt}\right), \\
            \frac{dp_2}{dt} & = \frac{\gamma}{V_2}\left(RT_2\frac{dm_2}{dt} - p_2\frac{dV_2}{dt}\right),
        \end{aligned}
    \end{cases}
\end{equation}
где индексы 1 и 2 соответствуют левой и правой полостям пневмоцилиндра.

Изменение массы газа в каждой полости определяется суммарным массовым
расходом через оба распределителя, подключенных к этой полости:

\begin{equation}\label{eq:ch2/eq17}
    \begin{cases}
        \begin{aligned}
            \frac{dm_1}{dt} & = G_{1\text{вх}} - G_{1\text{вых}}, \\
            \frac{dm_2}{dt} & = G_{2\text{вх}} - G_{2\text{вых}},
        \end{aligned}
    \end{cases}
\end{equation}
где $G_{1\text{вх}}$ и $G_{2\text{вых}}$ -- массовые расходы воздуха, поступающего в полости через впускные распределители;
$G_{1\text{вх}}$ и $G_{2\text{вых}}$ -- массовые расходы воздуха, выходящего из полостей через выпускные распределители.

Изменение объемов полостей связано с движением поршня:

\begin{equation*}
    \frac{dV_1}{dt} = F_1\frac{dx}{dt}, \quad \frac{dV_2}{dt} = -F_2\frac{dx}{dt},
\end{equation*}
где $A_1$ и $A_2$ -- эффективные площади поршня в левой и правой полостях соответственно.

Таким образом, окончательная система уравнений, описывающая изменение
давлений в полостях пневмоцилиндра с учетом конфигурации распределителей, принимает вид:
\begin{equation}
    \begin{aligned}
        \frac{dp_1}{dt} & = \frac{\gamma}{V_1}\left(RT_1(G_{1\text{вх}} - G_{1\text{вых}}) - p_1 F_1\frac{dx}{dt}\right), \\
        \frac{dp_2}{dt} & = \frac{\gamma}{V_2}\left(RT_2(G_{2\text{вх}} - G_{2\text{вых}}) + p_2 F_2\frac{dx}{dt}\right).
    \end{aligned}
\end{equation}

\subsection{Уравнения изменения температур в полостях пневмоцилиндра}\label{sec:ch2/sec2/subsec3}

Для вывода уравнений изменения температур в полостях пневмоцилиндра аналогично воспользуется
термодинамическими уравнениями Максвелла.

Запишем уравнение изобарического расширения:
\begin{equation*}
    \left(\frac{\partial T}{\partial p}\right)_V = \frac{T}{c_p} \left(\frac{\partial V}{\partial T}\right)_p,
\end{equation*}
где  $c_p$ -- удельная теплоемкость при постоянном давлении.

Второе уравнение Максвелла, известное как соотношение изохорического нагрева, записывается как:

\begin{equation*}
    \left(\frac{\partial T}{\partial V}\right)_p = -\frac{T}{c_V} \left(\frac{\partial p}{\partial T}\right)_V.
\end{equation*}
где $c_V$ -- удельная теплоемкость при постоянном объеме.

Для описания изменения температуры используется уравнение полного дифференциала температуры:

\begin{equation*}
    dT = \left(\frac{\partial T}{\partial p}\right)_V dp + \left(\frac{\partial T}{\partial V}\right)_p dV.
\end{equation*}

Подставляя в это выражение уравнения Максвелла, получаем расширенное уравнение изменения температуры:

\begin{equation*}
    dT = \frac{T}{c_p} \left(\frac{\partial V}{\partial T}\right)_p dp - \frac{T}{c_V} \left(\frac{\partial p}{\partial T}\right)_V dV.`'
\end{equation*}

Для дальнейшего преобразования применяется уравнение состояния идеального газа:

\begin{equation}\label{eq:ch2/klayperon_mendeleev}
    pV = mRT.
\end{equation}

Дифференцируя уравнение \eqref{eq:ch2/klayperon_mendeleev} по температуре при постоянном объеме, получаем соотношение изохорического давления:

\begin{equation*}
    \left(\frac{\partial p}{\partial T}\right)_V = \frac{mR}{V} = \frac{p}{T}.
\end{equation*}

Аналогично, дифференцируя уравнение \eqref{eq:ch2/klayperon_mendeleev} по температуре при постоянном давлении,
получаем соотношение изобарического объема:

\begin{equation*}
    \left(\frac{\partial V}{\partial T}\right)_p = \frac{mR}{p} = \frac{V}{T}.
\end{equation*}

Подставляя эти выражения в расширенное уравнение изменения температуры, получаем:

\begin{equation*}
    dT = \frac{T}{c_p} \cdot \frac{V}{T} \cdot dp - \frac{T}{c_V} \cdot \frac{p}{T} \cdot dV.
\end{equation*}

Упрощая это выражение и переходя к производным по времени, получаем дифференциальное уравнение изменения температуры:

\begin{equation*}
    \frac{dT}{dt} = \frac{V}{c_p} \cdot \frac{dp}{dt} - \frac{p}{c_V} \cdot \frac{dV}{dt}.
\end{equation*}

Для дальнейшего преобразования учитывается соотношение теплоемкостей идеального газа, известное как показатель адиабаты:

\begin{equation*}
    \frac{c_p}{c_V} = \gamma.
\end{equation*}

Используя это соотношение, выражаем теплоемкости через газовую постоянную и показатель адиабаты:

\begin{equation*}
    c_p = \frac{\gamma}{\gamma - 1} R, \quad c_V = \frac{1}{\gamma - 1} R.
\end{equation*}

Подставляя эти выражения в дифференциальное уравнение изменения температуры и учитывая уравнение состояния идеального газа, после алгебраических преобразований получаем окончательное уравнение для изменения температуры в полости пневмоцилиндра, которое можно назвать уравнением термодинамического состояния полости:

\begin{equation*}
    \frac{dT}{dt} = \frac{T}{p} \cdot \frac{dp}{dt} - (\gamma - 1) \frac{T}{V} \cdot \frac{dV}{dt}.
\end{equation*}

Это уравнение применимо к обеим полостям пневмоцилиндра. Для левой и правой полостей соответственно
получаем уравнения термодинамического состояния левой и правой полостей:

\begin{equation*}
    \begin{cases}
        \begin{aligned}
            \frac{dT_1}{dt} & = \frac{T_1}{p_1} \cdot \frac{dp_1}{dt} - (\gamma - 1) \frac{T_1}{V_1} \cdot \frac{dV_1}{dt}, \\
            \frac{dT_2}{dt} & = \frac{T_2}{p_2} \cdot \frac{dp_2}{dt} - (\gamma - 1) \frac{T_2}{V_2} \cdot \frac{dV_2}{dt},
        \end{aligned}
    \end{cases}
\end{equation*}
где индексы 1 и 2 относятся к левой и правой полостям пневмоцилиндра соответственно.



\section{Моделирование силы трения}\label{sec:ch2/sec2/subsec4}

\section{Моделирование силы реакции опоры}\label{sec:ch2/sec2/subsec5}

\section{Моделирование дискретных распределителей}\label{sec:ch2/sec3}

\subsection{Уравнения массового расхода рабочего тела}\label{sec:ch2/sec3/subsec1}

\subsection{Динамика переключения распределителей}\label{sec:ch2/sec3/subsec2}


\section{Адаптация математической модели к эффективному численному расчету на ЭВМ}\label{sec:ch2/sec5}

\section{Верификация математической модели}\label{sec:ch2/sec6}
