\section{Постановка задачи структурно-параметрического синтеза пневмопривода}

Структурно-параметрический синтез позиционного пневмопривода с дискретными распределителями
представляет собой комплексную задачу, предполагающую выбор оптимальной структуры управления и определение её
параметров для обеспечения требуемых показателей качества работы системы. Особенностью данной задачи является
необходимость учёта противоречивых требований к статико-динамическим и ресурсным показателям пневмопривода,
что предопределяет её многокритериальный характер.

Требования к позиционному пневмоприводу с дискретными распределителями могут
быть формализованы в виде совокупности критериев, отражающих различные аспекты его функционирования.
Исходя из проведённого в предыдущих главах анализа, основные требования к системе могут быть разделены на три ключевые группы:

\begin{enumerate}
	\item \textbf{Требования к точности позиционирования}, выражаемые через предельно допустимую ошибку в
	      установившемся режиме $\Delta x_{\text{доп}}$.
	\item \textbf{Требования к качеству переходного процесса}, включающие ограничения по быстродействию
	      $t_{\text{п}} \leq t_{\text{п.доп}}$, перерегулированию $\sigma \leq \sigma_{\text{доп}}$ и другим динамическим показателям.
	\item \textbf{Ресурсные требования}, характеризующие интенсивность
	      переключений распределителей $N_{\text{перекл}} \leq N_{\text{перекл.доп}}$ за цикл позиционирования.
\end{enumerate}

Таким образом, обобщённая постановка задачи структурно-параметрического
синтеза может быть представлена следующим образом:
\begin{equation}
	F_{\text{опт}} = \text{arg}\,\text{min}_{S \in \mathbf{S}, \mathbf{p} \in \mathbf{P}_S} \mathbf{J}(S, \mathbf{p}),
\end{equation}
где $F_{\text{опт}}$ -- оптимальная конфигурация пневмопривода;
$S$ -- структура системы управления из множества допустимых структур $\mathbf{S}$;
$\mathbf{p}$ -- вектор параметров выбранной структуры управления из множества допустимых значений $\mathbf{P}_S$;
$\mathbf{J}(S, \mathbf{p})$ -- векторный критерий оптимальности.
\nomenclature{$F_{\text{опт}}$}{Оптимальная конфигурация пневмопривода\nomrefeqpage}
\nomenclature{$S$}{Структура системы управления\nomrefeqpage}
\nomenclature{$\mathbf{p}$}{Вектор параметров выбранной структуры управления\nomrefeqpage}
\nomenclature{$\mathbf{J}(S, \mathbf{p})$}{Векторный критерий оптимальности\nomrefeqpage}

Для различных структур управления необходимо определить соответствующие векторы варьируемых параметров:
\begin{itemize}
	\item для ПИД-регулятора с ШИМ: $\mathbf{p}_{\text{ПИД}} = [K_p, K_i, K_d, f_{\text{ШИМ}}, K_{\text{т}}]^T$;
	\item для управления в скользящих режимах с интегральной поверхностью: $\mathbf{p}_{\text{УСР-И}} = [\lambda_1, \lambda_2, \varepsilon_1, \varepsilon_2, \ldots, \varepsilon_n]^T$;
	\item для управления в скользящих режимах с терминальной поверхностью: $\mathbf{p}_{\text{УСР-Т}} = [\beta, \gamma, p, q, \varepsilon_1, \varepsilon_2, \ldots, \varepsilon_n]^T$;
	\item для нечёткого управления: $\mathbf{p}_{\text{НУ}} = [\mathbf{x}_m, \mathbf{y}_m, \mathbf{z}_m, \mathbf{w}]^T$;
	\item для прогнозного управления: $\mathbf{p}_{\text{MPC}} = [N_p, N_u, Q, R, \Delta t]^T$.
\end{itemize}
где компоненты векторов представляют собой настраиваемые параметры соответствующих алгоритмов управления, описанные в главе 3.

Для оценки эффективности функционирования различных структур
пневмопривода с дискретными распределителями необходимо определить
количественные критерии, отражающие основные требования к системе.
В рамках данного исследования в качестве критериев оптимизации предлагается использовать точность позиционирования,
интегральный критерий качества переходного процесса и интенсивность переключений распределителей.

\textbf{Точность позиционирования ($AC$).}
Критерий точности позиционирования характеризует статическую ошибку системы в установившемся режиме и определяется как:
\begin{equation}
	AC = \lim_{t \to \infty} |e(t)| = \lim_{t \to \infty} |x_{\text{зад}} - x(t)|,
\end{equation}
где $x_{\text{зад}}$ -- заданное положение, а $x(t)$ -- текущее положение штока пневмоцилиндра.

На практике, учитывая конечное время моделирования $T_{\text{мод}}$, данный критерий вычисляется как:
\begin{equation}
	AC = |e(T_{\text{мод}})| = |x_{\text{зад}} - x(T_{\text{мод}})|,
\end{equation}
при условии, что время моделирования выбрано достаточно большим для достижения установившегося состояния.
\nomenclature{$AC$}{Точность позиционирования}

\textbf{Интегральный критерий качества переходного процесса ($ITAE$).}
Для оценки качества переходного процесса предлагается использовать интегральный критерий с учётом времени и модуля ошибки:
\begin{equation}
	ITAE = \int_0^{T_{\text{мод}}} t |e(t)| dt,
\end{equation}
где $t$ -- время, $|e(t)|$ -- модуль ошибки позиционирования.
\nomenclature{$ITAE$}{Интегральный критерий качества переходного процесса}

Данный критерий обеспечивает комплексную оценку динамических свойств системы, учитывая как
быстродействие, так и характер затухания переходного процесса. Использование модуля ошибки
позволяет адекватно оценивать качество процессов с колебаниями относительно установившегося значения.

\textbf{Интенсивность переключений распределителей ($SI$).}
Для оценки ресурсных показателей системы предлагается критерий,
характеризующий интенсивность переключений распределителей:
\begin{equation}
	SI = \sum_{i=1}^4 N_i,
\end{equation}
где $N_i$ -- количество переключений $i$-го распределителя за цикл позиционирования.
\nomenclature{$SI$}{Интенсивность переключений распределителей}

Данный критерий напрямую связан с ресурсом электромагнитных клапанов и общей надёжностью системы.
Минимизация интенсивности переключений способствует увеличению срока службы пневмопривода и снижению эксплуатационных затрат.

Ключевой особенностью рассматриваемой задачи является наличие противоречий между представленными критериями
оптимизации. Данные противоречия обусловлены физическими особенностями пневмопривода с дискретными распределителями и проявляются в следующих взаимосвязях:

\begin{enumerate}
	\item \textbf{Противоречие между точностью позиционирования и интенсивностью переключений.} Повышение
	      точности позиционирования ($AC \to \min$) требует более частого переключения распределителей
	      для точной коррекции положения штока, что приводит к увеличению $SI$.

	\item \textbf{Противоречие между качеством переходного процесса и интенсивностью переключений.}
	      Улучшение динамических характеристик системы ($ITAE \to \min$) обычно достигается за счёт
	      более агрессивного управления, что также увеличивает число переключений $SI$.

	\item \textbf{Взаимосвязь между точностью и качеством переходного процесса.} Данная взаимосвязь более сложная и может иметь
	      различный характер в зависимости от структуры управления. В некоторых случаях улучшение динамических
	      характеристик способствует повышению точности, в других -- между этими критериями также возникает противоречие.
\end{enumerate}

Помимо указанных взаимосвязей, при решении задачи структурно-параметрического
синтеза необходимо учитывать ряд ограничений:

\begin{enumerate}
	\item \textbf{Конструктивные ограничения} на параметры пневмопривода,
	      связанные с физическими свойствами системы:
	      \begin{equation}
		      \mathbf{g}_{\text{конст}}(\mathbf{p}) \leq 0,
	      \end{equation}
	      где $\mathbf{g}_{\text{конст}}$ -- вектор-функция конструктивных ограничений.

	\item \textbf{Ограничения на параметры алгоритмов управления}, определяемые
	      требованиями к устойчивости системы и реализуемости алгоритмов:
	      \begin{equation}
		      \mathbf{p}_{\min} \leq \mathbf{p} \leq \mathbf{p}_{\max},
	      \end{equation}
	      где $\mathbf{p}_{\min}$ и $\mathbf{p}_{\max}$ -- векторы минимальных и максимальных значений параметров.

	\item \textbf{Ограничения на показатели качества системы}, задаваемые техническим заданием:
	      \begin{equation}
		      \begin{aligned}
			      AC   & \leq AC_{\text{доп}}   \\
			      ITAE & \leq ITAE_{\text{доп}} \\
			      SI   & \leq SI_{\text{доп}}
		      \end{aligned}
	      \end{equation}
\end{enumerate}

Учитывая многокритериальный характер задачи и наличие противоречий между критериями,
целесообразно применить подход, основанный на выявлении множества Парето-оптимальных решений,
которые представляют собой компромисс между различными критериями. При этом выбор конкретного
решения из множества Парето-оптимальных будет зависеть от приоритетов конкретной задачи позиционирования.

Таким образом, задача структурно-параметрического синтеза пневмопривода может быть сформулирована как
нахождение множества Парето-оптимальных решений для различных структур управления и
последующий выбор наилучшей структуры и её параметров на основе заданных приоритетов задачи.
