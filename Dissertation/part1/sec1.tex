
\section{Области применения пнневмоприводных устройств и их особенности}\label{sec:ch1/sec1}


Пневмоприводные устройства находят широкое применение в различных областях техники и промышленности; от автоматизации в
машиностроительной области до химической и пищевой. Одной из первых работ в этой области была работа А.П. Германа
<<Применение сжатого воздуха в горном деле>>, опубликованная в 1933 году \cite*{german:a}. Исторический аспект развития пневмопривода
исследован учеными, такими как И.И. Артоболевский, который сыграл важную роль в систематизации знаний по теории и
проектированию пневмосистем.

Расчет данного класса приводов сопряжен со значительными трудностями. Это обусловлено тем, что движение твердых тел,
таких как поршни, должно рассматриваться как следствие протекающих газо- и термодинамических процессов, характеризующих
перемещения частиц воздуха. При этом необходимо комплексно учитывать сжимаемость воздушной среды, а также неравномерность
перемещения механических элементов привода под воздействием изменяющихся нагрузок, сил трения, веса и иных факторов.

Ввиду высокой сложности данной предметной области, при разработке теоретических основ пневматических приводов
исследователям приходилось вводить значительное количество упрощающих допущений, что в ряде случаев приводило к
расхождениям между расчетными и экспериментальными данными.

Наиболее ранние
работы были сосредоточены на разработке методов для приводов с постоянным противодавлением, равным атмосферному.
В частности, Н.А. Бухарин \cite*{бухарин} предложил метод определения времени срабатывания ПП, а Н.М. Маркевич \cite*{маркевич}
разработал альтернативную методику динамического расчета. Особо выделяется исследование, где с применением электронно-вычислительной
техники были получены обобщенные графики и номограммы, позволяющие упростить практические расчеты пневматических приводов различных
конструкций.

Обобщение результатов многочисленных теоретических и экспериментальных изысканий, а также сопоставление их с
данными натурных испытаний, позволило авторам разработать рациональные инженерные методики расчета пневматических
приводов и определить оптимальные области их эффективного применения в зависимости от конструктивных параметров.
В основу разработанных методик были положены фундаментальные труды ученых в области термодинамики, теории механизмов
и машин, газовой динамики, в том числе работы В. Шюле \cite*{шюле:техническая_термодинамика}, А.М. Литвина \cite*{литвин:техническая_термодинамика},
И.И. Артоболевского \cite*{артболевский:теория_механизмов_машин}, К.И. Страховича \cite*{страхович:прикладная_газодинамика},
И.П. Гинзбурга \cite*{гинзбург:прикладная_газодинамика}, М.А. Мамонтова \cite*{мамонтов:вопросы_термодинамики,мамонтов:некоторые_случаи_течения_газа}
и других.

Большая область применения ПП связана с особенностями рабочей среды --- воздуха. В отличие
от жидкости, которую при некоторых допущениях можно считать несжимаемой, воздух наоборот, имеет достаточно
малую плотность порядка $\rho_{\text{воздух}} = \text{1,1839}$ \si{\kilogram\per\cubic\metre} при нормальных условиях
и достаточно высокую сжимаемость, которой нельзя пренебречь при научных или инженерных расчетах. Также, из-за
отсутствия смазочных свойств воздуха и наличия определенного содержания водяного пара, который может конденсироваться
в рабочих камерах пневмодвигателей при интенсивных термодинамических процессах, возникает необходимость в предварительной
обработке воздуха, чтобы обеспечить надлежащую работоспособность и увеличить срок службы приводных устройств.

Рассмотрим основные преимущества и недостатки пневмопривода в сравнение с остальными типами приводов.

\textbf{Простота конструкции, экологическая безопасность и легкость в монтаже}. Одно из главных преимуществ
ПП заключается
в их простоте и относительной дешевизне конструкции. Элементы, из которых состоит ПП --- устройства
подготовки воздуха, пневмодвигатели, распределители и т. д. достаточно просты в изготовлении и не требуют использования редких или дорогостоящих
материалов, что снижает стоимость производства. Дополнительно, системы на основе ПП легко масштабируются,
что позволяет использовать их в различных отраслях промышленности.


Рабочая среда ПП является более безопасным по сравнению
с гидравлическими системами, где используются специальные, возможно воспламеняющиеся
или токсичные, жидкости. Именно поэтому ПП находят широкое применение в
промышленности, включая пищевую, химическую и нефтегазовую отрасли.


Воздух, использующийся в ПП, легко доступен, а также очень легко и
быстро переходит из сжатого состояния в исходное (и наоборот), что позволяет практически моментально
реагировать на изменения в системе. Управление ПП может быть весьма простым или,
наоборот, сложным и автоматизированным, что делает их гибким решением для большинства промышленных задач.


Кроме того, благодаря простой конструкции, предусмотрительному размещению деталей и
отсутствию особых требований к условиям эксплуатации, монтаж ПП обычно не
требует больших затрат по сравнению с другими видами приводов, что делает их еще более
привлекательным вариантом для предприятий различных отраслей.


\textbf{Надежность работы пневмопривода при тяжелых условиях окружающей среды}. ПП, в меру своих
особенностей, способностью работать в сложных и грубых условиях. Они способны выдерживать и продолжать
работать в широком диапазоне температур, включая крайне высокие и крайне низкие. Это особенно важно
для применения в отраслях, где температура может колебаться или когда оборудование подвержено
значительным температурным перепадам.

Кроме того, ПП являются относительно невосприимчивы к воздействию пыли и влаги. Воздух довольно просто
очистить от примесей и осушить (пыль, влага), предотвращая её засорение и обеспечивая надежную работу.
Естественная устойчивость к влажности делает ПП подходящими для использования в условиях высокой влажности.

Эти свойства делают ПП отличными для использования в самых разных промышленных секторах, начиная
от пищевой промышленности, где температура и влажность могут быть довольно высокими, до строительства и
горнодобывающей промышленности, где оборудование может быть подвержено значительному загрязнению и пыли.
Мало того, что они способны работать в этих условиях, они также продолжают работать надежно и эффективно.

\textbf{Пожаро- и взрывобезопасность пневмопривода}.
ПП обладают высокой степенью пожаро- и взрывобезопасности
благодаря отсутствию возможности искрообразования во время их работы.
В отличие от электроприводов, где электрические искры представляют потенциальную опасность.
Это делает ПП надежным и безопасным решением для применения в условиях, подразумевающих
высокие требования по пожаро- и взрывобезопасности, таких как нефтегазовая промышленность и химическое
производство. Эта характеристика делает ПП особо привлекательными в контексте строгого соблюдения
стандартов безопасности в промышленности.

\textbf{Срок службы пневмопривода}.
ПП обладают значительно более длительным сроком службы по сравнению с гидро- и электроприводами.
Это обусловлено несколькими ключевыми факторами.

Во-первых, ПП обычно имеют более простую конструкцию,
чем гидро- и электроприводы, что снижает вероятность возникновения неисправностей и износа. Отсутствие сложных
механизмов также означает, что ПП требуют меньше обслуживания и ремонта.

Во-вторых, ПП работают на сжатом воздухе, что исключает ряд проблем, с которыми часто сталкиваются
гидро- и электроприводы. Например, в гидроприводе довольно часто возникают утечки масла,
коррозия и износ подвижных механизмов, что может привести к сокращению срока службы и необходимости в ремонте.
ПП, в свою очередь, избавлены от этих проблем, что способствует увеличению их долговечности.

\textbf{Возможность развивать большую скорость выходного звена}.
ПП обладают способностью развивать высокие скорости на выходном звене,
так как реализация больших скоростей в гидро- и электроприводах ограничивается их
большей инерционностью (масса жидкости и инерция роторов) и отсутствием демпфирующего
эффекта, которым обладает воздух. способность пневмопривода достигать высоких скоростей на выходном
звене до 10 \si{\metre\per\second} для пневмоцилиндра и до \num{100000}~$\text{мин}^{-1}$~для пневмомоторов
делает их привлекательным решением для использования во многих сферах промышленности.
Это достоинство в полной мере
реализуется в приводах циклического действия, где высокая скорость операций
играет важную роль. Кроме того, ПП идеально подходят для использования в
высокопроизводительных оборудованиях, таких как манипуляторы, сварочные машины,
плоскошлифовальные станки и другие виды оборудования, где необходимы быстрые и достаточно точные движения.

\textbf{Невосприимчивость пневмопривода к радиационным и электромагнитным возмущениям}.
Пневмоприводы обладают преимуществом нечувствительности к радиационному и электромагнитному излучению
благодаря отсутствию использования электрических компонентов \cite{kasimov:a, kasimov:b}, высокой
рабочей надежности и устойчивости к
воздействию внешних электрических полей. Это делает пневмоприводы предпочтительным выбором в условиях,
где требуется обеспечение стабильной работы оборудования в окружающей среде с высоким уровнем радиационного или
электромагнитного излучения, таких как в ядерной и космической промышленности, а также в устройствах,
работающих вблизи силовых и электрических установок.

\textbf{Возможность передачи пневмо энергии на относительно большие расстояния}.
ПП обладают преимуществом в передаче пневмоэнергии на относительно большие расстояния по
пневмолиниям. В сравнении с электроприводом,
ПП уступает в этом вопросе, однако превосходит гидропривод за счет меньших
потерь по длине в протяженных магистральных линиях. Для электрической энергии характерно передача по линиям
электропередач на значительные расстояния без существенных потерь, в то время как передача пневмоэнергии
экономически целесообразна до нескольких десятков километров, что
обычно реализуется в пневмосистемах промышленных
предприятий с централизованным питанием от компрессорной станции.


Однако, несмотря на все свои преимущества ПП обладает рядом недостатков.

\textbf{Высокий уровень шума}.
ПП при работе создает достаточно большое шумовое давление, прежде всего связанное с
истечением сжатого воздуха из отверстий распределителей, дросселированием. Уровень шума может
достигать порядка 130~$\div$~145~дБ \cite{pr11092544}. Наиболее шумными являются
компрессорные установки, распределители, в которых одна из линий связана с атмосфеорой и
пневмодвигатели.

\textbf{Высокая сжимаемость и малая напряженность рабочей среды}.
Напряженность рабочей среды в ПП аналогична электроприводу и составляет 0.4~$\div$~1~\si{\mega\pascal} \cite{tugengold:a},
что примерно на 2 порядка меньше чем у гидропривода (3~$\div$~100~\si{\mega\pascal} \cite{tugengold:a}).
Такое низкое значение напряженности рабочей среды делает ПП достаточно инерционным.
При большой инерционной нагрузке выходное звено пневмопривода, начинает совершать колебательный процесс, а при
достаточной силовой нагрузке, выходное звено начинает <<продавливаться>>, что естественно снижает точность
позиционирования или точность слежения за входным сигналом. От сюда следует тот факт, что у пневмопривода возникает
трудность в поддерживании постоянной скорости движения и позиционирования в промежуточных положениях.

\textbf{Большая стоимость пневмоэнергии}.
ПП обладает достаточно низким КПД, что увеличивает стоимость технологического процесса при
той же полезной мощности, по сравнению с электро- и гидроприводом. КПД пневмопривода может составлять
порядка 10\%, а в некоторых случаях и до 1\%.
Поэтому применение пневмопривода в длительных процессах является нецелесообразным.

\textbf{Большие габаритные размеры}.
Из-за той же небольшой напряженности рабочей среды, средние габаритные размеры пневмопривода и электропривода
при аналогичной мощности в несколько раз больше чем у гидропривода.

На основе описанных недостатков и преимуществ пневмопривода, а также беря во внимание
характеристики различных типов приводов, можно составить сводную таблицу характеристик.
Ниже в таблице \cref*{tab:part1:actuators_comparision} представлены
сводные характеристики различных типов приводов.

\begingroup
\small
\centering
\begin{longtable}[htpb]{|p{3.5cm}||p{3.5cm}|p{3.5cm}|p{3.5cm}|}
    \caption{Сравнение различных типов приводов\label{tab:part1:actuators_comparision}}                                                                                                                                      \\
    \hline
    \multirow{2}{*}{Показатель} & \multicolumn{3}{c|}{Тип привода}                                                                                                                                                           \\ \cline{2-4}
                                & Пневмопривод                                                  & Электропривод                                                           & Гидропривод                                      \\ \hline
    \endfirsthead
    \caption*{Продолжение таблицы~\thetable}                                                                                                                                                                                 \\
    \hline
    \multirow{2}{*}{Показатель} & \multicolumn{3}{c|}{Тип привода}                                                                                                                                                           \\ \cline{2-4}
                                & Пневмопривод                                                  & Электропривод                                                           & Гидропривод                                      \\ \hline
    \endhead
    \endfoot
    \endlastfoot
    Развиваемая нагрузка        & до 30 \si{\kilo\newton}                                       & до 1000 \si{\kilo\newton}                                               & до 3000 \si{\kilo\newton}                        \\
    \hline
    Предельная нагрузка
                                & допустима                                                     & ограничена силой тока                                                   & допустима                                        \\
    \hline
    Скорость передачи энергии
                                & ограничена скоростью распостранения продольной волны в среде.
    (При нормальных условиях теоретический максимум
    \num{331,45} \si{\metre\per\square\second})
                                & ограничена скоростью распостранения электромагнитной волны.
    (При нормальных условиях теоретический максимум
    \num{299792458} \si{\metre\per\square\second})
                                & ограничена скоростью распостранения продольной волны в среде.
    (При нормальных условиях
    \num{1390} \si{\metre\per\square\second}
    для машинного масла)                                                                                                                                                                                                     \\
    \hline
    Передача энергии на расстоянии
                                & до 1000 \si{\metre}                                           & до 100 \si{\metre} (при большой силе тока)                              & 100 \si{\metre}                                  \\
    \hline
    Накопление энергии
                                & легко осуществимо                                             & затруднительно                                                          & затруднительно                                   \\
    \hline
    Линейное движение
                                & легко осуществимо                                             & затруднительно                                                          & легко осуществимо                                \\

    \hline
    Вращательное движение
                                & легко осуществимо                                             & легко осуществимо                                                       & легко осуществимо                                \\
    \hline
    Максимальная линейная скорость испольнительного механизма
                                & 10 \si{\square\metre}
                                & зависит от устройства и конкретных условий
                                & 0,5 \si{\square\metre}                                                                                                                                                                     \\
    \hline
    Экологичность               & безвреден из-за использования атмосферного воздуха            & утечки масла, которые являются токсичными отходами для окружающей среды & создает зачастую маломощный электромагнитный фон \\
    \hline
    Точность позиционирования   & 1 \si{\micro\metre}                                           & 1 \si{\micro\metre}                                                     & 100 \si{\micro\metre}                            \\
    \hline
\end{longtable}
\normalsize
\endgroup


В основном при выборе типа привода оказывают влияние следующие факторы: закон движения выходного звена, качество выполнения
технологических операций, стоимость всего оборудования, энергоэффективность, эксплуатационные параметры, экологические характеристики,
компоновка и технологические возможности элементов оборудования, система управления, доступность и однотипность источников
энергии, пожарная и взрывоопасность и т. д., и благодаря своим преимуществам, ПП особенно широкое применение нашли в цикловой автоматике.
Где требуются высокие скорости движения и достаточно точное позиционирование.


Однако,
во многих случаях целесообразно применять комбинированные системы приводов, такие как
электрогидравлические, электропневматические \cite{рабинович1973системы,трофимов:a}. Их особенность
состоит в том, что они использую преимущества обоих исполнений. Появляется возможность точного позиционирования в промежуточных
положениях. Точное слежение за определенной траекторией движения, которая достаточно простым способом можно
задать в микроконтроллере. Таким образом появляется грамотно разделение обязанностей, где за непосредственное движение объекта регулирования
отвечает ПП, а обработка информации приходится электронно-вычислительными средствами.
