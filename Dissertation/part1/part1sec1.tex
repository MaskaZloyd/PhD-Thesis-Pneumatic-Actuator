
\section{Области применения пневмоприводных устройств и их особенности}\label{sec:ch1/sec1}

Пневмоприводные устройства широко применяются в различных отраслях промышленности, включая машиностроение,
химическое и пищевое производство. Основополагающие исследования в данной области были заложены в работах А.П.
Германа \cite*{german:a} и И.И. Артоболевского, систематизировавшего теоретические основы проектирования пневмосистем.

Моделирование и расчет пневмоприводов представляет значительную сложность вследствие необходимости комплексного учета
газодинамических процессов, сжимаемости рабочей среды и нелинейной динамики механических элементов. При разработке
теоретических основ исследователи вводили ряд упрощающих допущений, что обуславливало расхождения между расчетными
и экспериментальными данными.

Значительный вклад в развитие методов расчета внесли Н.А. Бухарин \cite*{бухарин} и Н.М. Маркевич \cite*{маркевич},
разработавшие методики определения динамических характеристик пневмоприводов. Теоретический фундамент составили
труды В. Шюле \cite*{шюле:техническая_термодинамика}, А.М. Литвина \cite*{литвин:техническая_термодинамика}, И.И. Артоболевского
\cite*{артболевский:теория_механизмов_машин} и других ученых в области термодинамики и газовой динамики.

Специфика функционирования пневмопривода определяется фундаментальными свойствами рабочей среды -- воздуха.
В отличие от других типов приводов, пневматические системы характеризуются работой со средой малой плотности
($\rho_{\text{воздух}} = \text{1,1839}$ \si{\kilogram\per\cubic\metre} при нормальных условиях)
и высокой сжимаемости. При этом особенности воздуха, как рабочего тела, включая отсутствие смазывающих свойств
и наличие водяного пара, требуют организации предварительной подготовки для обеспечения надежного функционирования привода.

Данные физические характеристики рабочей среды в сочетании с конструктивными особенностями пневматических
систем задают специфический набор преимуществ и ограничений, которые необходимо учитывать при проектировании
автоматизированных систем управления.
Детальный анализ этих характеристик позволяет установить рациональные области применения пневмопривода
и оптимальные способы его интеграции в технологические процессы.

Ключевым достоинством пневматических приводов является простота их конструкции и низкая
стоимость изготовления, что обеспечивается технологичностью производства основных компонентов
и отсутствием потребности в дефицитных материалах \cite{ritter2009nonlinear,dong2003prospects}. В отличие от гидравлических систем, использующих потенциально
опасные рабочие жидкости, пневмоприводы характеризуются повышенной безопасностью благодаря применению воздуха в качестве рабочей среды, что особенно важно в
пищевой и химической промышленности \cite{raisoni2017utilization,kreinin2020optimization}.

Надежность функционирования пневмоприводов в сложных условиях эксплуатации обеспечивается их способностью работать
в широком диапазоне температур и при высокой запыленности благодаря возможности очистки и осушки рабочей среды
\cite{perez2021reliable,lin2020selfsensing,robertson2017soft}. Важным преимуществом является отсутствие искрообразования,
что делает пневмоприводы незаменимыми для применения во взрывоопасных средах \cite{dong2003prospects,khitrovo2022jet},
особенно в нефтегазовой и химической промышленности \cite{li2008piezoelectric,zhao2017design}.

Существенным техническим преимуществом является возможность достижения высоких скоростей движения выходного
звена -- до 10 \si{\metre\per\second} для пневмоцилиндров и до \num{100000}~$\text{об.}\cdot\text{мин}^{-1}$ для пневмомоторов
\cite{tanaka2013comparative,wakana2013flexible}. Это обусловлено низкой инерционностью рабочей среды и наличием естественного
демпфирования \cite{volder2010review,yaoxing2013servo}, что позволяет реализовывать высокоскоростные режимы работы при сохранении
устойчивости системы. Пневмоприводы демонстрируют особую эффективность в системах циклического действия, где требуется сочетание
высоких скоростей с точным позиционированием \cite{mosadegh2014soft}.

Дополнительными преимуществами являются невосприимчивость к радиационным и электромагнитным
воздействиям \cite{kasimov:a,kasimov:b}, что расширяет области применения в специальных условиях,
а также возможность передачи пневмоэнергии на расстояния до нескольких десятков километров \cite{luo2011energy,wang2019isobaric}.
Это позволяет создавать централизованные системы питания промышленных предприятий с высокой экономической эффективностью \cite{bodh2016pneumatic,hassan2015booster}.

Наряду с преимуществами, пневматические приводы характеризуются и рядом существенных недостатков.
Одним из основных является высокий уровень шума при работе, достигающий 130~--~145~дБ \cite{pr11092544},
что обусловлено процессами истечения сжатого воздуха из распределителей и дросселированием.
Особенно значительное шумовое воздействие создают компрессорные установки и распределители с атмосферным выхлопом.

Физические свойства воздуха, как рабочей среды, характеризующейся высокой
сжимаемостью и низким рабочим давлением, обуславливают ряд технических ограничений \cite{tugengold:a}.
При значительных инерционных нагрузках выходное звено пневмопривода подвержено колебательным процессам, а
при высоких силовых нагрузках наблюдается эффект <<продавливания>>, что существенно снижает точность позиционирования
и качество слежения за входным сигналом. Эти особенности создают сложности в обеспечении равномерного
движения и точного позиционирования в промежуточных положениях.

Существенным недостатком являются увеличенные габаритные размеры пневмопривода, что обусловлено
использованием низкого рабочего давления воздуха (\num{0.4} -- \num{1} МПа) по
сравнению с гидравлическими системами (\num{3} -- \num{100} МПа) \cite{tugengold:a}.
Для обеспечения требуемых силовых характеристик требуется увеличение активных площадей исполнительных
механизмов, что приводит к существенному росту габаритов. При одинаковой развиваемой мощности
размеры пневмопривода в 3--4 раза превышают аналогичные показатели гидравлических
систем. Это создает значительные ограничения при проектировании компактных систем,
особенно в области робототехники и многозвенных манипуляторов, где массогабаритные характеристики приводов существенно влияют на динамические свойства всей системы.

При выборе типа привода для конкретной технологической системы учитывается комплекс определяющих факторов.
К ним относятся требуемые законы движения выходного звена, показатели качества выполнения технологических операций, экономические
аспекты внедрения и эксплуатации оборудования, энергетическая эффективность, надежность и долговечность. Существенное значение имеют
также экологические характеристики, особенности компоновки и технологические возможности элементов оборудования, сложность реализации
систем управления, доступность и унификация источников энергии, требования по пожарной и взрывобезопасности
\cite{zisser2013position,takemura2000hybrid,vladur2016dynamic}.


Пневматические приводы часто используются в системах цикловой
автоматики благодаря их характерному преимуществу -- сочетанию высокого быстродействия
с сохранением точности позиционирования. В данной области
пневмоприводы эффективно реализуют требования к скоростным
характеристикам движения и точности останова в заданных положениях
\cite{hildebrandt2010optimal,carducci2006viscous,salim2015control}.

В современных технических системах широкое распространение получили комбинированные приводные
устройства, в частности, электрогидравлические и электропневматические приводы \cite{рабинович1973системы,трофимов:a}.
Данные решения позволяют эффективно использовать преимущества различных технологий при одновременной компенсации их недостатков.

Интеграция электронных систем управления с пневматическими исполнительными
механизмами обеспечивает качественно новые функциональные возможности. К их числу
относятся прецизионное позиционирование в произвольных промежуточных положениях,
реализация сложных законов движения с использованием цифровых систем управления, а
также адаптивная корректировка параметров регулирования в режиме реального времени\cite{salim2011control,situm2013control}.

Современный подход к проектированию пневматических систем требует их рассмотрения как единого
мехатронного комплекса, где пневматические и электронные компоненты образуют тесно интегрированную
структуру. В такой системе процессы преобразования энергии, формирования
механического движения и реализации алгоритмов управления представляют собой взаимосвязанные
аспекты единого функционального целого. Применение микропроцессорных систем в этом
контексте не просто расширяет возможности управления, но создает качественно новый уровень
взаимодействия между силовой и информационной составляющими привода, обеспечивая их
синергетическое функционирование\cite{safuan2010microcontroller, vladur2016dynamic, lababidi2023position}.

В современных производственных системах значительная часть технологических операций связана с необходимостью точного
перемещения и позиционирования рабочих органов. Особое место среди современных пневматических систем занимают позиционные
пневмоприводы, обеспечивающие точное позиционирование выходного звена с последующей его фиксацией в
заданном положении, что играет важную роль в автоматизации производственных процессов \cite{papoutsidakis2007control,qiu2024position}.
