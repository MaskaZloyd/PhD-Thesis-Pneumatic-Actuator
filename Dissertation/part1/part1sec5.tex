\section{Выводы по главе 1}\label{sec:ch1/conclusion}

Анализ текущего состояния исследований в области позиционных пневмоприводов с
дискретными распределителями позволил выявить ряд существенных особенностей и проблем, требующих решения:

\begin{enumerate}
	\item Установлено, что позиционные пневмоприводы с дискретными
	      распределителями представляют перспективное направление развития
	      пневмоавтоматики благодаря их низкой стоимости, высокой надежности
	      и простоте конструкции. При этом их эффективное функционирование
	      осложняется существенной нелинейностью динамических процессов,
	      обусловленной как физическими свойствами сжатого воздуха, так и
	      дискретным характером управляющих воздействий.

	\item Выявлено наличие трех противоречивых требований к характеристикам таких систем:
	      \begin{itemize}
		      \item обеспечение высокой точности позиционирования;
		      \item достижение требуемого быстродействия;
		      \item минимизация количества переключений распределителей.
	      \end{itemize}

	\item Обнаружены ключевые противоречия в оптимизации систем:
	      \begin{itemize}
		      \item повышение точности требует более частого переключения
		            распределителей, что приводит к ускоренному износу распределительной аппаратуры;
		      \item увеличение быстродействия негативно сказывается на
		            точности из-за возрастания динамических ошибок;
		      \item стремление к минимизации числа переключений
		            ограничивает возможности как точной коррекции положения, так и обеспечения требуемого быстродействия.
	      \end{itemize}

	\item Проведенный анализ существующих исследований показал
	      недостаточную проработку вопросов поиска компромисса между
	      частотой переключений распределителей и качеством позиционирования.
	      Существующие публикации, как правило, фокусируются на отдельных показателях качества без учета их взаимного влияния.

	\item Установлено, что оптимальным решением является структура с
	      четырьмя дискретными распределителями, обеспечивающая необходимую
	      функциональность системы позиционирования при сохранении простоты
	      технической реализации.

	\item Показано, что для эффективного решения задачи позиционирования
	      необходим интегративный подход к проектированию, учитывающий
	      взаимосвязанность процессов в пневматической, механической и
	      управляющей подсистемах.

	\item На основе проведенного анализа сформулирована необходимость
	      применения методов многокритериальной оптимизации, в частности,
	      построения фронта Парето, что позволит выявить предельно достижимые
	      характеристики системы при различных соотношениях критериев.
\end{enumerate}

Таким образом, определены основные направления дальнейших исследований:
\begin{itemize}
	\item разработка уточненной математической модели пневмопривода,
	      учитывающей взаимосвязь термодинамических и механических процессов;
	\item синтез алгоритмов управления, обеспечивающих компромисс между
	      точностью позиционирования, быстродействием и интенсивностью переключений распределителей;
	\item построение и анализ фронта Парето для оптимизации параметров системы.
\end{itemize}