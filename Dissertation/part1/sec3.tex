\section{Анализ задачи разгона и повышения максимальной скорости позиционного пневмопривода}\label{sec:ch1/sec3}

Как было сказано ранее, основным исследователем в области ПП устройств является Е.В. Герц. В работах
\cite*{герц:скоростной_привод,герц:групповой_привод}
проанализирована проблема повышения быстродействия и производительности пневматических приводов
в различных отраслях машиностроения. Рассмотрены вопросы динамики, соствлены математические модели, на основе которых
возможно произвести расчет высокоскоростных ПП. Отмечено, что типовые ПП, выпускаемые промышленностью,
обладают недостаточной скоростью рабочих органов --- не более 0,5 $\div$ 1 \si{\metre\per\square\second}. Это значительно ниже
требований ряда ответственных технологических процессов, где необходима скорость 5  $\div$ 20 \si{\metre\per\square\second},
например, при клеймении, маркировке, в ударных системах.

Для решения данной проблемы автор предлагает использование более совершенных высокоскоростных
ПП. Основной принцип повышения скорости рабочих органов в таких приводах
заключается в обеспечении значительного перепада давлений на рабочем поршне. Этого удается достичь
за счет введения в конструкцию дополнительной встроенной полости-резервуара, из которого сжатый воздух
поступает непосредственно в рабочую полость привода. При этом давление в рабочей полости
можно поддерживать близким к магистральному (0,5~$\div$~0,7 \si{\mega\pascal}), а в выхлопной --- стремиться к атмосферному,
создавая, таким образом, необходимый большой перепад. Кроме того, увеличение площади поршня также
способствует росту ускорения, но этот путь ограничен требованиями к компактности и массе привода.

Проведенные исследования и практический опыт эксплуатации, по мнению автора, показали, что
высокоскоростные ПП со встроенным резервуаром имеют ряд существенных преимуществ
по сравнению с традиционными аналогами. Они позволяют значительно увеличить рабочее усилие и
скорость поршня, повысить энергетическую эффективность механизма за счет более рационального
расширения воздуха, обеспечивают возможность точной регулировки выпуска воздуха из резервуара,
а также увеличивают ударную мощность рабочих органов. Такие приводы продемонстрировали скорости
поршня до 5,3 \si{\metre\per\square\second} на коротких ходах. Вместе с тем, их конструкция несколько сложнее типовых,
и они требуют больших затрат на подготовку и расход сжатого воздуха.

Автор отмечает, что использование высокоскоростных
ПП со встроенным резервуаром является эффективным решением
для повышения быстродействия и производительности ряда ответственных
технологических процессов в современном машиностроении.

В работе \cite*{божкова:повышение_быстродействия} рассматривался выбор оптимальных параметров управления пневматическим приводом промышленного робота.
Целью было обеспечение высокой скорости работы при сохранении требуемой точности позиционирования.

Авторы Л.В. Божкова и О.А. Дащенко предложили повысить быстродействие путем одновременной работы
динамически независимых узлов привода. Это позволяет избежать последовательной активации пневматических двигателей,
что является типичной практикой для современных промышленных роботов.

Для реализации программного управления без обратной связи, авторы дают ряд рекомендаций по наладке системы.
Это позволяет сократить время переналадки, обеспечить плавность и безударность движения.

Таким образом, предложенный подход направлен на повышение производительности промышленных
роботов с пневматическим приводом без ущерба для точности позиционирования.


В патенте \cite*{патент:рязанов} предложена конструкция позиционного пневмопривода,
в которой управляющая полость исполнительного пневмоцилиндра сообщена с регулируемым
дозатором в виде плавающего поршня с дросселирующим каналом. Запорный элемент входной
камеры дозатора жестко связан с плавающим поршнем, а выходная камера дозатора постоянно
сообщена с управляющей полостью пневмоцилиндра.

Повышение производительности пневмопривода достигается благодаря созданию управляемого
перепада давления на быстроперемещающийся плавающий поршень дозатора из-за дросселирования
потока через канал, а также за счет упрощения конструкции путем жесткой связи запорного элемента
с поршнем и обеспечения быстрой прямой подачи рабочей среды из выходной камеры дозатора в
управляющую полость цилиндра. Это ускоряет процессы подачи и перекрытия рабочей среды, повышая
быстродействие и производительность пневмопривода.

В следующем патенте \cite*{патент:Кистиченко} подошли к вопросу повышения быстродействия с точки зрения систем управления.
Автором
предложена конструкция позиционного пневмопривода, которая обеспечивает повышение
его быстродействия.

Ключевым элементом изобретения является использование специальной схемы управления
пневмораспределителями и пневмоклапанами. Она позволяет осуществлять быстрый выхлоп
рабочей среды из полостей пневмоцилиндра при реверсе движения поршня.

В типичных пневмоприводах при реверсировании возникает необходимость создания разности
давлений в рабочих полостях цилиндра для изменения направления движения поршня.
Это приводит к появлению выстоя поршня в точке остановки, что снижает быстродействие системы.

Предлагаемое техническое решение устраняет данный недостаток за счет управляемой
отсечки полостей цилиндра от магистрали слива. Когда поршень подходит к точке реверса,
полость, из которой будет происходить выхлоп, отсекается от магистрали слива на таком уровне
давления, который обеспечивает быстрое преодоление сил трения и ускорение поршня в обратном
направлении. Это позволяет практически исключить выстой поршня и сократить длительность переходных
процессов позиционирования.

Управление пневмораспределителями и клапанами осуществляется программным устройством
на основе информации о текущем положении поршня и давлениях в полостях. Таким образом,
система автоматически подстраивается под динамические характеристики привода, обеспечивая
оптимальный режим работы.

В результате применения данной конструкции пневмопривода достигается значительное,
до 3~$\div$~5 раз, повышение его быстродействия по сравнению с традиционными решениями.
Это позволяет повысить производительность автоматизированного оборудования, использующего
такие ПП.

Развитие ПП шло по пути повышения ускорения и максимальной скорости движения рабочих органов.
Это достигалось за счет использования более совершенных конструкций, обеспечивающих значительный перепад
давлений на поршне, а также совершенствования систем управления для оптимизации процессов реверсирования
движения поршня. Эти исследования позволяли существенно увеличить скорость и ускорение рабочих органов по
сравнению с типовыми решениями.

Параллельно с этим, развивались и исследования, направленные на точное позиционирование пневмопривода в
момент останова. Данный вопрос напрямую связан с эффективным торможением рабочих органов. Таким образом,
исследователи ПП решали задачу не только повышения динамических характеристик,
но и обеспечения точного позиционирования в конечных точках.

Однако на данный момент, несмотря на достигнутые успехи в обеих областях, исследования
практически не продолжаются. Для промышленности необходимы решения, в которых баланс между
скоростью и точностью позиционирования является определяющим фактором.
