
\section{Адаптация математической модели к эффективному численному расчету на ЭВМ}\label{sec:ch2/sec4}

Эффективное численное моделирование динамики электропневматического привода требует
оптимизации математической модели для выполнения расчетов на ЭВМ.
Данный этап критически важен для обеспечения высокой производительности
вычислений при проведении многочисленных итераций в задачах оптимизации,
анализа чувствительности и робастности системы управления.

Основные цели оптимизации математической модели включают:

\begin{enumerate}
    \item Снижение вычислительной сложности уравнений;
    \item Уменьшение времени выполнения расчетов;
    \item Повышение численной устойчивости алгоритмов;
    \item Эффективное использование параллельных вычислительных архитектур.
\end{enumerate}

Для достижения этих целей будут применены следующие методы:

\begin{enumerate}
    \item Векторизация уравнений;
    \item Упрощение и линеаризация нелинейных выражений;
    \item Предварительное вычисление констант и коэффициентов;
\end{enumerate}
\subsection{Векторизация уравнений}\label{sec:ch2/sec4/subsec1}
Векторизация уравнений представляет собой эффективный метод оптимизации вычислений,
позволяющий использовать преимущества SIMD-инструкций (Single Instruction, Multiple Data)
современных процессоров. Данный подход особенно актуален для математической
модели электропневматического привода с дискретными распределителями, где многие операции могут быть выполнены
параллельно \cite*{eichenberger2004simd} над несколькими элементами данных.

Основная идея векторизации заключается в преобразовании скалярных операций в
векторные, что позволяет обрабатывать несколько элементов данных
одновременно \cite*{nuzman2011vaporsimd}. В контексте рассматриваемой модели это означает переход от
поэлементных вычислений к операциям над векторами и матрицами.
Процесс векторизации начинается с представления состояния системы
в виде единого вектора. Для электропневматического привода
вектор состояния может быть записан как:

\begin{equation}
\label{eq:ch2/state_vector}
    \mathbf{y} = [x, v, p_1, p_2, T_1, T_2, u_1, u_2, u_3, u_4]^T.
\end{equation}

Векторизованная форма уравнения движения поршня \eqref{eq:ch2/eq8} может быть представлена следующим образом:

\begin{equation}
\label{eq:ch2/vec_motion}
    M\ddot{x} = \mathbf{F}^T\mathbf{p} - p_\text{атм}(\mathbf{F}^T\mathbf{1}) - R_\text{тр}(\dot{x}) - R_\text{оп}(x, \dot{x}),
\end{equation}
где $\mathbf{F} = [F_1, -F_2]^T$ -- вектор эффективных площадей поршня;
$\mathbf{p} = [p_1, p_2]^T$ -- вектор давлений в полостях цилиндра;
$\mathbf{1} = [1, 1]^T$ - единичный вектор.

Уравнения изменения давлений в полостях \eqref{eq:ch2/final_pressure_system} могут быть векторизованы следующим образом:
\begin{equation}
\label{eq:ch2/vec_pressure}
    \dot{\mathbf{p}} = \frac{\gamma}{\mathbf{V}(\mathbf{x})} \odot (R\mathbf{T} \odot \mathbf{G} - \mathbf{p} \odot (\mathbf{F}\dot{x})),
\end{equation}
где $\odot$ обозначает поэлементное умножение векторов;
$\mathbf{T} = [T_1, T_2]^T$ -- вектор температур в полостях;
$\mathbf{G} = [G_1, G_2]^T$ -- вектор суммарных массовых расходов для каждой полости;
$\mathbf{V}(\mathbf{x}) = [V_1(x), V_2(x)]^T$ -- вектор-функция объемов полостей.

Уравнения изменения температур \eqref{eq:ch2/energy_balance_final} также могут быть представлены в векторной форме:
\begin{equation}
\label{eq:ch2/vec_temperature}
    \dot{\mathbf{T}} = \frac{\gamma-1}{\mathbf{m}C_v} \odot (R\mathbf{T} \odot \mathbf{G} \pm \mathbf{p} \odot (\mathbf{F}\dot{x})),
\end{equation}
где $\mathbf{m} = [m_1, m_2]^T$ -- вектор масс воздуха в полостях.

Динамика переключения распределителей \eqref{eq:ch2/switching_dynamics} может быть векторизована для всех четырех распределителей одновременно:
\begin{equation}
\label{eq:ch2/vec_switching}
    \tau \dot{\mathbf{u}} + \mathbf{u} = \mathbf{u}_\text{зад},
\end{equation}
где $\mathbf{u} = [u_1, u_2, u_3, u_4]^T$ -- вектор текущих положений запорно-регулирующих элементов распределителей;
$\mathbf{u}_\text{зад}$ -- вектор заданных положений.

Для расчета массовых расходов через распределители можно использовать векторизованную форму уравнения массового расхода:
\begin{equation}
\label{eq:ch2/vec_mass_flow}
    \mathbf{G} = \psi(\mathbf{p}_\text{вх}, \mathbf{p}_\text{вых}) \odot \mathbf{C}_d \odot \mathbf{F}_\text{пр} \odot \mathbf{u} \odot \frac{\mathbf{p}_\text{вх}}{\sqrt{R\mathbf{T}_\text{вх}}},
\end{equation}
где $\psi(\mathbf{p}_\text{вх}, \mathbf{p}_\text{вых})$ -- векторизованная расходная функция;
$\mathbf{C}_d$ и $\mathbf{F}_\text{пр}$ -- векторы коэффициентов расхода и эффективных площадей проходных сечений распределителей соответственно.

Применение векторизации позволяет эффективно использовать SIMD-инструкции процессора,
что приводит к значительному ускорению вычислений. В зависимости от архитектуры процессора и
специфики реализации, ускорение может достигать 2-4 раза \cite*{tian2013simd} по сравнению с исходной скалярной версией.

\subsection{Оптимизация нелинейных функций}\label{sec:ch2/sec4/subsec2}

Так же важным этапом оптимизации является упрощение и линеаризация нелинейных функций,
которые могут оказывать существенное влияние на вычислительную сложность модели и деаль ее
жесткой. В контексте электропневматического привода, к нелинейным функцям относятся экспоненциальная
функция в модели трения, функция sign, реакция упоров и условные операции.

Для оптимизации вычисления экспоненциальной функции в модели трения предлагается использовать аппроксимацию Паде.
Данный метод обеспечивает высокую точность аппроксимации при сравнительно низкой вычислительной
сложности. Аппроксимация Паде для экспоненциальной функции может быть представлена в виде:

\begin{equation}
    e^x \approx \frac{1 + \frac{x}{2} + \frac{x^2}{10} + \frac{x^3}{120}}{1 - \frac{x}{2} + \frac{x^2}{10} - \frac{x^3}{120}}.
\end{equation}

Данная аппроксимация обеспечивает высокую точность в диапазоне $x \in [-2,5; 2,5]$,
что достаточно для моделирования эффекта Штрибека в рассматриваемой системе.

Реакция упоров, обычно моделируемая с использованием условных операций, может быть
оптимизирована применением непрерывной аппроксимации:

\begin{equation}
    R_\text{оп} = k_\text{оп}(x - x_\text{мин})\cdot S(x - x_\text{мин}) + k_\text{оп}(x - x_\text{макс})\cdot S(x - x_\text{макс}),
\end{equation}
где $S(x)$ -- сглаживающая функция, например:

\begin{equation}
    S(x) = \frac{1}{1 + e^{-\alpha x}},
\end{equation}
где $\alpha$ -- параметр, определяющий крутизну перехода.

Для оптимизации условных операций в модели предлагается
использовать непрерывные аппроксимации. Например, функция
максимума может быть аппроксимирована как:

\begin{equation}
    \max(a, b) \approx \frac{a + b + \sqrt{(a - b)^2 + \varepsilon}}{2},
\end{equation}
где $\varepsilon$ -- малое положительное число, обеспечивающее гладкость функции.

Применяя предложенные оптимизации, уравнения реакции упоров и силы трения могут быть представлены в виде:

\begin{equation}
    \begin{alignedat}{2}
        g(v) & = \left[R_\text{к} + (R_\text{с} - R_\text{к})\frac{1 + \frac{z}{2} + \frac{z^2}{10} + \frac{z^3}{120}}{1 - \frac{z}{2} + \frac{z^2}{10} - \frac{z^3}{120}}\right] \\
        R_\text{оп} & = k_\text{оп}(x - x_\text{мин})\cdot \frac{1}{1 + e^{-\alpha(x - x_\text{мин})}} + k_\text{оп}(x - x_\text{макс})\cdot \frac{1}{1 + e^{-\alpha(x - x_\text{макс})}},
    \end{alignedat}
\end{equation}

\subsection{Аналитическое вычисление якобиана для численного решения ОДУ}\label{sec:ch2/sec5/subsec3}

В контексте оптимизации математической модели пневматического привода была реализована процедура аналитического
вычисления матрицы Якоби для повышения эффективности численного интегрирования. Данный подход позволяет
существенно снизить вычислительные затраты и повысить численную стабильность решения системы дифференциальных уравнений.

Якобиан системы определяется как матрица частных производных:

\begin{equation}
\label{eq:ch2/jacobian_matrix}
    J = \frac{\partial \mathbf{f}}{\partial \mathbf{y}} =
    \begin{pmatrix}
        \frac{\partial f_1}{\partial y_1}    & \frac{\partial f_1}{\partial y_2}    & \cdots & \frac{\partial f_1}{\partial y_{10}}    \\
        \frac{\partial f_2}{\partial y_1}    & \frac{\partial f_2}{\partial y_2}    & \cdots & \frac{\partial f_2}{\partial y_{10}}    \\
        \vdots                               & \vdots                               & \ddots & \vdots                                  \\
        \frac{\partial f_{10}}{\partial y_1} & \frac{\partial f_{10}}{\partial y_2} & \cdots & \frac{\partial f_{10}}{\partial y_{10}}
    \end{pmatrix}
\end{equation}

где $\mathbf{f}$ -- вектор-функция правых частей системы дифференциальных уравнений.


Рассмотрим вычисление элементов матрицы Якоби для каждого уравнения системы:

\paragraph{Уравнение движения поршня}
Для уравнения движения поршня $\dot{x} = v$ имеем:
\begin{equation}
\label{eq:ch2/jacobian_motion}
    \begin{aligned}
        \frac{\partial \dot{x}}{\partial x}   & = 0                          \\
        \frac{\partial \dot{x}}{\partial v}   & = 1                          \\
        \frac{\partial \dot{x}}{\partial y_i} & = 0, \quad i = 3, \ldots, 10
    \end{aligned}
\end{equation}

\paragraph{Уравнение ускорения поршня}
Для уравнения ускорения поршня:
\begin{equation}
\label{eq:ch2/jacobian_acceleration}
    \begin{aligned}
        \frac{\partial \dot{v}}{\partial x}   & = -\frac{1}{M}\left(\frac{\partial R_\text{тр}}{\partial x} + \frac{\partial R_\text{оп}}{\partial x}\right) \\
        \frac{\partial \dot{v}}{\partial v}   & = -\frac{1}{M}\left(\frac{\partial R_\text{тр}}{\partial v} + \frac{\partial R_\text{оп}}{\partial v}\right) \\
        \frac{\partial \dot{v}}{\partial p_1} & = \frac{F_1}{M}                                                                                              \\
        \frac{\partial \dot{v}}{\partial p_2} & = -\frac{F_2}{M}                                                                                             \\
        \frac{\partial \dot{v}}{\partial y_i} & = 0, \quad i = 5, \ldots, 10
    \end{aligned}
\end{equation}
где частные производные силы трения и реакции опоры вычисляются на основе их моделей
\eqref{eq:ch2/friction_force} и \eqref{eq:ch2/support_reaction} соответственно.

\paragraph{Уравнения изменения давлений}
Для уравнений изменения давлений в полостях цилиндра:
\begin{equation}
\label{eq:ch2/jacobian_pressure}
    \begin{aligned}
        \frac{\partial \dot{p_i}}{\partial x}   & = -\frac{\gamma p_i}{V_i^2}\frac{\partial V_i}{\partial x}(RT_iG_i - p_iF_iv) + \frac{\gamma}{V_i}\left(RT_i\frac{\partial G_i}{\partial x} - F_iv\frac{\partial p_i}{\partial x} - p_iF_i\frac{\partial v}{\partial x}\right) \\
        \frac{\partial \dot{p_i}}{\partial v}   & = \frac{\gamma}{V_i}\left(RT_i\frac{\partial G_i}{\partial v} - p_iF_i\right)                                                                                                                                                  \\
        \frac{\partial \dot{p_i}}{\partial p_i} & = \frac{\gamma}{V_i}\left(RT_i\frac{\partial G_i}{\partial p_i} - F_iv\right)                                                                                                                                                  \\
        \frac{\partial \dot{p_i}}{\partial T_i} & = \frac{\gamma}{V_i}\left(RG_i + RT_i\frac{\partial G_i}{\partial T_i}\right)                                                                                                                                                  \\
        \frac{\partial \dot{p_i}}{\partial u_j} & = \frac{\gamma RT_i}{V_i}\frac{\partial G_i}{\partial u_j}, \quad j = 1, \ldots, 4
    \end{aligned}
\end{equation}
где $i = 1, 2$ для левой и правой полостей соответственно.

\paragraph{Уравнения изменения температур}
Аналогично для уравнений изменения температур:
\begin{equation}
\label{eq:ch2/jacobian_temperature}
    \begin{aligned}
        \frac{\partial \dot{T_i}}{\partial x}   & = \frac{\gamma-1}{m_iC_v}\left(R\frac{\partial (T_iG_i)}{\partial x} \pm \frac{\partial (p_iF_iv)}{\partial x}\right) \\
        \frac{\partial \dot{T_i}}{\partial v}   & = \frac{\gamma-1}{m_iC_v}\left(R\frac{\partial (T_iG_i)}{\partial v} \pm p_iF_i\right)                                \\
        \frac{\partial \dot{T_i}}{\partial p_i} & = \frac{\gamma-1}{m_iC_v}\left(R\frac{\partial (T_iG_i)}{\partial p_i} \pm F_iv\right)                                \\
        \frac{\partial \dot{T_i}}{\partial T_i} & = \frac{\gamma-1}{m_iC_v}\left(RG_i + RT_i\frac{\partial G_i}{\partial T_i}\right)                                    \\
        \frac{\partial \dot{T_i}}{\partial u_j} & = \frac{(\gamma-1)RT_i}{m_iC_v}\frac{\partial G_i}{\partial u_j}, \quad j = 1, \ldots, 4
    \end{aligned}
\end{equation}

\paragraph{Уравнения динамики переключения распределителей}
Для уравнений динамики переключения распределителей:
\begin{equation}
\label{eq:ch2/jacobian_valves}
    \begin{aligned}
        \frac{\partial \dot{u_i}}{\partial u_i} & = -\frac{1}{\tau}     \\
        \frac{\partial \dot{u_i}}{\partial y_j} & = 0, \quad j \neq i+6
    \end{aligned}
\end{equation}
где $i = 1, \ldots, 4$ для каждого распределителя.

Для вычисления частных производных массового расхода $\partial G_i/\partial y_j$ необходимо учитывать нелинейную
зависимость расходной функции $\psi(p_1, p_2)$ от давлений.

Рассмотрим производные расходной функции по давлениям:
\begin{equation}
\label{eq:ch2/flow_function_derivatives}
    \begin{alignedat}{6}
        \frac{\partial \psi}{\partial p_1} & = \begin{cases}
                                                   -\frac{1}{p_1}\frac{\gamma+1}{\gamma-1}\left(\frac{p_2}{p_1}\right)^{\frac{2}{\gamma}}\left[1 - \left(\frac{p_2}{p_1}\right)^{\frac{\gamma-1}{\gamma}}\right]\frac{1}{\psi}, & \text{если } \frac{p_2}{p_1} > b_\text{кр}    \\
                                                   0,                                                                                                                                                                           & \text{если } \frac{p_2}{p_1} \leq b_\text{кр} \\
                                               \end{cases} \\
        \frac{\partial \psi}{\partial p_2} & = \begin{cases}
                                                   \frac{1}{p_2}\frac{\gamma+1}{\gamma-1}\left(\frac{p_2}{p_1}\right)^{\frac{2}{\gamma}-1}\left[1 - \left(\frac{p_2}{p_1}\right)\right]\frac{1}{\psi}, & \text{если } \frac{p_2}{p_1} > b_\text{кр}    \\
                                                   0,                                                                                                                                                  & \text{если } \frac{p_2}{p_1} \leq b_\text{кр}
                                               \end{cases}
    \end{alignedat}
\end{equation}

Теперь, используя эти производные, можно вычислить частные производные массового расхода по переменным состояния системы. Для примера рассмотрим производную массового расхода по давлению в первой полости цилиндра:
\begin{equation}
\label{eq:ch2/mass_flow_derivative}
    \frac{\partial G_1}{\partial p_1} = C_d F_\text{пр} u_1 \left(\frac{\partial \psi}{\partial p_1} \cdot \frac{p_1}{\sqrt{RT_1}} + \psi \cdot \frac{1}{\sqrt{RT_1}}\right)
\end{equation}

Аналогичным образом вычисляются производные по другим переменным состояния. Для производных по температуре и положениям запорно-регулирующих элементов распределителей имеем:
\begin{equation}
\label{eq:ch2/mass_flow_derivative_T}
    \begin{aligned}
        \frac{\partial G_1}{\partial T_1} & = -\frac{1}{2} C_d F_\text{пр} u_1 \psi \frac{p_1}{\sqrt{RT_1^3}} \\
        \frac{\partial G_1}{\partial u_1} & = C_d F_\text{пр} \psi \frac{p_1}{\sqrt{RT_1}}
    \end{aligned}
\end{equation}

Полная матрица Якоби формируется путем объединения всех вычисленных частных производных в
соответствии с их позициями в векторе состояния системы \eqref{eq:ch2/state_vector}.

Применение якобиана для в методах численных численного интегрирования, таких как
BDF (Backward Differentiation Formula) и Radau, позволяет уменьшить количество вычислений
и повысить численную стабильность. В частности, использование якобиана позволяет
уменьшить количество вычислений правых частей системы, что особенно важно для жестких систем.