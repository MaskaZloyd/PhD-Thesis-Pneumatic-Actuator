\chapter{АЛГОРИТМ МЕТОДИКИ СТРУКТУРНО-ПАРАМЕТРИЧЕСКОГО СИНТЕЗА}\label{app:methodology}


\section*{Введение}

Методика структурно-параметрического синтеза позиционного пневмопривода с дискретными распределителями
представляет собой комплексный подход к определению оптимальной структуры и параметров системы управления. Данная методика
позволяет найти рациональное решение, учитывающее конфликтующие критерии: точность позиционирования, качество переходных
процессов и интенсивность переключений распределителей.

На начальном этапе структурного синтеза формируются Парето-фронты для различных управляющих
структур, что позволяет выявить взаимосвязи между критериями оптимизации и определить предпочтительные области
применения каждой из структур. На основе полученных данных создается карта предпочтительных областей применения,
которая позволяет выбрать конкурсные структуры для дальнейшего параметрического синтеза в зависимости
от требований к системе.

Последующий параметрический синтез заключается в определении оптимальных значений параметров выбранной управляющей
структуры на основе заданных требований к критериям оптимизации. Это достигается путем решения обратной
задачи с использованием построенных Парето-фронтов.

Ниже представлено детальное описание алгоритма методики структурно-параметрического
синтеза позиционного пневмопривода с дискретными распределителями.


\section*{Алгоритм построения фронтов Парето}

Построение фронтов Парето для различных управляющих структур пневмопривода направлено на
выявление взаимосвязей между показателями оптимизации и формирование базы для структурного синтеза.

\textbf{Входные данные алгоритма:}
\begin{itemize}
    \item Управляющая структура пневмопривода $S$.
    \item Множество варьируемых параметров $\mathbf{P}$.
    \item Диапазоны варьирования параметров $[\mathbf{P}_{\min}, \mathbf{P}_{\max}]$.
    \item Математическая модель пневмопривода $M$.
    \item Критерии оптимизации $\mathbf{J} = \{AC, ITAE, SI\}$.
    \item Требования к точности суррогатной модели $\varepsilon_{surr}$.
\end{itemize}

\textbf{Выходные данные алгоритма:} фронт Парето $\mathcal{PF}_{S}$ для управляющей структуры $S$.

\textbf{Алгоритм построения фронтов Парето:}

\paragraph*{Формирование плана эксперимента.}

Определяется число экспериментов $N_{exp}$ по формуле:
\begin{equation}
    N_{exp} = 10 \cdot \dim(\mathbf{P})
\end{equation}
где $\dim(\mathbf{P})$ -- размерность вектора варьируемых параметров.

Формируется план эксперимента $X = \{x_1, x_2, \ldots, x_{N_{exp}}\}$ с использованием метода латинского
гиперкуба (Latin Hypercube Sampling), обеспечивающего равномерное покрытие пространства параметров при минимальном числе экспериментов.

\paragraph*{Проведение численного моделирования.}

Для каждого набора параметров $x_i$ из плана эксперимента выполняются следующие действия:
\begin{enumerate}
    \item Установка параметров структуры управления $\mathbf{p} = x_i$.
    \item Проведение численного моделирования с использованием модели $M$ и параметров $\mathbf{p}$.
    \item Расчет значений критериев оптимизации $\mathbf{j} = \{AC, ITAE, SI\}$.
    \item Сохранение результата в массив $Y$.
\end{enumerate}

В результате формируется массив пар «параметры-критерии» $D = \{(x_i, Y_i)\}_{i=1}^{N_{exp}}$,
являющийся основой для построения суррогатной модели.

\paragraph*{Построение суррогатной модели.}

На основе собранных данных строится суррогатная модель $SM$, аппроксимирующая зависимость критериев
оптимизации от параметров управляющей структуры. В качестве суррогатной модели
используются искусственные нейронные сети с архитектурой глубокого обучения.

Процесс построения модели включает следующие подэтапы:
\begin{enumerate}
    \item Нормализация входных и выходных данных.
    \item Разделение данных на обучающую и валидационную выборки.
    \item Подбор гиперпараметров модели с использованием байесовской оптимизации.
    \item Обучение модели на обучающей выборке.
\end{enumerate}

\paragraph*{Проверка адекватности суррогатной модели.}

Для оценки качества построенной суррогатной модели используется метод
кросс-валидации. Вычисляется среднеквадратичная ошибка на тестовой выборке:
\begin{equation}
    RMSE = \sqrt{\frac{1}{N_{test}} \sum_{i=1}^{N_{test}} \| SM(x_i) - Y_i \|_2^2}.
\end{equation}

Если $RMSE > \varepsilon_{surr}$, то выполняются следующие действия:
\begin{enumerate}
    \item Увеличение размера обучающей выборки путем дополнительных экспериментов.
    \item Корректировка структуры модели (числа слоев, нейронов и т.д.).
    \item Возврат к этапу построения суррогатной модели.
\end{enumerate}

\paragraph*{Многокритериальная оптимизация.}

Для поиска Парето-оптимальных решений используется алгоритм NSGA-III (Non-dominated Sorting Genetic Algorithm).
Процесс оптимизации включает следующие этапы:
\begin{enumerate}
    \item Инициализация популяции: $\mathcal{P} = \{p_1, p_2, \ldots, p_{N_{pop}}\}.$
    \item На протяжении $N_{gen}$ поколений:
          \begin{enumerate}
              \item Оценка целевых функций для каждой особи с использованием суррогатной модели $SM$: $\mathbf{J}(p_i).$
              \item Выполнение недоминируемой сортировки популяции.
              \item Расчет краудинг-дистанций для сохранения разнообразия решений.
              \item Отбор лучших решений для следующего поколения.
              \item Применение операторов скрещивания и мутации для создания новых решений.
          \end{enumerate}
\end{enumerate}

\paragraph*{Формирование фронта Парето.}

Из последнего поколения эволюционного алгоритма выбираются
недоминируемые решения, формирующие фронт Парето:
\begin{equation}
    \mathcal{PF}_{S} = \{p \in \mathcal{P} \mid p \text{ недоминируемо}\}
\end{equation}

Дополнительно выполняются следующие операции:
\begin{enumerate}
    \item Верификация полученных решений на полной модели $M$ для исключения возможных ошибок суррогатной модели.
    \item Упорядочение решений для обеспечения гладкости фронта.
    \item Удаление дубликатов и близко расположенных решений.
\end{enumerate}

В результате выполнения данного алгоритма для каждой исследуемой управляющей
структуры пневмопривода формируется соответствующий фронт Парето, отражающий
компромисс между критериями оптимизации $AC$ (точность позиционирования), $ITAE$
(интегральный критерий качества) и $SI$ (интенсивность переключений).

\section*{Алгоритм решения обратной задачи оптимизации}

Алгоритм решения обратной задачи оптимизации направлен на определение
оптимальных параметров выбранной управляющей структуры на основе
заданных требований к критериям оптимизации.

\textbf{Входные данные алгоритма:}
\begin{itemize}
    \item Множество фронтов Парето $\{\mathcal{PF}_{S_1}, \mathcal{PF}_{S_2}, \ldots, \mathcal{PF}_{S_n}\}$ для различных управляющих структур.
    \item Целевые значения критериев $\mathbf{J}_{target} = \{AC_{target}, ITAE_{target}, SI_{target}\}$.
    \item Весовые коэффициенты $\mathbf{w} = \{w_{AC}, w_{ITAE}, w_{SI}\}$, отражающие приоритеты критериев.
\end{itemize}

\textbf{Выходные данные алгоритма:} оптимальная управляющая структура $S^*$ и вектор оптимальных параметров $\mathbf{p}^*$.

\textbf{Алгоритм решения обратной задачи оптимизации:}

\paragraph*{Нормализация критериев.}

Для обеспечения сопоставимости критериев с различной
физической природой и масштабами выполняется их нормализация:

\begin{enumerate}
    \item Для каждого фронта Парето $\mathcal{PF}_{S_i}$ определяются минимальные и максимальные значения критериев:
          \begin{equation*}
              AC_{min}^i, AC_{max}^i, ITAE_{min}^i, ITAE_{max}^i, SI_{min}^i, SI_{max}^i.
          \end{equation*}

    \item Вычисляются глобальные минимумы и максимумы по всем фронтам:
          \begin{equation*}
              \begin{aligned}
                  AC_{min}   & = \min_i AC_{min}^i,   & AC_{max}   & = \max_i AC_{max}^i ,  \\
                  ITAE_{min} & = \min_i ITAE_{min}^i, & ITAE_{max} & = \max_i ITAE_{max}^i ,\\
                  SI_{min}   & = \min_i SI_{min}^i,   & SI_{max}   & = \max_i SI_{max}^i.
              \end{aligned}
          \end{equation*}

    \item Нормализуются целевые значения критериев:
          \begin{equation*}
              \begin{aligned}
                  AC_{target}^{norm}   & = \frac{AC_{target} - AC_{min}}{AC_{max} - AC_{min}}         \\
                  ITAE_{target}^{norm} & = \frac{ITAE_{target} - ITAE_{min}}{ITAE_{max} - ITAE_{min}} \\
                  SI_{target}^{norm}   & = \frac{SI_{target} - SI_{min}}{SI_{max} - SI_{min}}
              \end{aligned}
          \end{equation*}
\end{enumerate}

\paragraph*{Поиск ближайшей точки на каждом фронте Парето.}

Для каждого фронта Парето $\mathcal{PF}_{S_i}$ выполняются следующие действия:

\begin{enumerate}
    \item Инициализация минимального расстояния $d_{min}^i = \infty$
    \item Инициализация оптимальных параметров $\mathbf{p}_{opt}^i = \emptyset$ и значений критериев $\mathbf{J}_{opt}^i = \emptyset$

    \item Для каждой точки $p \in \mathcal{PF}_{S_i}$ с соответствующими значениями критериев $\mathbf{J}_p = \{AC_p, ITAE_p, SI_p\}$:
          \begin{enumerate}
              \item Нормализация значений критериев:
                    \begin{equation*}
                        \begin{aligned}
                            AC_p^{norm}   & = \frac{AC_p - AC_{min}}{AC_{max} - AC_{min}}     ,    \\
                            ITAE_p^{norm} & = \frac{ITAE_p - ITAE_{min}}{ITAE_{max} - ITAE_{min}} ,\\
                            SI_p^{norm}   & = \frac{SI_p - SI_{min}}{SI_{max} - SI_{min}}.
                        \end{aligned}
                    \end{equation*}

              \item Вычисление взвешенного расстояния:
                    \begin{equation*}
                        d = \sqrt{
                            \begin{aligned}
                                  & w_{AC} \cdot (AC_p^{norm} - AC_{target}^{norm})^2 +       \\
                                + & w_{ITAE} \cdot (ITAE_p^{norm} - ITAE_{target}^{norm})^2 + \\
                                + & w_{SI} \cdot (SI_p^{norm} - SI_{target}^{norm})^2.
                            \end{aligned}
                        }
                    \end{equation*}

              \item Если $d < d_{min}^i$, то:
                    \begin{equation*}
                        \begin{aligned}
                            d_{min}^i          & = d                                  ,         \\
                            \mathbf{p}_{opt}^i & = \text{параметры, соответствующие точке $p$}, \\
                            \mathbf{J}_{opt}^i & = \mathbf{J}_p.
                        \end{aligned}
                    \end{equation*}
          \end{enumerate}

    \item Сохранение результата: $distances = distances \cup \{(S_i, d_{min}^i, \mathbf{p}_{opt}^i, \mathbf{J}_{opt}^i)\}.$
\end{enumerate}

\paragraph*{Выбор оптимальной структуры.}

На основе полученных расстояний определяется оптимальная управляющая структура:

\begin{enumerate}
    \item Сортировка массива $distances$ по возрастанию $d_{min}^i$
    \item Выбор структуры $S^*$ с минимальным расстоянием:
          \begin{equation}
              S^* = \arg\min_{S_i} d_{min}^i
          \end{equation}
    \item Определение оптимальных параметров $\mathbf{p}^*$ для структуры $S^*$
\end{enumerate}

\section*{Практическое применение методики}

Предложенная методика структурно-параметрического синтеза позиционного
пневмопривода с дискретными распределителями имеет следующие практические преимущества:

\begin{enumerate}
    \item Позволяет выявить предельные возможности различных структур
	управления по критериям точности позиционирования, качества
	переходных процессов и интенсивности переключений распределителей.
    
    \item Дает возможность сформировать карту предпочтительных областей
	применения различных структур управления, что существенно упрощает
	процесс выбора оптимальной структуры на этапе предварительного проектирования.
    
    \item Обеспечивает нахождение оптимальных параметров выбранной структуры
	управления на основе заданных требований к системе без необходимости
	проведения дополнительных экспериментов.
    
    \item Сокращает время проектирования за счет использования суррогатных
	моделей, позволяющих быстро оценивать эффективность различных вариантов
	структур и параметров.
    
    \item Дает возможность учитывать конфликтующие требования к системе
	и находить компромиссные решения, удовлетворяющие конкретным
	практическим задачам.
\end{enumerate}

Применение данной методики при проектировании позиционных пневмоприводов с
дискретными распределителями позволяет достичь оптимального соотношения между
точностью позиционирования, быстродействием и ресурсом системы в соответствии
с требованиями конкретного технологического процесса без необходимости проведения
трудоемких экспериментальных исследований.

% \section*{Алгоритм построения фронтов Парето (этап 1)}

% Первый этап методики заключается в построении фронтов Парето для различных управляющих структур пневмопривода с целью выявления взаимосвязей между критериями оптимизации. Алгоритм данного этапа представлен ниже.

% Входными данными алгоритма являются:
% \begin{itemize}
% 	\item Управляющая структура пневмопривода $S$
% 	\item Множество варьируемых параметров $\mathbf{P}$
% 	\item Диапазоны варьирования параметров $[\mathbf{P}_{\min}, \mathbf{P}_{\max}]$
% 	\item Математическая модель пневмопривода $M$
% 	\item Критерии оптимизации $\mathbf{J} = \{AC, ITAE, SI\}$
% 	\item Требования к точности суррогатной модели $\varepsilon_{surr}$
% \end{itemize}

% Выходными данными алгоритма является фронт Парето $\mathcal{PF}_{S}$ для управляющей структуры $S$.

% \textbf{Алгоритм построения фронтов Парето состоит из следующих шагов:}

% \textbf{Шаг 1. Формирование плана эксперимента.}
% \vspace{0.5em}

% Определяется число экспериментов $N_{exp}$ по формуле:
% \begin{equation}
% 	N_{exp} = 10 \cdot \dim(\mathbf{P})
% \end{equation}
% где $\dim(\mathbf{P})$ -- размерность вектора варьируемых параметров.

% Формируется план эксперимента $X = \{x_1, x_2, \ldots, x_{N_{exp}}\}$ с использованием метода латинского гиперкуба (Latin Hypercube Sampling), обеспечивающего равномерное покрытие пространства параметров при минимальном числе экспериментов.

% \textbf{Шаг 2. Проведение численного моделирования.}
% \vspace{0.5em}

% Для каждого набора параметров $x_i$ из плана эксперимента выполняются следующие действия:
% \begin{enumerate}
% 	\item Установка параметров структуры управления $\mathbf{p} = x_i$
% 	\item Проведение численного моделирования с использованием модели $M$ и параметров $\mathbf{p}$
% 	\item Расчет значений критериев оптимизации $\mathbf{j} = \{AC, ITAE, SI\}$
% 	\item Сохранение результата в массив $Y$
% \end{enumerate}

% В результате формируется массив пар «параметры-критерии» $D = \{(x_i, Y_i)\}_{i=1}^{N_{exp}}$, являющийся основой для построения суррогатной модели.

% \textbf{Шаг 3. Построение суррогатной модели.}
% \vspace{0.5em}

% На основе собранных данных строится суррогатная модель $SM$, аппроксимирующая зависимость критериев оптимизации от параметров управляющей структуры. В качестве суррогатной модели могут использоваться:
% \begin{itemize}
% 	\item Искусственные нейронные сети с архитектурой глубокого обучения
% 	\item Модели на основе гауссовских процессов
% 	\item Ансамбли регрессионных деревьев
% \end{itemize}

% Процесс построения модели включает следующие подэтапы:
% \begin{enumerate}
% 	\item Нормализация входных и выходных данных
% 	\item Разделение данных на обучающую и валидационную выборки
% 	\item Подбор гиперпараметров модели с использованием байесовской оптимизации
% 	\item Обучение модели на обучающей выборке
% \end{enumerate}

% \textbf{Шаг 4. Проверка адекватности суррогатной модели.}
% \vspace{0.5em}

% Для оценки качества построенной суррогатной модели используется метод кросс-валидации. Вычисляется среднеквадратичная ошибка на тестовой выборке:
% \begin{equation}
% 	RMSE = \sqrt{\frac{1}{N_{test}} \sum_{i=1}^{N_{test}} \| SM(x_i) - Y_i \|_2^2}
% \end{equation}

% Если $RMSE > \varepsilon_{surr}$, то выполняются следующие действия:
% \begin{enumerate}
% 	\item Увеличение размера обучающей выборки путем дополнительных экспериментов
% 	\item Корректировка структуры модели (числа слоев, нейронов и т.д.)
% 	\item Возврат к шагу 3
% \end{enumerate}

% \textbf{Шаг 5. Многокритериальная оптимизация.}
% \vspace{0.5em}

% Для поиска Парето-оптимальных решений используется алгоритм NSGA-II или NSGA-III (Non-dominated Sorting Genetic Algorithm). Процесс оптимизации включает следующие этапы:
% \begin{enumerate}
% 	\item Инициализация популяции: $\mathcal{P} = \{p_1, p_2, \ldots, p_{N_{pop}}\}$
% 	\item На протяжении $N_{gen}$ поколений:
% 	      \begin{enumerate}
% 		      \item Оценка целевых функций для каждой особи с использованием суррогатной модели $SM$: $\mathbf{J}(p_i)$
% 		      \item Выполнение недоминируемой сортировки популяции
% 		      \item Расчет краудинг-дистанций для сохранения разнообразия решений
% 		      \item Отбор лучших решений для следующего поколения
% 		      \item Применение операторов скрещивания и мутации для создания новых решений
% 	      \end{enumerate}
% \end{enumerate}

% \textbf{Шаг 6. Формирование фронта Парето.}
% \vspace{0.5em}

% Из последнего поколения эволюционного алгоритма выбираются недоминируемые решения, формирующие фронт Парето:
% \begin{equation}
% 	\mathcal{PF}_{S} = \{p \in \mathcal{P} \mid p \text{ недоминируемо}\}
% \end{equation}

% Дополнительно выполняются следующие операции:
% \begin{enumerate}
% 	\item Верификация полученных решений на полной модели $M$ для исключения возможных ошибок суррогатной модели
% 	\item Упорядочение решений для обеспечения гладкости фронта
% 	\item Удаление дубликатов и близко расположенных решений
% \end{enumerate}

% В результате выполнения данного алгоритма для каждой исследуемой управляющей структуры пневмопривода формируется соответствующий фронт Парето,
% отражающий компромисс между критериями оптимизации $AC$ (точность позиционирования),
% $ITAE$ (интегральный критерий качества) и $SI$ (интенсивность переключений).

% \section*{Алгоритм решения обратной задачи оптимизации (этап 2)}

% Второй этап методики предполагает решение обратной задачи оптимизации, заключающейся в определении оптимальных параметров выбранной управляющей структуры на основе заданных требований к критериям оптимизации.

% Входными данными алгоритма являются:
% \begin{itemize}
% 	\item Множество фронтов Парето $\{\mathcal{PF}_{S_1}, \mathcal{PF}_{S_2}, \ldots, \mathcal{PF}_{S_n}\}$ для различных управляющих структур
% 	\item Целевые значения критериев $\mathbf{J}_{target} = \{AC_{target}, ITAE_{target}, SI_{target}\}$
% 	\item Весовые коэффициенты $\mathbf{w} = \{w_{AC}, w_{ITAE}, w_{SI}\}$, отражающие приоритеты критериев
% \end{itemize}

% Выходными данными алгоритма являются оптимальная управляющая структура $S^*$ и вектор оптимальных параметров $\mathbf{p}^*$.

% \textbf{Алгоритм решения обратной задачи оптимизации состоит из следующих шагов:}

% \textbf{Шаг 1. Нормализация критериев.}
% \vspace{0.5em}

% Для обеспечения сопоставимости критериев с различной физической природой и масштабами выполняется их нормализация:

% \begin{enumerate}
% 	\item Для каждого фронта Парето $\mathcal{PF}_{S_i}$ определяются минимальные и максимальные значения критериев:
% 	      \begin{equation*}
% 		      AC_{min}^i, AC_{max}^i, ITAE_{min}^i, ITAE_{max}^i, SI_{min}^i, SI_{max}^i
% 	      \end{equation*}

% 	\item Вычисляются глобальные минимумы и максимумы по всем фронтам:
% 	      \begin{equation*}
% 		      \begin{aligned}
% 			      AC_{min}   & = \min_i AC_{min}^i,   & AC_{max}   & = \max_i AC_{max}^i   \\
% 			      ITAE_{min} & = \min_i ITAE_{min}^i, & ITAE_{max} & = \max_i ITAE_{max}^i \\
% 			      SI_{min}   & = \min_i SI_{min}^i,   & SI_{max}   & = \max_i SI_{max}^i
% 		      \end{aligned}
% 	      \end{equation*}

% 	\item Нормализуются целевые значения критериев:
% 	      \begin{equation*}
% 		      \begin{aligned}
% 			      AC_{target}^{norm}   & = \frac{AC_{target} - AC_{min}}{AC_{max} - AC_{min}}         \\
% 			      ITAE_{target}^{norm} & = \frac{ITAE_{target} - ITAE_{min}}{ITAE_{max} - ITAE_{min}} \\
% 			      SI_{target}^{norm}   & = \frac{SI_{target} - SI_{min}}{SI_{max} - SI_{min}}
% 		      \end{aligned}
% 	      \end{equation*}
% \end{enumerate}

% \textbf{Шаг 2. Поиск ближайшей точки на каждом фронте Парето.}
% \vspace{0.5em}

% Для каждого фронта Парето $\mathcal{PF}_{S_i}$ выполняются следующие действия:

% \begin{enumerate}
% 	\item Инициализация минимального расстояния $d_{min}^i = \infty$
% 	\item Инициализация оптимальных параметров $\mathbf{p}_{opt}^i = \emptyset$ и значений критериев $\mathbf{J}_{opt}^i = \emptyset$

% 	\item Для каждой точки $p \in \mathcal{PF}_{S_i}$ с соответствующими значениями критериев $\mathbf{J}_p = \{AC_p, ITAE_p, SI_p\}$:
% 	      \begin{enumerate}
% 		      \item Нормализация значений критериев:
% 		            \begin{equation*}
% 			            \begin{aligned}
% 				            AC_p^{norm}   & = \frac{AC_p - AC_{min}}{AC_{max} - AC_{min}}         \\
% 				            ITAE_p^{norm} & = \frac{ITAE_p - ITAE_{min}}{ITAE_{max} - ITAE_{min}} \\
% 				            SI_p^{norm}   & = \frac{SI_p - SI_{min}}{SI_{max} - SI_{min}}
% 			            \end{aligned}
% 		            \end{equation*}

% 		      \item Вычисление взвешенного расстояния:
% 		            \begin{equation*}
% 			            d = \sqrt{
% 				            \begin{aligned}
% 					              & w_{AC} \cdot (AC_p^{norm} - AC_{target}^{norm})^2 +       \\
% 					            + & w_{ITAE} \cdot (ITAE_p^{norm} - ITAE_{target}^{norm})^2 + \\
% 					            + & w_{SI} \cdot (SI_p^{norm} - SI_{target}^{norm})^2
% 				            \end{aligned}
% 			            }
% 		            \end{equation*}

% 		      \item Если $d < d_{min}^i$, то:
% 		            \begin{equation*}
% 			            \begin{aligned}
% 				            d_{min}^i          & = d                                           \\
% 				            \mathbf{p}_{opt}^i & = \text{параметры, соответствующие точке $p$} \\
% 				            \mathbf{J}_{opt}^i & = \mathbf{J}_p
% 			            \end{aligned}
% 		            \end{equation*}
% 	      \end{enumerate}

% 	\item Сохранение результата: $distances = distances \cup \{(S_i, d_{min}^i, \mathbf{p}_{opt}^i, \mathbf{J}_{opt}^i)\}$
% \end{enumerate}

% \textbf{Шаг 3. Выбор оптимальной структуры.}
% \vspace{0.5em}

% На основе полученных расстояний определяется оптимальная управляющая структура:

% \begin{enumerate}
% 	\item Сортировка массива $distances$ по возрастанию $d_{min}^i$
% 	\item Выбор структуры $S^*$ с минимальным расстоянием:
% 	      \begin{equation}
% 		      S^* = \arg\min_{S_i} d_{min}^i
% 	      \end{equation}
% 	\item Определение оптимальных параметров $\mathbf{p}^*$ для структуры $S^*$
% \end{enumerate}
