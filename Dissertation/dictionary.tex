\chapter*{Словарь терминов}             % Заголовок
\addcontentsline{toc}{chapter}{Словарь терминов}  % Добавляем его в оглавление

\textbf{Адиабатический процесс} : Термодинамический процесс, происходящий без теплообмена системы с окружающей средой.

\textbf{Векторное поле} : Математический объект, описывающий направления и интенсивность движения системы в фазовом пространстве.

\textbf{Горизонт прогноза} : Интервал времени в будущем, на который выполняется прогнозирование поведения системы в методе прогнозного управления.

\textbf{Горизонт управления} : Количество шагов управляющих воздействий, оптимизируемых одновременно в методе прогнозного управления.

\textbf{Гистерезис} : Явление, при котором физическая система неоднозначно зависит от своей предыстории (выходная величина зависит не только от входной, но и от предыдущего состояния системы).

\textbf{Дефаззификация} : Процесс преобразования нечеткого множества или нечеткой величины в четкое (конкретное) значение.

\textbf{Закон Сен-Венана--Ванцеля} : Закон, описывающий расход газа через отверстие или канал в зависимости от отношения давлений.

\textbf{Запирание полости} : Состояние, при котором все распределители, соединенные с полостью пневмопривода, закрыты, что приводит к изоляции этой полости от системы питания и атмосферы.

\textbf{Интенсивность переключений} : Критерий оптимизации, характеризующий частоту и количество переключений распределителей за цикл работы.

\textbf{Колебательность} : Характеристика системы, отражающая склонность к осцилляциям вокруг положения равновесия.

\textbf{Критическое отношение давлений} : Пороговое значение отношения давлений, при котором происходит переход от докритического к закритическому течению газа.

\textbf{Лингвистическая переменная} : Переменная, значениями которой являются не числа, а слова или предложения естественного или формального языка.

\textbf{Метод центра тяжести} : Способ дефаззификации в нечеткой логике, при котором четкое значение определяется как центр тяжести площади под кривой результирующей функции принадлежности.

\textbf{Многокритериальная оптимизация} : Процесс нахождения решения, оптимального сразу по нескольким противоречивым критериям.

\textbf{Многорежимное управление} : Подход к управлению, при котором система имеет несколько различных режимов работы с возможностью переключения между ними.

\textbf{Поверхность скольжения} : Многообразие в пространстве состояний, определяющее динамику системы при управлении в скользящих режимах.

\textbf{Прогнозная модель} : Упрощенная математическая модель объекта, используемая для прогнозирования его будущего поведения в прогнозном управлении.

\textbf{Парето-оптимальность} : Состояние системы, при котором улучшение по одному критерию невозможно без ухудшения по другому критерию.

\textbf{Парето-фронт} : Множество всех Парето-оптимальных решений в пространстве критериев оптимизации.

\textbf{Переменная структура} : Свойство системы, при котором её структура (или закон управления) меняется в зависимости от состояния системы.

\textbf{Перерегулирование} : Явление, при котором выходная величина системы превышает установившееся значение при переходном процессе.

\textbf{Политропный процесс} : Термодинамический процесс, при котором теплоемкость системы остается постоянной.

\textbf{Пространство состояний} : Математическое пространство, где каждая точка однозначно определяет состояние динамической системы.

\textbf{Расходная функция} : Функция, описывающая зависимость массового расхода через дросселирующее устройство от отношения давлений.

\textbf{Релейная система} : Система управления, в которой управляющее воздействие может принимать только дискретные значения.

\textbf{Режим торможения} : Специальный режим работы пневмопривода, направленный на эффективное снижение скорости движения.

\textbf{Скользящий режим} : Режим движения системы вдоль заданной поверхности переключения в пространстве состояний системы.

\textbf{Статическая ошибка} : Ошибка, возникающая в системе автоматического управления в установившемся режиме работы.

\textbf{Суррогатное моделирование} : Метод создания быстрых приближенных моделей сложных систем для использования в задачах оптимизации.

\textbf{Терм} : Элемент терм-множества лингвистической переменной, представляющий определенное значение этой переменной.

\textbf{Термодинамический процесс} : Изменение состояния термодинамической системы, сопровождающееся изменением ее термодинамических параметров.

\textbf{Терминальная поверхность} : Тип поверхности скольжения, обеспечивающий достижение целевого состояния за конечное время.

\textbf{Трение LuGre} : Модель трения, учитывающая микроскопические деформации контактирующих поверхностей, эффект Штрибека и вязкое трение.

\textbf{Уравнение Максвелла} : Дифференциальное уравнение, связывающее изменение термодинамических функций состояния с изменением параметров системы.

\textbf{Упреждающее управление} : Стратегия управления, при которой управляющее воздействие формируется с учетом прогноза будущего поведения системы.

\textbf{Фазовый портрет} : Графическое представление траекторий динамической системы в фазовой плоскости или фазовом пространстве.

\textbf{Фазовое пространство} : Пространство, образованное совокупностью всех возможных состояний динамической системы.

\textbf{Фаззификация} : Процесс преобразования четких (конкретных) значений в степени принадлежности к нечетким множествам.

\textbf{Фронт Парето} : Множество всех недоминируемых решений в задаче многокритериальной оптимизации.

\textbf{Функция принадлежности} : Функция, определяющая степень принадлежности элемента к нечеткому множеству.

\textbf{Эффект Дала} : Микроскопическое смещение объекта до начала макроскопического движения, учитываемое в моделях трения.

\textbf{Эффективная площадь} : Условная площадь проходного сечения распределителя, используемая для расчета массового расхода.

\textbf{Эвристический алгоритм} : Приближенный метод решения задач, не гарантирующий нахождение оптимального решения, но дающий приемлемое решение за разумное время.
%\textbf{AC} : Accuracy -- критерий точности позиционирования, определяемый как модуль ошибки в установившемся режиме.
%\textbf{BDF} : Backward Differential Formula -- метод обратных дифференциальных формул для численного решения жестких систем дифференциальных уравнений.
%\textbf{ITAE} : Integral Time Absolute Error -- интегральный критерий качества переходного процесса с учетом времени.
%\textbf{MPC} : Model Predictive Control — прогнозное управление с использованием модели объекта.
%\textbf{ПИД} : Пропорционально-интегрально-дифференциальный (регулятор) -- тип регулятора, использующий трехкомпонентный закон управления.
%\textbf{РО} : Рабочий орган -- исполнительный механизм, непосредственно выполняющий технологические операции.
%\textbf{SI} : Switching Intensity -- критерий интенсивности переключений распределителей.
%\textbf{УСР-И-3} : Управление в скользящих режимах с интегральной поверхностью с тремя режимами работы.
%\textbf{УСР-И-5} : Управление в скользящих режимах с интегральной поверхностью с пятью режимами работы.
%\textbf{УСР-И-7} : Управление в скользящих режимах с интегральной поверхностью с семью режимами работы.
%\textbf{УСР-Т-3} : Управление в скользящих режимах с терминальной поверхностью с тремя режимами работы.
%\textbf{УСР-Т-5} : Управление в скользящих режимах с терминальной поверхностью с пятью режимами работы.
%\textbf{УСР-Т-7} : Управление в скользящих режимах с терминальной поверхностью с семью режимами работы.
%\textbf{ШИМ} : Широтно-импульсная модуляция -- способ управления мощностью путем изменения скважности импульсов постоянной частоты.