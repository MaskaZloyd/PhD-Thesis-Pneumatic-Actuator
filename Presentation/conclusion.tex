\begin{frame}{Основные результаты исследования}
	\begin{enumerate}
		\item \scriptsize Разработан комплекс математических моделей различных структур управления пневмоприводом с дискретными распределителями, включая модифицированный
		      ПИД-регулятор с ШИМ, многорежимное скользящее, нечеткое и прогнозное управление.
		\item \scriptsize Создана математическая модель силовой структуры
		      пневмопривода, учитывающая нелинейные динамические эффекты термодинамических процессов и трения.
		\item \scriptsize Выявлены оптимальные режимы переключения распределителей на
		      основе анализа фазовых портретов и термодинамических процессов в полостях пневмоцилиндра.
		\item \scriptsize Разработана методика структурно-параметрического синтеза
		      на основе многокритериальной оптимизации параметров различных структур управления.
		\item \scriptsize Построены фронты Парето для различных алгоритмов управления, определяющие
		      взаимосвязи между точностью позиционирования, динамическими характеристиками и интенсивностью переключений распределителей.
		\item \scriptsize Определены предпочтительные области применения исследованных
		      алгоритмов управления в зависимости от приоритетных требований к системе.
		\item \scriptsize Экспериментально подтверждена эффективность разработанных алгоритмов и
		      методик на специализированном лабораторном стенде.
		\item \scriptsize Сформированы практические рекомендации по выбору оптимальных структур и
		      параметров пневмопривода для различных условий эксплуатации.

	\end{enumerate}

	% \vspace{0.1em}

	% \begin{center}
	% 	\begin{tikzpicture}
	% 		\node[draw, rounded corners, fill=blue!10, text width=10cm, align=center] {
	% 			Разработанные алгоритмы и методика синтеза обеспечивают комплексное повышение статико-динамических и ресурсных показателей пневмопривода и рекомендуются к внедрению в системах промышленной автоматизации
	% 		};
	% 	\end{tikzpicture}
	% \end{center}
\end{frame}

\begin{frame}
	\frametitle{Научная новизна}
	\begin{itemize}

		\item \scriptsize Предложена модифицированная структура ПИД-регулятора, включающая блок прогнозирования тормозного пути и
		      формирование упреждающего управляющего воздействия, что обеспечивает адаптивное торможение и существенно снижает
		      перерегулирование и колебания системы по сравнению с классической схемой ПИД-регулятора с ШИМ.

		\item \scriptsize Разработан способ прогнозного управления пневмоприводом с дискретными распределителями,
		      основанный на оптимизации последовательности переключений распределителей.

		\item \scriptsize Предложена двухуровневая методика структурно-параметрического синтеза, где
		      на первом этапе с использованием суррогатного моделирования формируются фронты Парето,
		      отражающие компромисс между статико-динамическими и ресурсными показателями, а на
		      втором этапе по данным фронтам определяется оптимальный набор параметров структуры управления.

		\item \scriptsize Получены фронты Парето для различных управляющих структур пневмопривода с дискретными распределителями,
		      позволяющие определить количественные соотношения между точностью позиционирования, динамическими
		      характеристиками и интенсивностью переключений распределителей, что обеспечивает научно обоснованный
		      выбор предпочтительных областей применения каждого алгоритма управления в зависимости
		      от приоритетных требований к системе и характера решаемых задач.

		\item \scriptsize На основе полученных фронтов Парето выделены предпочтительные области применения различных
		      структур управления пневмоприводом, что позволило сформировать практические рекомендации по выбору
		      оптимальной структуры устройства.
	\end{itemize}
\end{frame}
\note{
	Проговаривается вслух научная новизна
}

% \begin{frame}
%     \frametitle{Научная и практическая значимость}
%     \begin{itemize}
%         \item Получены выражения для \dots.
%         \item Определены условия \dots.
%         \item Разработаны устройства \dots.
%     \end{itemize}
% \end{frame}
% \note{
%     Проговариваются вслух научная и практическая значимость
% }

% \begin{frame}
%     \frametitle{Свидетельство о регистрации программы}
%     \begin{figure}[h]
%         \centering
%         \includegraphics[height=0.7\textheight]{registration}
%     \end{figure}
% \end{frame}
% \note{
%     Получено свидетельство о регистрации разработанной программы \textsc{Hello~world™}.
% }

% \begin{frame}
%     \frametitle{Акт о внедрении}
%     \begin{figure}[h]
%         \centering
%         \fbox{
%             \begin{minipage}[t]{0.4\linewidth}
%                 \includegraphics[width=\linewidth]{implementation}
%             \end{minipage}
%         }
%     \end{figure}
% \end{frame}
% \note{
%     Получен акт о внедрении.
% }

\begin{frame} % публикации на одной странице
	% \begin{frame}[t,allowframebreaks] % публикации на нескольких страницах
	\frametitle{Основные публикации}
	\nocite{vakbib1}%
	\nocite{vakbib2}%
	%
	%% authorwos
	\nocite{wosbib1}%
	%
	%% authorscopus
	\nocite{scbib1}%
	%
	%% authorconf
	\nocite{confbib1}%
	\nocite{confbib2}%
	%
	%% authorother
	\nocite{bib1}%
	\nocite{bib2}%
	\ifnumequal{\value{bibliosel}}{0}{
		\insertbiblioauthor
	}{
		\printbibliography%
	}
\end{frame}
\note{
	Результаты работы опубликованы в N печатных изданиях,
	в~т.\:ч. M реферируемых изданиях.
}

% \begin{frame}
%     \frametitle{Участие в конференциях}
%     \begin{itemize}
%         \item Научная сессия МГУ, Москва 2013--2015;
%         \item \rom{24} Russian Conference (RuC 2014), Obninsk, Russia, 2014
%         \item \rom{7} International Conference (IAC 16), Busan, Korea,
%               2016;
%         \item \rom{28} Other Conference (AC 16), East Lansing, MI USA, 2016;
%         \item \dots
%     \end{itemize}
% \end{frame}
% \note{
%     Работа была представлена на ряде конференций.
% }

\begin{frame}[plain, noframenumbering] % последний слайд без оформления
	\begin{center}
		\Huge
		Спасибо за внимание!
	\end{center}
\end{frame}
