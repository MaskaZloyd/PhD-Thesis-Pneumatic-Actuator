% Общие стили оформления.
% Возможные варианты значений ищите в описании библиотеки beamer
% \usetheme{Pittsburgh}
% \usecolortheme{whale}

\usetheme[secheader]{Boadilla}
\usecolortheme{seahorse}
% \usecolortheme{}

% Размер полей слайдов
\setbeamersize{text margin left=1cm,%
               text margin right=1cm}

% выключение кнопок навигации
\beamertemplatenavigationsymbolsempty

% Размеры шрифтов
\setbeamerfont{title}{size=\large}
\setbeamerfont{subtitle}{size=\small}
\setbeamerfont{author}{size=\normalsize}
\setbeamerfont{institute}{size=\small}
\setbeamerfont{date}{size=\normalsize}
\setbeamerfont{bibliography item}{size=\small}
\setbeamerfont{bibliography entry author}{size=\small}
\setbeamerfont{bibliography entry title}{size=\small}
\setbeamerfont{bibliography entry location}{size=\small}
\setbeamerfont{bibliography entry note}{size=\small}
% Аналогично можно настроить и другие размеры.
% Названия классов элементов можно найти здесь
% http://www.cpt.univ-mrs.fr/~masson/latex/Beamer-appearance-cheat-sheet.pdf

% Цвет элементов
\setbeamercolor{footline}{fg=blue}
\setbeamercolor{bibliography item}{fg=black}
\setbeamercolor{bibliography entry author}{fg=black}
\setbeamercolor{bibliography entry title}{fg=black}
\setbeamercolor{bibliography entry location}{fg=black}
\setbeamercolor{bibliography entry note}{fg=black}
% Аналогично можно настроить и другие цвета.
% Названия классов элементов можно найти здесь
% http://www.cpt.univ-mrs.fr/~masson/latex/Beamer-appearance-cheat-sheet.pdf

% Нумеровать список статей
% https://tex.stackexchange.com/a/419506/104425
\setbeamertemplate{bibliography item}{\insertbiblabel}
% или убрать номера
% \setbeamertemplate{bibliography item}{}

% Использовать шрифт с засечками для формул
% https://tex.stackexchange.com/a/34267/104425
\usefonttheme[onlymath]{serif}

% https://tex.stackexchange.com/a/291545/104425
\makeatletter
\def\beamer@framenotesbegin{% at beginning of slide
    \usebeamercolor[fg]{normal text}
    \gdef\beamer@noteitems{}%
    \gdef\beamer@notes{}%
}
\makeatother

% footer презентации
\setbeamertemplate{footline}{
    \leavevmode%
    \hbox{%
        \begin{beamercolorbox}[wd=.333333\paperwidth,ht=2.25ex,dp=1ex,center]{}%
            % И. О. Фамилия, Организация кратко
            \thesisAuthorShort, \thesisOrganizationShort
        \end{beamercolorbox}%
        \begin{beamercolorbox}[wd=.333333\paperwidth,ht=2.25ex,dp=1ex,center]{}%
            % Город, 20XX
            \thesisCity, \thesisYear
        \end{beamercolorbox}%
        \begin{beamercolorbox}[wd=.333333\paperwidth,ht=2.25ex,dp=1ex,right]{}%
            Стр. \insertframenumber{} из \inserttotalframenumber \hspace*{2ex}
        \end{beamercolorbox}}%
    \vskip0pt%
}

% вывод на экран заметок к презентации
\ifnumequal{\value{presnotes}}{0}{}{
    \setbeameroption{show notes}
    \ifnumequal{\value{presnotes}}{2}{
        \setbeameroption{show notes on second screen=\presposition}
    }{}
}
